%!TEX root = _main.tex
\clearpage
\addcontentsline{toc}{chapter}{Résumé}
\adjustmtc
\section*{Résumé}

L’évolution technologique du web durant ces dernières années a favorisé l’arrivée d’environnements virtuels collaboratifs pour la modélisation 3D à grande échelle. Alors que la collaboration réuni dans un même espace partagé des utilisateurs distants géographiquement pour un objectif de collaboration commun, les ressources qu'ils apportent (calcul, stockage ...) sont encore trop rarement utilisées et cela constitue un défi. Il s'agit en effet de proposer un système simple, efficace et transparent pour les utilisateurs afin de permettre une collaboration efficace à la fois sur le volet computationnel mais aussi, bien entendu, sur l'aspect métier lié à la modélisation et la visualisation 3D dans un environnement hétérogènes en termes de performances de calcul, de rendu et de connexion.
Pour rendre efficace le passage à l’échelle, en conservant une source de vérité centralisée, de nombreux systèmes utilisent une architecture réseau dite "hybride", combinant client serveur et pair-à-pair. Cependant, la synchronicité élevée et la réplication des données sur tous les sites peut mener à une divergence des copies dans un environnement réparti. C’est pourquoi il est nécessaire de s’intéresser à la réplication optimiste adaptée aux propriétés ces environnements collaboratifs 3D : la dynamicité des utilisateurs et leur nombre, le type de donnée traitées (3D) et leur masse. Le cadre d’un système collaboratif impose également la conservation des propriétés de Causalité, Convergence et préservation de l’Intention (CCI).

Cette thèse présente un modèle pour les systèmes d’édition collaborative en 3D dans un environnement web. Dans ce modèle est intégrée une architecture cliente (3DEvent) qui permet de déporter les aspects métiers de la 3D au plus près de l’utilisateur sous la forme d’évènements. La mise en place de cette architecture basée-évènements repose sur le constat d’un fort besoin de traçabilité et d’historique sur les données 3D lors de l’assemblage d’un modèle. Cet aspect est porté intrinsèquement par le patron de conception event-sourcing. Ce modèle est complété par la définition d’un intergiciel en pair-à-pair. Sur ce dernier point, nous proposons d'utiliser une technologie récente comme WebRTC qui présente une API familière aux développeurs de services en infonuagique. Une évaluation portant sur deux études utilisateur concernant l’acceptance du modèle proposé a été menée dans le cadre de tâches d’assemblage de modèles 3D sur plusieurs groupes d’utilisateurs.

\paragraph{Mot clés : }environnement virtuel collaboratif, réseau pair-à-pair, WebRTC, web 3D, conception 3D, traitement réparti évènementiel, architecture hybride, event-sourcing.
\pagebreak
\section*{Abstract}

Web technologies evolutions during last decades fostered the development of 
collaborative virtual environments for 3D design at large scale. Despite the fact 
that collaborative environments gather in a same shared space geographically 
distant users in a common objective, the ressources of their clients (calcul, 
storage ...) are often underused because of the challenge it represents. It is indeed 
a matter of offering an easy-to-use, efficient and transparent collaborative system 
to the user supporting both computationnal and 3D design visualisation and 
business logic needs in heterogeneous environments in terms of computing, 
rendering and connexion performances. To scale well while conserving a 
centralised authoritative source of data, numerous systems use a network 
architecture called "hybrid", combining both client-server and peer-to-peer. 
However, real-time updates and data replication on different sites lead to 
divergence of copy in such a distributed environment. That is why optimistic 
replication is well adapted to 3D collaborative envionments by taking into account 
different parameters: the dynamicity of users and their numbers, the 3D data type 
used and the large amount and size of it. It is also imperative to respect 
collaborative system properties of Causality, Convergence and Intention 
preservation (CCI).

This document presents a model for 3D web-based collaborative editing systems. This model integrates 3DEvent, an client-based architecture allowing us to bring 3D business logic closer to the user using events. Indeed, the need of traçability and history awareness is required during 3D design especially when several experts are involved during the process. This aspect is intrinsec to event sourcing design pattern. This architecture is completed by a peer-to-peer middleware responsible for the synchronisation and the consistency of the system. To implement it, we propose to use the recent web standard API called WebRTC, close to cloud development services know by developers. To evaluate the model, two user studies were conducted on several group of users concerning its responsiveness and the acceptance by users in the frame of cooperative assembly tasks of 3D models.




\paragraph{Keywords: }collaborative virtual environment, peer-to-peer network, WebRTC, web 3D, 3D design, distributed event-based system, hybrid architecture, event-sourcing.

\pagebreak