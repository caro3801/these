%!TEX root = _main.tex
\clearpage
\pagebreak
{\centering
\section*{Résumé}}
\small 
L’évolution technologique du web durant ces dernières années a favorisé l’arrivée 
d’environnements virtuels collaboratifs pour la modélisation 3D à grande échelle. 
Alors que la collaboration réunit dans un même espace partagé des utilisateurs 
distants géographiquement pour un objectif de collaboration commun, les 
ressources 
matérielles qu'ils apportent (calcul, stockage, 3D ...) avec leurs connaissances 
sont encore trop rarement utilisées et cela constitue un défi. Il s'agit en effet 
de proposer un système simple, performant et transparent pour les utilisateurs 
afin de permettre une collaboration efficace à la fois sur le volet computationnel 
mais aussi, bien entendu, sur l'aspect métier lié à la modélisation 3D sur le web.
Pour rendre efficace le passage à l’échelle, de nombreux systèmes utilisent une 
architecture réseau dite "hybride", combinant client serveur et pair-à-pair. La 
réplication optimiste s'adapte bien aux propriétés des ces environnements répartis 
: la dynamicité des utilisateurs et leur nombre, le type de donnée traitées (3D) 
et leur taille. 

Cette thèse présente un modèle pour les systèmes d’édition collaborative en 3D 
sur le web. L'architecture cliente (3DEvent) permet de déporter les 
aspects métiers de la 3D au plus près de l’utilisateur sous la forme d’évènements. 
Cette architecture orientée événements repose sur le constat 
d’un fort besoin de traçabilité et d’historique sur les données 3D lors de 
l’assemblage d’un modèle. Cet aspect est porté intrinsèquement par le patron de 
conception event-sourcing. Ce modèle est complété par la définition d’un 
intergiciel en pair-à-pair. Sur ce dernier point, nous proposons d'utiliser la 
technologie WebRTC qui présente une API familière aux développeurs de services 
en infonuagique. Une évaluation portant sur deux études utilisateur concernant 
l’acceptance du modèle proposé a été menée dans le cadre de tâches 
d’assemblage de 
modèles 3D sur plusieurs groupes d’utilisateurs.

\textbf{Mot clés :} environnement virtuel collaboratif, réseau pair-à-pair, 
WebRTC, web 3D, conception 3D, traitement réparti évènementiel, architecture 
hybride, event-sourcing.

{\centering
\section*{Abstract}}

Web technologies evolutions during last decades fostered the development of 
collaborative virtual environments for 3D design at large scale. Despite the fact 
that collaborative environments gather in a same shared space geographically 
distant users in a common objective, the hardware ressources of their clients (
calcul, storage, graphics ...) are often underused because of the challenge it 
represents. It is indeed a matter of offering an easy-to-use, efficient and 
transparent collaborative system to the user supporting both computationnal and 
3D 
design visualisation and business logic needs in heterogeneous web 
environments. 
To scale well, numerous systems use a network architecture called "hybrid",  
combining both client-server and peer-to-peer. Optimistic 
replication is well adapted to distributed application such as 3D collaborative 
envionments : the dynamicity of users and their numbers, the 3D data type 
used and the large amount and size of it.

This document presents a model for 3D web-based collaborative editing systems. 
This model integrates 3DEvent, an client-based architecture allowing us to bring 
3D business logic closer to the user using events. Indeed, the need of traceability 
and history awareness is required during 3D design especially when several 
experts are involved during the process. This aspect is intrinsec to event-sourcing 
design pattern. This architecture is completed by a peer-to-peer middleware 
responsible for the synchronisation and the consistency of the system. To 
implement it, we propose to use the recent web standard API called WebRTC, 
close to cloud development services know by developers. To evaluate the model, 
two user studies were conducted on several group of users concerning its 
responsiveness and the acceptance by users in the frame of cooperative 
assembly tasks of 3D models.

\textbf{Keywords:} collaborative virtual environment, peer-to-peer network, 
WebRTC, Web 3D, 3D design, distributed event-based system, hybrid 
architecture, event-sourcing.