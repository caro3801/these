\appendix

	\chapter{Ressources pour l'implantation et les expérimentations}
	\section{Description des événements par agrégat dans 3DEvent}
	\begin{landscape}
\begin{table}[]
	
	\centering
	\caption{3DEvent : résumé des événements par agrégats}
	\label{tab:events}
	\begin{tabular}{lll}
		\toprule
		\textbf{Événement}& \textbf{Dénomination} & \textbf{Description} \\ \midrule
		\textbf{Agrégat Scène}     &                      &             \\ \hline
		Scène créée &  sceneCreated                  & Une scène a été 
		créée            \\
		Scène supprimée &        sceneRemoved              &  Une scène a été 
		supprimée           \\
		Scène renommée &     sceneRenamed                 &     Une scène a été 
		renommée        \\
		\textbf{Agrégat Maillage}  &                      &             \\ \hline
		Maillage ajouté &     meshAdded                 
		&  \begin{tabular}[c]{@{}l@{}} Un maillage a été ajouté dans la Scène à 
		partir \\d'une géométrie de la 
		bibliothèque \end{tabular}  \\
		Maillage déposé &     meshDropped               
		&      \begin{tabular}[c]{@{}l@{}} Un maillage a été déposé dans 
		l'environnement 3D\\ de la Scène à 
		partir d'une géométrie de la bibliothèque \end{tabular} \\
		Maillage supprimé & meshRemoved       &           Un maillage a été 
		supprimé de la Scène   \\
		Maillage translaté &   meshTranslated  	 &    Un maillage a 
		subit une translation dans la Scène         \\
		Maillage pivoté &      meshRotated                &     Un maillage a 
		subit une rotation dans la Scène              \\
		Maillage mis à l'échelle &  meshScaled           &      Un maillage a 
		subit une homothétie dans la Scène         \\
		\textbf{Agrégat Géométrie} &                      &             \\ \hline
		\begin{tabular}[c]{@{}l@{}}Géométrie importée\\dans la bibliothèque 
		\end{tabular}&  geometryImported           &        
		\begin{tabular}[c]{@{}l@{}} Une géométrie est créée à partir d'un fichier 
		importé par\\ 
		un utilisateur et ajoutée à la bibliothèque de la 
		Scène \end{tabular}     \\
		\textbf{Agrégat Utilisateur}        &                      &            \\ \hline
		Utilisateur créé&   userCreated                   &        Un utilisateur a été créé 
		dans l'application    \\
		Scène rejointe par utilisateur&     userJoinedScene                 &      Un 
		utilisateur a rejoint une scène       \\
		Scène quittée par utilisateur&        userLeftScene              &    Un utilisateur 
		a quitté une scène         \\
		Nom modifié &         usernameChanged           &     Un utilisateur a modifié 
		son nom        \\
		Couleur modifiée &       colorChanged                  &      Un utilisateur a 
		modifié son code couleur       \\ \bottomrule
	\end{tabular}
\end{table}

\end{landscape}

\section{Messages réseaux pour la synchronisation des Event Stores}
\label{annexe:protocole}
\begin{table}[h]
	\centering
	\small
	\caption{Type de messages lors de la synchronisation}
	\label{table:messagetype}
	\begin{tabular}{ll}
		\toprule
		\textbf{Message}                & \textbf{Description} \\ \hline
		STREAM\_SYNC\_ASK      &  Demande de synchronisation d'un 
		\textit{stream}           \\
		CHUNK                  &     Réception d'une donnée \textit{chunk})        
		\\
		READY\_ASK             &      Prêt pour la démarrer la demande de 
		données de 
		sync.        \\
		READY                  &       Prêt pour démarrer la réception de 
		données de 
		sync.      \\
		ALL\_EVENTS\_SYNC\_ASK &     Demande de toutes les données 
		typées 
		événement           \\
		EVENTS\_SYNC           &        Réception de données (en cours de 
		synchronisation)       \\
		META\_DATA\_ASK        &     Demande de métadonnées       \\
		META\_DATA             &      Réception de métadonnées       \\
		SYNC                   &      Réception de données (en cours de 
		synchronisation)         \\
		EVENT                  &     Réception d'une donnée typée 
		événement        \\
		END\_SYNC              & Fin de la synchronisation \\ \bottomrule
	\end{tabular}
\end{table}
\todo{parler des tableaux}

\begin{table}[h]
	\centering
	\caption{Statut du n\oe ud}
	\label{table:nodestatus}
	\begin{tabular}{ll}
		\toprule
		\textbf{Message}             & \textbf{Description} \\ \midrule
		ERROR               &      En erreur (désynchronisation)       \\
		READY               &       Prêt à recevoir des messages      \\
		META\_DATA\_ASK     &      En demande de métadonnées       \\
		META\_DATA\_RECEIVE &      En réception de métadonnées       \\
		CLOSE               &     Déconnecté (connexion fermée)        \\
		RECEIVE\_SYNC       &      En réception de données à synchroniser    \\
		CONNECTED           &      Connecté (connexion ouverte)        \\
		INIT                &     Initialisation   \\
		OK                  &    Connecté et synchronisé      \\
		SEND\_SYNC          &    En demande de synchronisation         \\
		END\_SYNC           &      Synchronisation terminée      \\ \bottomrule
	\end{tabular}
\end{table}

\pagebreak
\section{Experimentation 1 materials}
\subsection{Description des modèles utilisés}
\begin{table}[h]
	\caption{\label{table:models_xp1}Model descriptions for the experiments}
	\centering
	\begin{tabular}{lccc}
		\textbf{Experiment} & \textbf{objects} & \textbf{size}	& \textbf{users}\\ \hline
		Wind turbine     			& 6					& 1.0 MB        	& 2 \\ 
		Pick up					& 8					& 1.3 MB         	& 4	\\
		Castle from \textit{server}	& 35				& 1.3 MB        	& 4	\\
		Castle from \textit{peer}& 35 & 1.3 MB & 4 \\ \hline
	\end{tabular}
\end{table}

\subsection{Questionnaire pour l'enquête utilisateur}
\label{q:xp1}
Did you reach the goal or are you satisfied with the result?
\begin{itemize}
	\item Type of external communication channel used?
	\item Collaboration satisfaction?
	\item User interface quality: tools? Object manipulation? Global?
	\begin{itemize}
		\item Practice: difficulty or frustration?
		\item Collaborative rendering (latency, consistency, recovery)?
	\end{itemize}
\end{itemize}
\section{Expérimentation 2}
\label{app:quest}
\subsection{Description des modèles utilisés}
\begin{table}[ht]
	\centering
	\caption{Modèles utilisés durant l'expérimentation}
	\label{table:models_xp2}
	\begin{tabular}{cccc}\hline
		\textbf{Modèle}  & \textbf{Nombre de parties} &  \textbf{Nombre de triangles} 
		& \textbf{Taille totale}  \\ \hline
		Rotor      &   10  &    62k & 4Mo        \\
		Camera box        &   12  &    67k & 5Mo        \\
		Car      &   16  &      170k & 8Mo \\ 
		Living room      &   16  &      200k & 9Mo \\    \bottomrule
	\end{tabular}
\end{table}
\subsection{Questionnaire pour l'enquête utilisateur}
\label{q:xp2}
Seven-points scale questions from 1 (don't agree) to 7 (agree) :
\begin{itemize}
	\item Did you enjoy this?
	\item After trial, I was confident to do object manipulation in 3D virtual 
	environment ?
	\item App learning: I am satisfied with the ease of use of the application
	\item App learning: It was easy to learn the tool
	\item Solo: I completed the task quickly
	\item Solo: I completed the task efficiently
	\item Collaboration: I contributed to complete the task quickly
	\item Collaboration: I contributed to complete the task efficiently
	\item In general, I am satisfied with the collaboration experience
	\item In general, The collaboration quality was pleasant in terms of LATENCY
	\item In general, The collaboration quality was pleasant in terms of 
	CONSISTENCY
	\item In general, The collaboration quality was pleasant in terms of RECOVERY
\end{itemize}
Other questions:
\begin{itemize}
	\item General efficiency is improved with the number of users?
	\item General speed is improved with the number of users?
	\item I would qualify this application: Non interactive/Interactive/Near 
	real-time/Real-time
	\item Negative/positive/comments feedbacks
\end{itemize}

