%!TEX root = main.tex
\chapter{Introduction}
\chaptertable
\section{Contexte}
Le besoin de collaboration et de manipulation en temps-réel d'objets en \gls{3D} 
ainsi que leur contrôle de version a été très tôt une raison de la dissémination des 
outils pour la \gls{CAO}. Tandis que l'ingénierie et la visualisation scientifique ont 
produit des quantités de données massives dans des domaines tels que la 
conception architecturale, l'industrie des jeux ou encore l'impression \gls{3D}, un 
besoin croissant de maintenir, visualiser et manipuler de larges scènes \gls{3D} 
polygonales pouvant être éditées par de multiples utilisateurs de manière 
concurrente se fait sentir \info{better read the first 
	one!}\cite{Chandrasegaran2013,Wu2014}. Or, la quantité et la taille des données 
étant toujours croissante, il devient de plus en plus difficile de partager les 
modèles, particulièrement avec les utilisateurs n'ayant pas accès aux derniers 
logiciels et matériels graphiques. En conséquence,
le développement de plateformes légères déployées sur le web pour de répondre à ces 
besoins est en forte progression. 
Dans un \gls{EV3D}, la simulation virtuelle et simulation distribuée d'un environnement 3D, un grand nombre d'utilisateurs situés à différents endroits géographiques peuvent 
interagir les uns avec les autres en temps-réel. Un \gls{EVC3D} insiste sur l'aspect collaboratif des interactions qui se produisent entre les utilisateurs.
Cette distorsion de l'espace physique et temporel impose aux \glspl{EVC3D} des 
mécanismes de communication rapides, conservant un environnement cohérent
des données partagées durant la collaboration. 
L'utilisation d'un client comme medium pour 
accéder à l'\gls{EVC3D} est requise pour envoyer des demandes au serveur. Le 
client peut être manipulé par un utilisateur ou un agent logiciel (bot).
Un \gls{EVC3D} se compose également d'un protocole de communication qui 
permet aux différents clients d'échanger les mises à jours correspondants aux 
modifications effectuées dans l'espace 3D virtuel partagé. La distribution des 
données devient alors un enjeu de taille dans ce type d'application en terme de 
temps, de sécurité et de fiabilité. 


%!TEX root = main.tex
%\chapter{Contexte}




\section{La collaboration}
La collaboration \info{https://wiki.p2pfoundation.net/Collaboration} est souvent 
définie \info{ref}comme un processus récursif
\footnote{Le « principe de boucle 
	récursive » se retrouve dans le concept de la pensée complexe. Edgar Morin  
	explique qu'« un processus récursif est un processus où les produits et les 
	effets 
	sont en même temps causes et producteurs de ce qui les produit » 
	\cite[p. 100]{Morin1990}.} où deux (ou plus) personnes (ou organisations) 
travaillent ensemble à la croisée de buts communs 
en partageant leurs connaissances pour apprendre et bâtir un consensus. 
La collaboration permet l'émergence de conceptions partagées dans la réalisation 
de visions partagées dans des environnements et des systèmes complexes. 
Les imbrications de chaque domaine et la transdiciplinarité sont acceptés comme 
dans le concept de pensée complexe. D'ailleurs, dans sa définition de la 
complexité, Edgar Morin fait référence au sens étymologique latin \og 
complexus\fg{} qui signifie \og ce qui est tissé ensemble\fg{} \cite{Morin1990a}.
La plupart des collaborations requièrent un élément dirigeant qui peut prendre une 
forme sociale (personne) au sein d'un groupe décentralisé et égalitaire (tout le 
monde au même niveau). L'élément dirigeant va souvent aider également trouver 
des consensus. 
La disponibilité des ressources peut également devenir un élément dirigeant dans 
la collaboration.
Une équipe travaillant de manière collaborative peut concentrer plus de 
ressources, de reconnaissances et de récompenses lors d'une 
compétition comportant des ressources finies. 
La collaboration est aussi présente dans la recherche de buts opposés mettant en 
avant la notion de collaboration contradictoire (en opposition avec la collaboration 
constructive) ; la négociation et la compétition peuvent également faire partie du 
terrain 
de la collaboration.

Une application collaborative peut aussi intégrer les notions de coordination et de 
coopération:
\begin{itemize}
	\item La \textbf{coordination} se base sur le principe d'harmonisation des 
	tâches, des rôles et du calendrier dans des systèmes et environnements 
	simples.
	\item La \textbf{coopération} permet de résoudre des problèmes dans des 
	systèmes et environnements complexes dans le les participants aurait été 
	incapables (temps, espace, connaissance, matériel) d'accomplir le travail seul.
\end{itemize}

Dans les années 2000, deux classifications ont été retenue concernant la 
collaboration.
La première classification, décrite en 2007 par Gotta \cite{Gotta2007}, propose 
un modèle segmentant la collaboration de manière structurée en quatre catégories 
: de la plus dirigée à la plus volontaire, en passant par l'hybride.
\begin{itemize}
	\item \textit{Collaboration centrée processus.} Les conditions requises du 
	processus 
	nécessitent l'engagement de l'utilisateur, qui doit, de part son rôle ou sa 
	responsabilité, diriger ses efforts dans la collaboration avec les autres. Cette 
	stratégie se concentre sur les activités de manipulation collaborative 3D plutôt 
	que sur leur contexte organisationnel afin de favoriser la synergie autour de la 
	réussite d'un processus. Par exemple, pour la création et l'utilisation d'un 
	modèle 	3D dans un \gls{BIM}, il s'agit de favoriser la prise de décision sur un 
	projet et 	communiquer à propos.
	
	\item \textit{Collaboration centrée activité}
	Les activités partagées créent un sentiment de co-dé\-pendance qui motive la 
	collaboration entre les membres. La co-dépendance prend l'avantage sur le 
	propre intérêt de chacun comme motivation pour collaborer. Le groupe a besoin 
	de chacun pour que l'objectif soit considéré comme réalisé. L'intérêt personnel 
	ou l'allégeance à l'esprit d'équipe peut aussi promouvoir la collaboration. Par 
	exemple, la visualisation collaborative des activités des différents contributeurs 
	dans l'\gls{EVC3D} permet à chacun de rendre compte de ses réalisations. Ce 
	type de collaboration doit beaucoup à l'ergonomie de l'activité qui insiste sur la 
	différence entre le travail prescrit et le travail réel : la tâche et l'activité.
	
	\item \textit{Collaboration centrée communauté.}
	La participation de la communauté à la collaboration induit la contribution. En 
	effet, les interactions professionnelles ou sociales peuvent encourager ou 
	persuader les utilisateurs de partager leurs informations ou connaissances 
	(exemple : les logiciels \textit{open-source})
	
	\item \textit{Collaboration centrée réseau.}
	Les connexions réseau favorisent la coopération réciproque. Dans le but de 
	récupérer des avis ou du savoir faire externe, un utilisateur peut faire appel à 
	son réseau social pour supplémenter une autre interaction collaborative. 
	Souvent utilisé dans le cadre d'urgences écologiques ou sanitaires (exemple: 
	contributions OpenStreetMap lors d'ouragan ou de tsunami), la collaboration 
	centrée réseau est très présente dans des situations l'expertise est fortement 
	valorisée comme dans les \gls{BIM} ou la visualisation scientifique, les 
	contributeurs peuvent être intégrés en fonction des besoins des utilisateurs déjà 
	présents sur le projet.
\end{itemize}
Dans le cadre de cette thèse, en se référant à cette première classification, les 
aspects centrés sur les activités sont mis en avant. En effet, la collaboration 
portant sur la modélisation 3D attend un résultat porté sur l'activité de conception. 
Celle-ci nécessite l'implication de personnes avec différentes compétences / 
connaissances qui doivent s'entraider pour parvenir à la mise en commun des 
objets 3D et réaliser leur objectif. \info{add figure}

Une seconde classification proposée par Callahan et al. 
\cite{Callahan2008} s'intéresse au triplé collaboration par équipe, collaboration 
communautaire, collaboration en réseau. En contraste avec la précédente 
classification qui se concentre sur les équipes et une collaboration formelle et 
structurée, celle-ci offre plus d'ouverture :
\begin{itemize}
	\item \textit{Collaboration par équipe.}
	Dans une équipe tous les membres se connaissent. Il y a une interdépendance 
	claire des tâches à effectuer où la réciprocité est attendue, avec un échéancier 
	et des objectifs explicites. Pour réaliser son but, l'équipe doit réaliser les tâches 
	dans un temps imparti. La collaboration par équipe suggère que les membres 
	coopèrent sur un pied d'égalité (bien qu'il y ait souvent un chef) recevant une 
	reconnaissance égale.
	
	\item \textit{Collaboration communautaire.}
	L'objectif de ce type de collaboration est plus orienté sur la possibilité 
	d'apprendre que sur la tâche elle-même, même si les centres d'intérêt sont 
	partagés par la communauté. Les utilisateurs sont là pour partager et construire 
	la connaissance plus que compléter un projet. Les membres vont aller voir leur 
	communauté pour demander de l'aide sur un problème ou un avis et ramener la 
	solution à implémenter dans leur équipe. L'adhésion peut être limitée et 
	explicite, mais les périodes de temps sont souvent ouvertes. Les membres 
	sont considérés comme égaux bien que les plus expérimentés peuvent avoir 
	des statuts privilégiés. La réciprocité est un facteur important dans la 
	communauté pour que cela fonctionne.
	
	\item \textit{Collaboration en réseau.}
	La collaboration en réseau est une sur-couche de la collaboration 
	traditionnellement centrée sur la relation d'une équipe ou d'une communauté. 
	Elle s'appuie sur une action individuelle et un intérêt personnel qui resurgissent 
	ensuite sur le réseau sous la forme de personnes qui contribuent ou cherchent 
	quelque chose à partir du réseau. L'adhésion et les périodes sont ouvertes et 
	non limitées. Il n'y a pas de rôle explicite. Les membres ne se connaissent pas 
	forcément. Le pouvoir est distribué. Cette forme de collaboration est dirigée par 
	l'avènement des réseaux sociaux, des accès à internet omniprésents et la 
	capacité de se connecter avec divers individus malgré la distance.
\end{itemize}
Cette thèse, en se référant à cette seconde classification, s'intéresse plutôt 
sur la collaboration par équipe. La conception d'un objet 3D et ses différentes 
phases de modélisation constitue une problématique nécessitant l'apport de 
plusieurs intervenants avec leurs capacités propres et travaillant de concert à la 
réalisation d'un objectif commun dans un temps imparti (exemple : revue de 
projet). Là où la coopération et l'effort conjoint pour réaliser un objectif sont 
nécessaires, le facteur temps reste un élément important à prendre en compte 
pour évaluer la productivité d'une session collaborative.

Le travail dans un \gls{EVC3D} facilite la compréhension 
de certaines problématiques liées à l'espace 3D ; c'est également un point de 
rencontre et d'échanges entre contributeurs sur le court terme et le long terme. 
Le croisement de ces deux dimensions, spatiale et temporelle, implique une 
multiplication des points de vue et donc des données à traiter sur le problème lors 
de la collaboration.

\subsection{Les systèmes collaboratifs}
\subsection{Les systèmes d'édition collaboratifs}
\subsubsection{Modèle d'édition collaborative}

\section{La collaboration 3D en accord avec l'évolution du web}

\subsection{Introduction}
\subsection{Le web et le P2P : WebRTC}
\subsection{Le web et la 3D :  WebGL}

\section{Les architectures évènementielles pour la collaboration}
\subsection{Sensibilisation lors de la collaboration}
\subsection{Intégration des contraintes métiers}



Les bureaux d'études en ingénierie et en architecture travaillent sur des projets 
(visualisation \gls{CAO}, \gls{BIM}, gestion et arrangement d'espaces 
architecturaux) qui 
nécessitent la collaboration de professionnels venant de milieux différents avec 
des compétences et connaissances variées. Les modifications dans leurs 
modèles en \gls{3D} doivent être revues par des gestionnaires de projet, des 
clients et les intervenants impliqués qui peuvent à leur tour suggérer des 
modifications sur la conception tout cela en accord avec des contraintes métiers 
transparentes. Ces règles doivent donc apparaître dès la conception pour être 
intégrées tôt dans la modélisation 3D.

%Toutes ces entités ont besoin d'être capables de 
%charger les ressources pour pouvoir les inspecter et les analyser. Ces 
%contraintes 
%se retrouvent dans le domaine du \gls{BIM}, l'architecture, l'héritage culturel, ou 
%plus généralement dans des milieux transdisciplinaires concentrés sur la 
%\gls{3D}. 
%
%L'évolution de ces ressources passent par une visualisation et une manipulation 
%collaborative efficace en terme de chargement de ressources et de transmission 
%des mises à jour.
\section{Problématique}


Cette thèse se situe à l'interface de trois champs de recherches (Figure 
\ref{fig:problematique}). Le premier concerne les environnements de modélisation 
3D. Souvent commerciaux (Clara.io, OnShape, Verold Studio), les modeleurs 
utilisent des technologies supportées par les navigateurs web qui leur permettent 
d'être disponibles sur la plupart des plateformes. 
Cependant ces dernières reposent sur une gestion centralisée des données qui rend 
les utilisateurs très dépendants de la disponibilité de ces plateformes et d'une 
connexion internet pour la distribution des informations. 
Mises à part quelques exceptions, les fonctionnalités collaboratives 
sont souvent présentées comme mineures. Même si le \og partage\fg{} de la 
visualisation à la manière des \og réseaux sociaux\fg{} est assez courante, l'édition 
collaborative est souvent complexe à implémenter car les besoins sont nombreux. 
Parmi eux, on trouve tout d'abord le besoin d'avoir un système distribué 
conservant la cohérence des modifications de chacun des utilisateurs. Or, 
le nombre d'utilisateurs ne doit pas affecter l'expérience utilisateur ; le système 
doit supporter le passage à l'échelle. 
Le nombre d'utilisateurs élevé implique la mise en place d'un système de 
distribution de données adapté. Ce système est en plus contraint par la dynamicité
des arrivées et départs des collaborateurs (\textit{churn}) au cours d'une session. 
La dynamicité, qui ne permet pas d'avoir de pouvoir compter sur un nombre fixe de 
ressources, doit s'accorder avec les besoins variables en termes de ressources. 
Chaque client est porteur de ressources rarement complètement exploitées lors 
d'épisodes collaboratifs.

La visualisation et la manipulation collaborative d'objets \gls{3D} sont sujettes à 
plusieurs problématiques telles que la cohérence des ressources manipulées au 
cours du temps. Dans ce contexte, il existe un fort besoin de gérer l'évolution des 
versions permettant la revue des modèles \gls{3D}. 
En s'appuyant sur les ressources à dispositions, tels que les clients (navigateurs 
web) sur lesquels les utilisateurs manipulent les modèles \gls{3D}, le passage à 
l'échelle est plus flexible car les ressources augmentent avec le nombre de clients 
et procurant également plus d'autonomie aux utilisateurs utilisant leurs ressources 
propres.
L'architecture de communication et les contraintes liées aux architectures 
distribuées et décentralisées correspond au second axe de recherche. 
Enfin, le dernier champs 
s'adresse à la partie que nous appellerons \og métier\fg{} qui se rapporte au 
domaine de la manipulation d'objets 3D qui inclut la 
gestion du cycle de vie des données 3D de l'interaction utilisateur à son stockage 
en passant par sa distribution. Ces aspects se rapportent principalement à 
l'historique, la traçabilité de l'information et l'expertise embarquée dans le système.

\begin{figure}[hbt]
	\centering
	\includegraphics[width=\columnwidth]{problematique.eps}
	\caption{Place de la contributions dans les différents champs de recherche}
	\label{fig:problematique}
\end{figure}

L'objectif de cette thèse est triple : 
\begin{enumerate*}[label=(\roman*)]
	\item extraire les informations liées aux règles métier pour l'affichage et la 
	manipulation d'objets \gls{3D} en collaboration nécessaires à la traçabilité de 
	l'information,
	\item repérer les principales problématiques de gestion des données sur le 
	réseau pour avoir une transmission efficace et transparente pour l'utilisateur,
	\item proposer un \gls{framework} pour un \gls{EVC3D} web intégrant ces 
	contraintes réseau, métier, et \gls{3D} dans un navigateur.
\end{enumerate*}
Dans le but d'achever de tels objectifs, voici les cinq Questions de Recherche 
posées :
\begin{description}
	\item[QR 1] Quelle architecture réseau est la plus adaptée pour une gestion 
	efficace, robuste et temps-réel des données \gls{3D} dans un environnement 
	web ?
	
	\item[QR 2] Quelle architecture logicielle confère une traçabilité des données 
	conforme aux règles métier liées à la manipulation d'objet \gls{3D} ? 
	
	\item[QR 3] Quels sont les mécanismes assurant à l'utilisateur d'être à la fois 
	autonome tout en ayant la possibilité de collaborer ?
	
	
	\item[QR 4] Comment faciliter l'implémentation d'un tel système en garantissant 
	le respect des règles métier liés à la manipulation d'objet \gls{3D}?
	
	\item[QR 5] Quelles sont les métriques (réseau, collaboration) permettant 
	d'évaluer un tel système de manière quantitative? qualitative? %Comment les 
	%utilisateurs recoivent la chose....
	
\end{description}

\section{Contributions}

La principale contribution de cette thèse est la définition et l'implémentation d'un 
ensemble de pratiques dans la gestion des données pour la manipulation d'objets 
3D de manière collaborative sur web. Cela est très utile pour la gestion de 
l'historique d'une scène \gls{3D} en utilisant des contenus \gls{3D} de différentes 
natures. Pour ce faire, j'ai choisi le paradigme événementiel et je l'ai 
intégré à mon modèle de collaboration grâce à l'utilisation de solutions 
préexistantes pour le partage et la diffusion de contenus \gls{3D} dans le cadre 
d'applications collaboratives sur le web.

En constatant le manque de processus et d'outils qui pourraient réellement supporter la modélisation collaborative \gls{3D} sur le web intégrant l'identification de problèmes spécifiques et leur formulation liés a également été importante. 
La recherche, cependant, introduit de nouveaux concepts dans le domaine de la gestion de 
données \gls{3D}. Les contributions de cette thèse peuvent être résumées comme 
suit.


\subsection{Contributions théoriques}

\begin{itemize}
	\item \improve{réécrire + introduire ref chapitre}Proposition d'un système de 
	gestion et de visualisation de contenu \gls{3D} avec un historique non linéaire 
	orienté évènements
	\item Définition d'une architecture hybride (client serveur et \gls{P2P}) adapté à 
	la distribution de contenus \gls{3D} dans le cadre d'une collaboration sur web en 
	temps-réelle.
	\item Introduction des concepts d'évènements métiers liées à la \gls{3D} 
	comme moyen d'interagir dans une scène  \gls{3D} multi-échelle
\end{itemize}
\subsection{Contributions pratiques}
\begin{itemize}
	\item \improve{réécrire + introduire ref chapitre}Définition d'une API ouverte et 
	de l'application cliente utilisation les principes et les technologies du web.
	\item Implémentation d'un prototype 3DEvent Client et de son évaluation 
	utilisateur 
	\item Implémentation d'un prototype 3DEvent Architecture et de son évaluation
	
\end{itemize}

%\bibentry{Desprat2017}

\section{Organisation du manuscrit}

%\section*{Organisation du manuscrit}
La présence de la \gls{3D} et des environnements virtuels collaboratifs \gls{3D} est 
de plus en plus présents dans l'industrie de la \gls{CAO} nécessitant de nouveaux 
modèles, outils et méthodes facilitant la mobilité et l'autonomie des utilisateurs. Il 
est aussi important de noter que le contenu multimédia \gls{3D} est aussi de plus 
en plus présent dans notre quotidien. Pour tout cela il est important de développer 
de nouveaux outils pour créer manipuler et partager du contenu \gls{3D}.
Le chapitre \info{ref chapitre} présente une architecture de communication hybride 
permettant de faciliter la transmission des objets \gls{3D} en temps réel de 
manière transparente pour l'utilisateur. La création d'\gls{EVC3D} sur le web à 
besoin de nouvelles architectures tirant partie des nouveaux standards du 
\gls{W3C}. Puis, le chapitre \info{ref cqrs} présente deux \info{2 contrib cqrs à 
	identifier} contributions reposant sur l'architecture cliente de 3DEvent pour 
manipuler des objets \gls{3D} avec une grande traçabilité et de manière autonome. 
%La première contribution propose 
Finalement, le chapitre\info{ref chap final} adresse le problème et la spécificité de 
la transmission du contenu \gls{3D} lors de session collaborative en temps-réel. 
En considérant et développant un modèle orienté évènements, 3DEvent dérive 
des 
stratégies pour délivrer et obtenir une scène 3D en continu (\textit{streaming} en 
anglais) en s'appuyant sur les différentes configurations matérielles, utilisateurs, 
réseaux.