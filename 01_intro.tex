%!TEX root = main.tex
\chapter{Introduction}
\chaptertable
Le besoin de collaboration et de manipulation en temps-réel d'objets en \gls{3D} 
ainsi que leur contrôle de version a été très tôt une raison de la dissémination des 
outils pour la \gls{CAO}. Tandis que l'ingénierie et la visualisation scientifique ont 
produit des quantités de données massives dans des domaines tels que la 
conception architecturale, l'industrie des jeux ou encore l'impression \gls{3D}, un 
besoin croissant de maintenir, visualiser et manipuler de larges scènes \gls{3D} 
polygonales pouvant être éditées par de multiples utilisateurs de manière 
concurrente se fait sentir \info{better read the first 
	one!}\cite{Chandrasegaran2013,Wu2014}. Or, la quantité et la taille des données 
étant toujours croissante, il devient de plus en plus difficile de partager les 
modèles, particulièrement avec les utilisateurs n'ayant pas accès aux derniers 
logiciels et matériels graphiques. C'est pourquoi nous observons un 
développement en forte progression de plateformes légères déployées sur le web 
permettant de répondre à ces besoins. 
Dans un \gls{EVC3D}, la simulation distribuée d'un environnement virtuel 3D, un 
grand nombre d'utilisateurs situés à différents endroits géographiques peuvent 
interagir les uns avec les autres en temps-réel. 
Cette distorsion de l'espace physique et temporel impose aux \gls{EVC3D} des 
mécanismes de communication rapide, conservant un environnement cohérent
des données partagées durant la collaboration. 
L'utilisation d'un client comme medium pour 
accéder à l'\gls{EVC3D} est requise pour envoyer des demandes au serveur. Le 
client peut être manipulé par un utilisateur ou un agent logiciel (bot).
Un \gls{EVC3D} se compose également un protocole de communication qui 
permet aux différents clients d'échanger les mises à jours correspondants aux 
modifications effectuées dans l'espace 3D virtuel partagé. La distribution des 
données devient alors un enjeu de taille dans ce type d'application en terme de 
temps, de sécurité et de fiabilité. 

L'exigence vis-à-vis des temps de réponse est de plus en plus 
grande pour deux raisons principales : les limitations humaines (mémoire et 
attention limitées 
à court terme) et les aspirations de l'être humain (besoin d'être en contrôle sur les 
machines). S'accroissant au gré de la technologie et aux attentes 
utilisateurs (rétro-compatibilité, temps de 
chargement), elle varie selon le domaine mais reste très assez prégnante sur le 
web en général. 
Par exemple, dans un domaine annexe comme le e-commerce, une étude réalisée 
en 2006 explique qu’une grande partie des internautes abandonnaient leurs achats 
en ligne si les pages mettaient 4 secondes ou plus à charger. De nos jours, ce 
délai a été réduit au quart de seconde pour les grandes entreprises du web. Jakob 
Nielsen \cite{Nielsen1993} indique trois temps de réponse limites concernant les 
temps de réponses :
\begin{itemize}
	\item \textit{0,1 seconde} donne une sensation de réponse instantanée, comme 
	si le 
	résultat avait été produit par l'utilisateur et non l'ordinateur. Ce niveau de temps 
	de réponse soutient la sensation de manipulation directe.\footnote{En IHM, la 
		manipulation directe correspond à un mode d'interaction au cours duquel les 
		utilisateurs font des actions sur les objets d'intérêt affichés dans l'interface 
		utilisateur en utilisant des actions physiques, incrémentales et réversibles 
		dont 
		les effets sont immédiatement visibles sur l'écran.} 
	\item \textit{1 seconde} garde le flux de pensées de l'utilisateur sans 
	interuption. 
	L'utilisateur peut ressentir un délai et par conséquent savoir que c'est la 
	machine qui génère le résultat ; l'utilisateur a quand même une impression de 
	contrôle sur l'expérience générale et peut de déplacer librement dans l'interface 
	sans attendre la machine. Ce degré de réactivité est impératif pour une bonne 
	navigation.
	\item \textit{10 secondes} conservent l'attention de l'utilisateur. Entre 1 et 10 
	secondes, l'utilisateur se sent dépendant de la machine, mais peut faire avec. 
	Au delà, l'utilisateur va commencer à penser à d'autres choses, rendant difficile 
	le retour à la tâche une fois que la machine répond.
\end{itemize} 
Dans le domaine de l'édition et la manipulation collaborative 
d'objets 3D, les contraintes abordées se situent à divers degrés : au chargement 
et lors des mises à jour. Le chargement concerne la phase de téléchargement de 
l'application web et des données relatives mais également d'affichage des 
modèles 3D sont spécifiques dans ce domaine car ils ont 
tendance à être lourds (structure de données 3D et métadonnées) à charger 
(contrairement au texte par exemple). Cette phase peut s'accorder des délais 
relativement long (1-10s) compte tenu du fait que l'utilisateur connaît en partie ces 
contraintes liées à la taille des objets 3D. Concernant les mises à jour, l'édition 
collaborative requiert un temps de réponse raisonnablement court est contraint par 
l'exactitude des calculs et le temps dans lequel le résultat est produit. Pour les 
mises à jour internes -- produites par l'utilisateur, actions sur l'interface -- on 
s'accordera sur un délai inférieur à 0,1 seconde dans cette thèse. 
Pour les mises à jour externes -- produites par les collaborateurs -- les latences 
réseaux rentrent en comptes (serveur, pairs\ldots) on s'accordera sur un délai 
entre 1 et 10 secondes dans cette thèse selon les degrés d'asynchronicité 
possible. 

Les bureaux d'études en ingénierie et en architecture travaillent sur des projets 
(visualisation \gls{CAO}, \gls{BIM}, gestion et arrangement d'espaces 
architecturaux) qui 
nécessitent la collaboration de professionnels venant de milieux différents avec 
des compétences et connaissances variées. Les modifications dans leurs 
modèles en \gls{3D} doivent être revues par des gestionnaires de projet, des 
clients et les intervenants impliqués qui peuvent à leur tour suggérer des 
modifications sur la conception. Toutes ces entités ont besoin d'être capables de 
charger les ressources pour pouvoir les inspecter et les analyser. Ces contraintes 
se retrouvent dans le domaine du \gls{BIM}, l'architecture, l'héritage culturel, ou 
plus généralement dans des milieux transdisciplinaires concentrés sur la \gls{3D}. 
L'évolution de ces ressources passent par une visualisation et une manipulation 
collaborative efficace en terme de chargement de ressources et de transmission 
des mises à jour.