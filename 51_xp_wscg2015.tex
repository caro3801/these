%!TEX root = main.tex
%\section{Assemblage d'objets 3D sur le web avec architecture 
%événementielle}
%\label{sec:us}

\subsection{Présentation de l'expérimentation 1}

L'expérimentation propose de répliquer une collaboration réaliste entre 
différents participants travaillant à distance pour faire des tâches d'assemblage de 
modèle \gls{3D}. L'expérimentation reprend le cas d'étude de l'assemblage d'objets 
\gls{3D} 
(Section \ref{sec:use_case}). 

L'expérimentation est basée sur un prototype d'application développé selon 
l'implantation présentée dans la Section \todo{ref section proto}.
Le prototype est développé sur une plateforme web et consiste en un application 
de modélisation \gls{3D} collaborative multi-utilisateurs pour démontrer la 
faisabilité de 
l'architecture présentée dans \todo{ref section archi state}. 


\paragraph{Tâches à effectuer.}
L'expérimentation consiste en quatre épreuves qui partagent la même procédure.
Les modèles utilisés dans chaque épreuve sont décrits dans le tableau 
\ref{table:models_xp1} : \textit{Wind Turbine}, \textit{Pick up} et \textit{Castle}. 

Les épreuves sur les modèles \textit{Wind Turbine} et \textit{Pick Up} partagent le 
même objectif : 
assembler un modèle, dont les utilisateurs possède l'image, composé de 
différentes pièces qui sont téléversées par les utilisateurs. L'image correspond à 
l'assemblage final à obtenir et indique les différentes parties qui le 
compose (type et quantité). Ces deux épreuves s'arrêtent lorsque les participants 
le notifient oralement ou durent dix minutes maximum chacune.

Le modèle \textit{Castle} est utilisé dans deux épreuves : 
\textit{Castle from server} et \textit{Castle from peer}. L'objectif de ces épreuves
diffère un peu des épreuves précédentes car l'épreuve s'effectue sur un  modèle 
de château en kit (composer d'un dizaine d'objets différents : tour, murs, 
escalier\dots) : les 
participants ont dix minutes pour construire un 
château de manière collaborative en faisant appel à leur créativité. 
Dans l'épreuve \textit{Castle from server}, les objets sont récupérés 
automatiquement à partir du serveur ; 
dans l'épreuve \textit{Castle from peer}, les pairs peuvent ajouter de nouveaux 
objets du kit qu'ils possèdent sur leur machine.

En utilisant les outils de l'éditeur, les informations liées aux autres utilisateurs 
(représentation dans l'environnement \gls{3D} de leur position ou de leur sélection), 
les 
utilisateurs doivent manipuler les pièces (sélection, rotation, translation, 
homothétie) de manière collaborative afin d'atteindre l'objectif. 
\paragraph{Population}
L'expérience a été conduite sur trois groupes de deux ou quatre participants 
chacun. Un essai pour \textit{Wind Turbine} a été tenté à six participants 
simultanés.
Durant l'expérimentation, les utilisateurs était sur le même réseau que le serveur 
(\gls{LAN}). 
Les participants étaient des étudiants de Master ou de Doctorat en Informatique 
plutôt familiers avec les environnements \gls{3D}. Les participants étaient 
autorisés à 
communiquer entre eux (oralement, par chat, autre \dots).
\paragraph{Procédure}
L'expérimentation consiste en quatre épreuves qui partagent la même procédure:
\begin{enumerate}
	\item Phase d'essai 
	\begin{enumerate}
		\item explication du contexte et de l'expérimentation ;
		\item prise en main du système, familiarisation avec 
		l'interface ;
	\end{enumerate}
	\item Phase de collaboration : réalisation de l'épreuve (se répète autant de fois 
	que d'épreuves)
	\begin{enumerate}
		\item Présentation de l'épreuve et de son objectif
		\item Initialisation : nettoyage de la scène et chargement des modèles
		\item Réalisation de l'objectif
	\end{enumerate}
	\item Phase de retour d'expérience : remplissage questionnaire et discussion 
	informelle sur les épreuves.
\end{enumerate}

\paragraph{Initialisation}

L'épreuve est décrite et l'objectif de l'épreuve est présenté a tous les participants.
Pour chaque phase de collaboration, l'application nettoie les données issues de 
l'épreuve précédente. 
À l'exception des pièces du modèle \textit{Castle} qui sont chargées
en totalité sur le serveur dans \textit{Castle from server} et les pairs \textit{Castle 
from peer}), les pièces du modèle de l'épreuve sont distribuées de manière 
aléatoire entre les différents participants.

\paragraph{Données collectées}
Au cours des épreuves (principalement lorsque celles ci s'arrêtent)  les 
participants ont exprimé certains retours qualitatifs dont il a été pris note. Le type 
d'interaction entre les participants est également observé.

\paragraph{Questionnaire}
Au début de l'expérimentation, les participants reçoivent et prennent connaissance 
du questionnaire (voir Annexe \ref{q:xp1}). Ce questionnaire est rempli à plusieurs 
moment au cours de l'expérimentation. Au début, le participant décrit sa situation, 
sa familiarité avec les environnements \gls{3D}, le type de machine et de 
navigateur 
qu'il utilise. À la fin de chaque épreuve il répond aux questions concernant 
l'épreuve qu'il vient de passer. À la fin de l'expérimentation, il répond aux 
questions d'ordre plus général sur son expérience et les améliorations 
envisageables.
\subsection{Résultats et discussion}
Les participants ont globalement été satisfaits des résultats issus de la 
collaboration ainsi que du rendu visuel des assemblages effectués durant les 
épreuves. Cela est également dû au fait qu'ils ont systématiquement atteint les 
objectifs (dans le temps imparti) : réalisation de l'assemblage du modèle \gls{3D} 
présenté sans frustration. Ils se sont également \og amusés\fg{} sur les épreuves
avec le \textit{Castle} car ils étaient libres dans la création et souhaiter continuer 
l'épreuve au delà du temps imparti. L'utilisation de canaux de communication 
externes à l'application a été reportée plusieurs fois, principalement sous formes 
d'échanges oraux. Ces échanges concernait l'état de leur application et ce qu'ils 
étaient en train de faire ou ce qu'ils souhaitaient faire. Cette interaction provient, 
d'après les participants, du manque de retours visuels différenciés sur les 
manipulations effectués par les collaborateurs (exemple: pas de couleur différente 
pour sélection d'objet par un autre utilisateur). Cela peut plus généralement 
s'expliquer comme un manque sensibilisation à l'environnement collaboratif. 
L'\gls{IU} a été bien appréciée, parfois jugée \og trop simple\fg{} par certains 
utilisateurs. Ce choix avait été fait pour faciliter l'usage, l'apprentissage et 
l'adoption de l'\gls{IU} et s'est révélé anecdotique lors de l'expérimentation car le 
panel de personne de participants était familier de ce genre 
d'environnement (échantillon faible).
Les fonctionnalités liées à la manipulation d'objets reçoivent une bonne évaluation 
à l'exception de l'importation de modèles. Cela est dû, dans ce prototype, au 
fait que l'application ne peut digérer et transmettre des modèles trop lourds. Un 
participant a vu sa fenêtre \og geler\fg{} (navigateur Chrome) lors de l'épreuve 
\textit{Castle from peer}. Cela l'a amené à quitter la session et à revenir sur la 
scène pour pouvoir continuer de participer à la session collaborative. Sans 
perturber la session en cours pour ses collaborateur, le participant a pu revenir sur 
l'application, reprenant la collaboration avec les autres participants (rétablissement 
des connexions après un crash).En cela le participant a apprécié la robustesse de 
l'application sans être perturbé très longtemps par l'interruption.
Concernant la fluidité des manipulations et de la visualisation au sein de 
l'application durant la collaboration, les participants n'ont pas ressenti de latence 
perturbantes en général. Ils ont qualifié l'application de \og temps réel\fg{} plutôt 
qu'\og interactive\fg{}. La variation du nombre d'utilisateur sur les différentes 
épreuves n'a pas altéré la qualité du rendu et de réseau pour le participant.

\subsection{Conclusion de l'expérimentation 1}


Cette expérimentation présente différentes épreuves sur le cas d'études lié à cette 
thèse. Elle repose sur un modèle utilisant une architecture de communication 
hybride combinant client-serveur et \gls{P2P}. Le client est responsable de 
proposer un environnement \gls{3D} intégrer à l'interface permettant la 
manipulation et 
la visualisation des objets \gls{3D} manipulés de manière collaborative. Afin de 
pouvoir 
collaborer, il héberge également les connexions nécessaires à communiquer ses 
mises à jour vers les autres pairs (autant de connexion WebRTC DataChannel que 
de pairs) et vers la base de données via le serveur (une seule connexion 
WebSocket). Les modifications sont transmises sous forme de différentiel d'état et 
sont stocker sur les pairs localement et sur la base de données de manière 
distante. Le réseau \gls{P2P} est composé uniquement de client producteurs de 
données reliés de manière complète. 
Les évaluations liées à l'expérimentation sont plutôt encourageantes même si 
certains points doivent être améliorés. Concernant la fonctionnalité d'import de 
modèle, la transmission par les pairs du modèle a causé des latences sur les 
pairs destinataires (gèle de la fenêtre). Afin de réduire la latence sur des scènes 
aussi voire plus grandes, deux alternatives sont à étudier : l'utilisation d'un 
rendu progressif et compressé ainsi qu'une meilleure répartition de la charge entre 
les pairs pour la distribution des données. Dans ce dernier cas, l'utilisation d'un 
maillage partiel et d'une distribution des modèles incluant les pairs permettrait de 
responsabiliser d'avantage les pairs en désengorgeant un peu le serveur et surtout 
le réseau \gls{P2P}.
Concernant l'amélioration des fonctionnalités liés à l'interface, l'ajout de retours 
visuels liés aux interactions collaborations ainsi la visualisation de l'historique sont 
indispensables. Une évaluation quantitative approfondie pourrait permettre de 
mieux examiner les apports de l'architecture de communication hybride par rapport 
aux architectures classiques dans le contexte de la modélisation \gls{3D} 
collaborative. 
Cela pourrait se concrétiser par l'examen de la collaboration en récupérant le 
journal des actions, l'impact sur la visualisation (Frame Per Second), et 
l'observation des échanges WebRTC en entrées / sorties.

%An improvement of interface features and visual feedback of collaborative 
%manipulations was asked by the users. This evaluation will be supple- mented in 
%future works with a quantitative evaluation to compare our hybrid architecture to 
%others, particu- larly using server logs, FPS in client and WebRTC tools (Chrome 
%: 
%chrome://webrt-internal; Firefox : about:webrtc) which provides statistics and 
%graphs on the data exchanged between peers’ browsers.