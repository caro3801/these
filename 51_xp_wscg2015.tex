%!TEX root = main.tex
%\section{Assemblage d'objets 3D sur le web avec architecture 
%événementielle}
%\label{sec:us}

\subsection{Présentation de l'expérimentation}

 L'expérimentation propose de répliquer une collaboration réaliste entre 
différents participants travaillant à distance pour faire des tâches d'assemblage de 
modèle 3D.

L'expérimentation est basée sur un prototype d'application développé selon 
l'implémentation présentée dans la Section \todo{ref section proto}.
Le prototype est développé sur une plateforme web et consiste en un application 
de modélisation 3D collaborative multi-utilisateurs. 
D'un côté, l'interface permettant la visualisation et l'édition 

We developed a prototype of the the web-based multi- user collaborative modeling 
to demonstrate the feasi- bility of our model architecture.. In one hand, the vi- 
sualization was based on the WebGL technology us- ing ThreeJS to visualize 3D 
models online and offline without plugins. On the other hand, the real-time in- 
teractive collaboration relied on the hybrid architecture model exposed in the 
previous section. The Node.Js server platform allowed us to run a WebSocket 
server that handled the signaling mechanism for the WebRTC user connections 
creating the P2P mesh. The resources of the client were used in terms of 
graphics, storage, WebRTC capabilities in order to share the scene infor- mation 
between the users. We propose three experi- ments (Table 3) to evaluate our 
system in the following criteria: user-friendly interface, robustness and network 
latency.




\subsection{Résultats}
\subsection{Discussion et Conclusion}