%!TEX root = main.tex
%\section{Assemblage d'objets 3D sur le web avec architecture 
%événementielle}
%\label{sec:us}

\subsection{Présentation de l'expérimentation}

L'expérimentation propose de répliquer une collaboration réaliste entre 
différents participants travaillant à distance pour faire des tâches d'assemblage de 
modèle 3D. L'expérimentation reprend le cas d'étude de l'assemblage d'objets 3D 
(Section \ref{xp:use_case}). 

L'expérimentation est basée sur un prototype d'application développé selon 
l'implantation présentée dans la Section \todo{ref section proto}.
Le prototype est développé sur une plateforme web et consiste en un application 
de modélisation 3D collaborative multi-utilisateurs pour démontrer la faisabilité de 
l'architecture présentée dans \todo{ref section archi state}. 


\paragraph{Tâches à effectuer.}
L'expérimentation consiste en quatre épreuves qui partagent la même procédure.
Les modèles utilisés dans chaque épreuve sont décrits dans le tableau 
\ref{table:models_xp1} : \textit{Wind Turbine}, \textit{Pick up} et \textit{Castle}. 

Les épreuves sur les modèles \textit{Wind Turbine} et \textit{Pick Up} partagent le 
même objectif : 
assembler un modèle, dont les utilisateurs possède l'image, composé de 
différentes pièces qui sont téléversées par les utilisateurs. L'image correspond à 
l'assemblage final à obtenir et indique les différentes parties qui le 
compose (type et quantité). Ces deux épreuves s'arrêtent lorsque les participants 
le notifient oralement ou durent dix minutes maximum chacune.

Le modèle \textit{Castle} est utilisé dans deux épreuves : 
\textit{Castle from server} et \textit{Castle from peer}. L'objectif de ces épreuves
diffère un peu des épreuves précédentes car l'épreuve s'effectue sur un  modèle 
de château en kit (composer d'un dizaine d'objets différents : tour, murs, 
escalier\dots) : les 
participants ont dix minutes pour construire un 
château de manière collaborative en faisant appel à leur créativité. 
Dans l'épreuve \textit{Castle from server}, les objets sont récupérés 
automatiquement à partir du serveur ; 
dans l'épreuve \textit{Castle from peer}, les pairs peuvent ajouter de nouveaux 
objets du kit qu'ils possèdent sur leur machine.

En utilisant les outils de l'éditeur, les informations liées aux autres utilisateurs 
(représentation dans l'environnement 3D de leur position ou de leur sélection), les 
utilisateurs doivent manipuler les pièces (sélection, rotation, translation, 
homothétie) de manière collaborative afin d'atteindre l'objectif. 
\paragraph{Population}
L'expérience a été conduite sur trois groupes de trois ou quatre participants 
chacun. Un essai pour \textit{Wind Turbine} a été tenté à six participants 
simultanés.
Durant l'expérimentation, les utilisateurs était sur le même réseau que le serveur 
(\gls{LAN}). 
Les participants étaient des étudiants de Master ou de Doctorat en Informatique 
plutôt familiers avec les environnements 3D. Les participants étaient autorisés à 
communiquer entre eux (oralement, par chat, autre \dots).
\paragraph{Procédure}
L'expérimentation consiste en quatre épreuves qui partagent la même procédure:
\begin{enumerate}
	\item Phase d'essai 
	\begin{enumerate}
		\item explication du contexte et de l'expérimentation ;
		\item prise en main du système, familiarisation avec 
		l'interface ;
	\end{enumerate}
	\item Phase de collaboration : réalisation de l'épreuve (se répète autant de fois 
	que d'épreuves)
	\begin{enumerate}
		\item Présentation de l'épreuve et de son objectif
		\item Initialisation : nettoyage de la scène et chargement des modèles
		\item Réalisation de l'objectif
	\end{enumerate}
	\item Phase de retour d'expérience : remplissage questionnaire et discussion 
	informelle sur les épreuves.
\end{enumerate}

\paragraph{Initialisation}

L'épreuve est décrite et l'objectif de l'épreuve est présenté a tous les participants.
Pour chaque phase de collaboration, l'application nettoie les données issues de 
l'épreuve précédente. 
À l'exception des pièces du modèle \textit{Castle} qui sont chargées
en totalité sur le serveur dans \textit{Castle from server} et les pairs \textit{Castle 
from peer}), les pièces du modèle de l'épreuve sont distribuées de manière 
aléatoire entre les différents participants.

\paragraph{Données collectées}
Au cours des épreuves (principalement lorsque celles ci s'arrêtent)  les 
participants ont exprimé certains retours qualitatifs dont il a été pris note. Le type 
d'interaction entre les participants est également observé.

\paragraph{Questionnaire}
Au début de l'expérimentation, les participants reçoivent et prennent connaissance 
du questionnaire (voir Annexe \ref{q:xp1}). Ce questionnaire est rempli à plusieurs 
moment au cours de l'expérimentation. Au début, le participant décrit sa situation, 
sa familiarité avec les environnements 3D, le type de machine et de navigateur 
qu'il utilise. À la fin de chaque épreuve il répond aux questions concernant 
l'épreuve qu'il vient de passer. À la fin de l'expérimentation, il répond aux 
questions d'ordre plus général sur son expérience et les améliorations 
envisageables.
\subsection{Résultats et discussion}
Les participants ont globalement été satisfaits des résultats issus de la 
collaboration ainsi que du rendu visuel des assemblages effectués durant les 
épreuves. Cela est également dû au fait qu'ils ont systématiquement atteint les 
objectifs (dans le temps imparti) : réalisation de l'assemblage du modèle 3D 
présenté sans frustration. Ils se sont également \og amusés\fg{} sur les épreuves
avec le \textit{Castle} car ils étaient libres dans la création et souhaiter continuer 
l'épreuve au delà du temps imparti. L'utilisation de canaux de communication 
externes à l'application a été reportée plusieurs fois, principalement sous formes 
d'échanges oraux. Ces échanges concernait l'état de leur application et ce qu'ils 
étaient en train de faire ou ce qu'ils souhaitaient faire. Cette interaction provient, 
d'après les participants, du manque de retours visuels différenciés sur les 
manipulations effectués par les collaborateurs (exemple: pas de couleur différente 
pour sélection d'objet par un autre utilisateur). Cela peut plus généralement 
s'expliquer comme un manque sensibilisation à l'environnement collaboratif. 
L'\gls{IU} a été bien appréciée, parfois jugée \og trop simple\fg{} par certains 
utilisateurs. Ce choix avait été fait pour faciliter l'usage, l'apprentissage et 
l'adoption de l'\gls{IU} et s'est révélé anecdotique lors de l'expérimentation car le 
panel de personne de participants était familier de ce genre 
d'environnement (échantillon faible).
Les fonctionnalités liées à la manipulation d'objets reçoivent une bonne évaluation 
à l'exception de l'importation de modèles. Cela est dû, dans ce prototype, au 
fait que l'application ne peut digérer et transmettre des modèles trop lourds. Un 
participant a vu sa fenêtre \og geler\fg{} (navigateur Chrome) lors de l'épreuve 
\textit{Castle from peer}. Cela l'a amené à quitter la session et à revenir sur la 
scène pour pouvoir continuer de participer à la session collaborative. Sans 
perturber la session en cours pour ses collaborateur, le participant a pu revenir sur 
l'application, reprenant la collaboration avec les autres participants (rétablissement 
des connexions après un crash).En cela le participant a apprécié la robustesse de 
l'application sans être perturbé très longtemps par l'interruption.
Concernant la fluidité des manipulations et de la visualisation au sein de 
l'application durant la collaboration, les participants n'ont pas ressenti de latence 
perturbantes en général. Ils ont qualifié l'application de temps-réel plutôt 
qu'interactive. La variation du nombre d'utilisateur sur les différentes épreuves n'a 
pas altéré la qualité du rendu et de réseau pour le participant.

\subsection{Conclusion de l'expérimentation 1}

This paper proposed web-based 3D modeling collab- oration based on a hybrid 
communication architecture client-server and P2P network. The client is respon- 
sible for 3D rendering and handling the user interac- tions on a scene. It also hosts 
the peer connection to be able to communicate updates to other peers. The server 
is used to link the client with the NoSQL database in order to store the 
modifications, and manage the users presence on a scene and automatically 
create a P2P full mesh topology network between them. The P2P con- nection 
relies on a WebRTC communication that trans- mits information directly between 
browser with update messages, broadcasting according to the P2P star topol- ogy, 
using a signaling server to establish the commu- nication between two peers. The 
qualitative evalua- tions of the experiments were conclusive overall even if some 
points should be improved. On model import, the broadcast causes latency issues 
on client peers re- ceivers (camera freeze). To reduce latency with larger scenes 
we consider using progressive rendering and making a better use of peer-to-peer 
mesh to stream the model relying on seed peers like in the partial mesh topology. 
An improvement of interface features and visual feedback of collaborative 
manipulations was asked by the users. This evaluation will be supple- mented in 
future works with a quantitative evaluation to compare our hybrid architecture to 
others, particu- larly using server logs, FPS in client and WebRTC tools (Chrome : 
chrome://webrt-internal; Firefox : about:webrtc) which provides statistics and 
graphs on the data exchanged between peers’ browsers.