%!TEX root = main.tex
%\section{Assemblage d'objets 3D sur le web avec architecture 
%événementielle}
%\label{sec:us}

\subsection{Présentation de l'expérimentation 1}

L'expérimentation propose de répliquer une collaboration réaliste entre 
différents participants travaillant à distance sur des tâches d'assemblage de 
modèles \gls{3D}. L'expérimentation reprend le cas d'étude de l'assemblage 
d'objets \gls{3D} (Section \ref{sec:use_case}) et propose de répondre aux 
questions suivante : quel est l'apport d'une architecture hybride dans la 
collaboration temps réel ? Comment l'utilisateur s'approprie-t-il le prototype mis à 
disposition ? Est-ce que la collaboration rend la réalisation de la tâche plus rapide 
et / ou plus effective ? Un des objectifs de cette expérimentation est de montrer 
que le prototype réalisé sur la base d'une architecture hybride facilite la 
collaboration (échanges de mises à jour) sans perturber les habitudes des 
utilisateurs.

L'expérimentation est basée sur le prototype d'application 3DState développé 
selon l'implantation présentée dans la Section \ref{sec:3DState}.
Le prototype est développé sur une plateforme web et consiste en un application 
de modélisation \gls{3D} collaborative multi-utilisateurs pour démontrer la 
faisabilité de l'architecture présentée dans la Section \ref{sec:comm_state}. 


\paragraph{Tâches à effectuer.}
L'expérimentation consiste en quatre épreuves qui partagent la même procédure.
Les modèles utilisés dans chaque épreuve sont décrits dans le tableau 
\ref{table:models_xp1} : \textit{Wind Turbine}, \textit{Pick up} et \textit{Castle}. 

Les épreuves sur les modèles \textit{Wind Turbine} et \textit{Pick Up} ont le 
même objectif : 
assembler un modèle dont les utilisateurs possèdent l'image, composé de 
différentes pièces qui sont téléversées par les utilisateurs. L'image correspond à 
l'assemblage final à obtenir et indique les différentes parties qui le 
composent (type et quantité). Les épreuves s'arrêtent lorsque les participants 
le notifient à l'expérimentateur, ou durent dix minutes maximum chacune.

Le modèle \textit{Castle} est utilisé dans deux épreuves : 
\textit{Castle from server} et \textit{Castle from peer}. L'objectif de ces épreuves
diffère un peu des épreuves précédentes car l'épreuve s'effectue sur un  modèle 
de château en kit (composé d'une dizaine d'objets différents : tour, murs, 
escalier\dots) : les 
participants ont dix minutes pour construire un 
château de manière collaborative en faisant appel à leur créativité. 
Dans l'épreuve \textit{Castle from server}, les objets sont récupérés 
automatiquement à partir du serveur ; 
dans l'épreuve \textit{Castle from peer}, les pairs peuvent ajouter de nouveaux 
objets du kit qu'ils possèdent sur leur appareil.

En utilisant les outils de l'éditeur, les informations liées aux collaborateurs 
(représentation dans l'environnement \gls{3D} de leur position ou de leur sélection) permettent à un 
utilisateur de manipuler les pièces 3D (sélection, rotation, translation, 
homothétie) et les disposer de manière à atteindre l'objectif. 
\paragraph{Population}
L'expérience a été conduite sur trois groupes de deux, quatre, quatre et quatre 
participants chacun (cf Tableau \ref{table:models_xp1}). Une épreuve sur le 
modèle \textit{Wind Turbine} a été effectuée avec six participants 
en simultané pour observer le comportement du système dans un cas non prévu.
Durant l'expérimentation, les utilisateurs étaient sur le même réseau que le serveur 
(\gls{LAN}). 
Les participants étaient des étudiants de Master ou de Doctorat en Informatique 
plutôt familiers avec les environnements \gls{3D}. Les participants étaient 
autorisés à 
communiquer entre eux (oralement, par chat\dots).
\paragraph{Procédure}
L'expérimentation consiste en quatre épreuves qui partagent la même procédure:
\begin{enumerate}
	\item Phase d'essai (10 minutes)
	\begin{enumerate}
		\item explication du contexte et de l'expérimentation 
		\item prise en main du système, familiarisation avec 
		l'interface 
	\end{enumerate}
	\item Phase de collaboration (10 minutes / collaboration): réalisation de 
	l'épreuve (se répète autant de fois 
	que d'épreuves)
	\begin{enumerate}
		\item Présentation de l'épreuve et de son objectif
		\item Initialisation : nettoyage de la scène et chargement des modèles
		\item Réalisation de l'objectif
	\end{enumerate}
	\item Phase de retour d'expérience (10 minutes) : remplissage questionnaire et 
	discussion 
	informelle sur les épreuves.
\end{enumerate}

\paragraph{Initialisation}

L'épreuve est décrite et l'objectif de l'épreuve est présenté à tous les participants.
Pour chaque phase de collaboration, l'application nettoie les données issues de 
l'épreuve précédente. 
À l'exception des pièces du modèle \textit{Castle}, qui sont chargées
en totalité sur le serveur dans \textit{Castle from server} et les pairs dans 
\textit{Castle from peer}, les pièces du modèle de l'épreuve 
sont distribuées de manière aléatoire entre les différents participants.

\paragraph{Données collectées}
Entre chaque épreuve, les 
participants ont exprimé certains retours qualitatifs dont il a été pris note. Les modalités d'interaction entre les participants sont observées lors des épreuves.

\paragraph{Questionnaire}
Au début de l'expérimentation, les participants reçoivent et prennent 
con\-naissance 
du questionnaire (voir Annexe \ref{q:xp1}). Ce questionnaire est rempli à différents 
moments au cours de l'expérimentation. Au début, le participant décrit sa situation, 
sa familiarité avec les environnements \gls{3D}, le type de machine et de 
navigateur qu'il utilise. À la fin de chaque épreuve, il répond aux questions concernant 
l'épreuve qu'il vient de passer. À la fin de l'expérimentation, il répond aux 
questions d'ordre plus général sur son expérience et les améliorations 
envisageables.
\subsection{Résultats et discussion}
Les participants ont globalement été satisfaits des résultats issus de la 
collaboration, ainsi que du rendu visuel des assemblages effectués durant les 
épreuves. Cela s'explique principalement par le fait qu'ils ont atteint les 
objectifs fixés à chaque fois (dans le temps imparti) : réalisation de l'assemblage du modèle \gls{3D} 
présenté sans frustration. Ils se sont également \og amusés\fg{} sur les épreuves
avec le \textit{Castle} car ils étaient libres dans la création et ont même souhaité 
continuer l'épreuve au-delà du temps imparti. 

L'utilisation de canaux de communication 
externes à l'application a été reportée plusieurs fois, principalement sous la forme 
d'échanges oraux. Ces échanges concernaient l'état de leur application et ce qu'ils 
étaient en train de faire ou ce qu'ils souhaitaient faire. Cette interaction provient, 
d'après les participants, du manque de retours visuels différenciés sur les 
manipulations effectués par les collaborateurs (exemple: pas de couleur différente 
pour la sélection d'objet par un autre utilisateur). Cela peut plus généralement 
s'interpréter comme un manque de sensibilisation à l'environnement collaboratif. 


L'\gls{IU} a été bien appréciée, parfois jugée \og trop simple\fg{} par certains 
utilisateurs. Ce choix avait été fait pour faciliter l'usage, l'apprentissage et 
l'adoption de l'\gls{IU} et s'est révélé anecdotique lors de l'expérimentation car le 
panel était familier de ce genre d'environnement.

Les fonctionnalités liées à la manipulation d'objets reçoivent une bonne évaluation, 
à l'exception de l'importation de modèles. Cela est dû, dans ce prototype, au 
fait que l'application ne peut traiter et transmettre des modèles trop lourds. Un 
participant a vu sa fenêtre \og geler\fg{} (navigateur Chrome) lors de l'épreuve 
\textit{Castle from peer}. Cela l'a mené à quitter la session et à revenir sur la 
scène pour pouvoir continuer de participer à la session collaborative. Sans 
perturber la session en cours pour ses collaborateurs, le participant a pu revenir sur 
l'application, reprenant à la volée la collaboration avec les autres participants 
(rétablissement 
des connexions après un crash). En cela le participant a apprécié la robustesse de 
l'application, sans être perturbé très longtemps par l'interruption.

Concernant la fluidité des manipulations et de la visualisation au sein de 
l'application durant la collaboration, les participants n'ont pas ressenti de latence 
excessive (au-delà de 10s). Ils ont qualifié l'application de \og temps réel\fg{} plutôt 
que d'\og interactive\fg{}. La variation du nombre d'utilisateurs sur les différentes 
épreuves n'a pas altéré la qualité du rendu et de réseau pour le participant.

\subsection{Conclusion de l'expérimentation 1}


Cette expérimentation présente différentes épreuves sur le cas d'étude lié à cette 
thèse. Elle repose sur un modèle utilisant une architecture de communication 
hybride combinant client-serveur et \gls{P2P}. Le client est responsable de 
proposer un environnement \gls{3D} intégré à l'interface, permettant la 
manipulation et la visualisation des objets \gls{3D} manipulés de manière 
collaborative. Afin de pouvoir collaborer, il héberge également les connexions 
nécessaires pour communiquer ses mises à jour vers les autres pairs 
(autant de connexions WebRTC DataChannel que de pairs) et vers la base de données
via le serveur (une seule connexion WebSocket). Les modifications sont transmises
sous forme de différentiels d'état et sont stockées sur les pairs localement et 
sur la base de données de manière distante. 
Le réseau \gls{P2P} est composé uniquement de clients producteurs de 
données reliés de manière complète. 

Les évaluations liées à l'expérimentation sont plutôt encourageantes, même si 
certains points doivent être améliorés. Concernant la fonctionnalité permettant 
d'importer des géométries, leur transmission dans le réseau \gls{P2P} a causé 
des latences sur les pairs destinataires (gel de la fenêtre). 
Afin de réduire la latence lors de la transmission de larges scènes, deux alternatives 
sont à étudier : l'utilisation d'un 
rendu progressif et compressé ainsi qu'une meilleure répartition de la charge entre 
les pairs pour la distribution des données. Dans ce dernier cas, l'utilisation d'une 
topologie maillée partiellement et d'une distribution des modèles incluant les pairs 
permettraient de donner plus de responsabilité aux pairs dans la transmission des 
données en désengorgeant
un peu le serveur et surtout le réseau \gls{P2P}.

Concernant l'amélioration des fonctionnalités proposées par l'\gls{IU}, 
l'ajout de retours visuels liés aux interactions collaboratives, 
ainsi que la visualisation de l'historique sont 
indispensables. Une évaluation quantitative approfondie pourrait permettre de 
mieux examiner les apports de l'architecture de communication hybride par rapport 
aux architectures classiques dans le contexte de la modélisation \gls{3D} 
collaborative. 
Cela pourrait se concrétiser par l'examen de la collaboration en récupérant le 
journal des actions, l'impact sur la visualisation (Frame Per Second) et 
l'observation des échanges WebRTC en entrées / sorties.

%An improvement of interface features and visual feedback of collaborative 
%manipulations was asked by the users. This evaluation will be supple- mented in 
%future works with a quantitative evaluation to compare our hybrid architecture to 
%others, particu- larly using server logs, FPS in client and WebRTC tools (Chrome 
%: 
%chrome://webrt-internal; Firefox : about:webrtc) which provides statistics and 
%graphs on the data exchanged between peers’ browsers.