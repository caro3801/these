\subsection{Intergiciel pour l'échange de 3D par événements}

L'architecture de communication décrite dans le chapitre précédent nécessite 
l'implémentation d'un intergiciel \gls{P2P} compatible avec les besoins liés à la 
3D, le web et la collaboration. Cette section décrit les politiques de connexion 
pour l'éditeur présenté dans la section précédente ainsi que les mécanismes de 
synchronisation nécessaires à l'échange de données dans l'application. 

\subsubsection{\textit{Signaling} et connectivité}
L'assomption est faite que tous les clients utilisent des navigateurs qui 
implémentent et supportent le protocole WebSocket et l'\gls{API} 
RTCDataChannel. 

La topologie de l'architecture de communication repose sur la mise en relation 
automatique des clients par le biais du serveur pour établir une connexion 
\gls{WebRTC}. Pour ce faire, chaque client envoie son identifiant (ID) lors de sa 
première requête au serveur qui le stocke. Selon le paramétrage de la connectivité 
directe minimum établie préalablement, le serveur recherche aléatoirement l'ID 
d'autres clients qui satisfont la règle de connectivité. Cette règle de connectivité 
minimum permet d'ajuster la densité du maillage (connectivité élevée: maillage 
partiel dense, voire complet ; connectivité faible: maillage partiel éparse) et 
d'obtenir une topologie maillée adaptée aux besoins de l'application en termes de 
synchronisation (temps-réel ou non) ou aux capacités des appareils. Il faut noter 
cependant que plus la connectivité est faible, plus l'information a besoin de 
transiter par des intermédiaires pour atteindre tous les pairs et par conséquent le 
temps de transmission est augmenté (exemple d'une distribution en ligne). 


 C'est donc la partie gestion des instances du 
serveur qui gère la politique de connexion des données lors de la phase de \textit{signaling}. Dans le prototype réalisé, 
la connectivité minimum entre les différents pairs participant à la même scène est 
fixée à 2. Cette politique permet de créer un maillage réseau qui n'est pas trop 
dense en considérant que le nombre maximum d'utilisateur ne dépassera pas dix.

\subsubsection{De navigateur à navigateur}
Lors de la connexion d'un nouveau client à la scène, le serveur effectue la phase 
de \textit{signaling} permettant de le mettre en relation avec un autre client. Le 
mécanisme est répété tant que la règle de connectivité peut s'appliquer. Le client 
reçoit une notification de l'établissement de la connexion avec un autre client ce 
qui lui permet de démarrer l'échange de données.

Dans 3DEvent, le choix d'avoir un transport fiable et non ordonné a été fait pour 
respectivement garantir l'arrivée d'une donnée émise par l'utilisateur au sein de 
l'application et permettre des échanges asynchrones.
En cas d'arrêt soudain du serveur, si une connexion a été établie préalablement 
entre les clients et est toujours en cours, elle n'est pas impactée par la défaillance 
du serveur.
\subsubsection{Données d'échanges}

En sachant que le modèle est conçu pour des applications web, 3DEvent a besoin 
d'un format de données permettant de faire communiquer des acteurs hétérogènes 
du système.
Le format \gls{JSON} est assez générique pour être aussi utilisé lors de la 
sérialisation et la désérialisation des 
événements et plus largement des paquets transmis par RTCDataChannel. 
L'\acrshort{API} RTCDataChannel supporte beaucoup de types de données 
différents (chaînes de caractères, types binaires : Blob, ArrayBuffer\dots). Dans 
un environnement multi-utilisateurs tel que 3DEvent avec des données 
hétérogènes (3D, images, 
informations) l'interopérabilité est facilitée.

Le Listing  \ref{jsonstore} représente la structure de données utilisée sur le réseau 
lorsqu'un événement est transmis. Seule la partie $data$ est utilisée par le client
pour faire les manipulations en \gls{CQRS} car le type de l'instance d'événement 
manipulé est connu. 

La structure représentant l'événement transporte un numéro de version \og 
attendue\fg{} 
qui correspond à la version que l'agrégat doit avoir lorsque l'événement lui est 
appliqué. De cette manière la cohérence interne du pair est maintenu grâce à ce 
numéro de version.
Les événements contenant les objets 3D ($importGeometryEvent$) ont une 
propriété contenant la géométrie sous la forme d'un Blob contenant le fichier. 

\begin{lstlisting}[language=json,firstnumber=1,label=jsonstore,caption=Format
du Message transitant sur le réseau contenant l'événement MeshAddedToScene 
et ses 
paramètres]
{
	"packetId" : "567826g6-766b-f0g9676b677r",
	"streamId" : "scene-turbine",
	"eventType" : "MeshAddedToSceneEvent",
	"version" : 17,
	"data" : {
		"sceneId": "scene-turbine",
		"meshId":"406514c6-306b-f0f9643a037e",
		"geometryId": "37076875-ea1c-bbd300481345",
		"name": "blade",
		"color": "#963912",
		"author": "Foo"
	}
}
\end{lstlisting}


\subsubsection{Synchronisation des données}
\paragraph{Persistance à court terme}
Le navigateur (client) offre un espace de stockage avec l'interface \textit{Storage} 
de l'API Web Storage qui donne accès au \texttt{session storage} ou au  
\texttt{local storage}. Ce stockage est présent sur le client et fonctionne sur un 
système de clé / valeur. La configuration du client est également 
stockée localement. Grâce à ce système, il est possible d'avoir une 
persistance des données à travers les sessions du navigateur. Le contenu stocké 
correspond aux données générées par un utilisateur et par ses collaborateurs. Les 
répliques stockées sur chaque navigateur permettent une plus 
grande 
autonomie des utilisateurs en cas de déconnexion. C'est également un moyen de distribuer entre 
les clients les 
mises à jour qu'ils génèrent grâce au protocole de 
\gls{streaming3D} (cf. \ref{streamingprotocol}) sans passer par le serveur.

\paragraph{Protocole de streaming pour la synchronisation}
\label{streamingprotocol}

Il existe plusieurs méthodes de transmission de contenu au sein d'un réseau 
\gls{P2P} que l'on peut catégoriser selon deux modes : le téléchargement 
(\textit{download}) 
et le flux continu (\textit{streaming}). Le téléchargement requière que le contenu 
soit entièrement téléchargé pour pouvoir être lu, tandis que le flux continu peut 
être lu au fur et à mesure de sa récupération. Ce dernier mode, principalement 
utilisé pour la lecture de vidéo en ligne, s'applique bien à la transmission de 
contenu \gls{3D} : niveaux de détails \cite{Chu2012,Hu2008}, progressif 
\cite{Cheng2009,Limper2014}, mise en cache \cite{Jia2014}, compression / 
décompression
\cite{Lavoue2013,Ponchio2015,Maglo2013a}. 
Une catégorisation plus précise du flux continu peut être donnée selon la manière 
dont le contenu est généré : en direct (\textit{live} ou \textit{push}) ou à la demande 
(\textit{on-demand} ou \textit{pull}).  

Les requêtes sur le contenu que les n\oe uds doivent récupérer sont effectuées selon 
un ordre de priorité déterminé par l'intérêt que les n\oe uds portent au contenu. Par exemple, 
les scènes sont récupérées en priorité par rapport aux maillages et aux 
informations des autres utilisateurs. 

Les données sont transmises sous forme de morceaux (\textit{chunk}) de 16 
octets maximum car Chrome impose une limite sur la taille des messages 
envoyés. L'application se charge donc de la création des morceaux et de les 
envoyer. Ceci a été intégré dans le protocole d'échange de l'application. C'est une 
des raisons qui a encouragé le choix d'un transport non fiable et non ordonné car 
l'application est déjà en charge de manipuler les données qui vont transiter sur le 
réseau. D'autre part, le standard WebRTC n'étant pas implanté de la même 
manière sur tous les navigateurs, le fait déléguer une partie de la gestion des 
données à l'application permet de mieux gérer le traitement qu'elles subissent. Les 
effets collatéraux à ce choix sont doubles : la taille des messages qui transitent 
est allégée, cependant il faut d'envoyer plus de messages pour maintenir la 
 la cohérence et la synchronisation.


\subsubsection{Gestion des événements dans le flux d'un agrégat}

Une connexion RTCDataChannel peut être configurée selon 
les critères de fiabilité et d'ordonnancement. 
Dans l'intergiciel de 3DEvent, cette configuration est : non fiable et non ordonnée. 
C'est l'application qui est en charge de \og ré-ordonner\fg{} les événements. En 
effet, les événements sont associés à des agrégats. Comme chaque agrégat 
possède un flux d'événements dédié et numéroté il est alors simple pour 
l'application de les 
ordonner lorsqu'elle les réceptionne. 
Lors de la synchronisation des pairs, les 
événements manquants sont demandés successivement à tous les pairs. Lors 
qu'un pair reçoit les événements il les stocke dans le flux correspondant à 
l'agrégat. Si un événement est manquant, les suivants sont quand même stockés 
en laissant l'espace de l'événement manquant dans le tableau. L'événement est 
redemandé jusqu'à son obtention, laquelle provoque le déblocage de l'état de 
l'agrégat jusqu'au prochain événement manquant (ou la fin du flux). Les 
événements qui ont été stockés \og en attendant\fg{} permettent à l'application 
d'être directement en capacité de poursuivre la construction de l'état de l'agrégat. 
Cette mécanique tire avantage des Snapshots (présentés dans \ref{sec:snapshot}) 
qui réduisent la taille des flux en mémoire pour chaque agrégat qui part déjà d'un 
état avancé.


\begin{figure}
	\centering
	\inputTikZ{0.9}{eps/tikz/streams/aggregate.tex}
	\caption{Exemple d'agrégats}{Structure d'un agrégat et ses versions. Chaque 
		version est un état de l'agrégat qui correspond à l'empilement des instances 
		d'événements (ei) qu'il contient. Les types des événements sont relatifs au 
		type 
		d'agrégat dans lequel il est contenu.}
	\label{fig:aggregate}
\end{figure}