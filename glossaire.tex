%!TEX root = _main.tex
%definition glossaire

\newacronym{3D}{3D}{trois dimensions}

\newacronym{AIC}{AIC}{Architecture, Ingénierie et Construction}
\newacronym[longplural={Application Programming 
Interfaces},plural=APIs]{API}{API}{Application Programming Interface}

\newacronym{BI}{BI}{Business Intelligence}
\newacronym{BIM}{BIM}{Business Information Modeling}
\newacronym{BSON}{BSON}{Binary JSON}

\newacronym{BSP}{BSP}{Binary Space Partitioning}


\newacronym{CAO}{CAO}{Conception Assistée par Ordinateur}
\newacronym{CAP}{CAP}{Consistency, Availability, and Partition Tolerance}
\newacronym{CCI}{CCI}{Causality, Convergence, Intention}
\newacronym{CEP}{CEP}{Complex Event Processing}
\newacronym{CDN}{CDN}{Content Delivery Network}
\newacronym{CDP}{CDP}{Cohérence, Disponibilité, et Tolérance au 
Partitionnement}
\newacronym{CE}{CE}{Cohérence Éventuelle}
\newacronym{CPU}{CPU}{Central Processing Unit}
\newacronym{CQRS}{CQRS}{Command Query Responsability Segregation}
\newacronym[longplural={Commutative Replicated Data 
	Types },plural=CRDTs]{CRDT}{CRDT}{Conflict-Free Replication Data 
	Type}
\newacronym{CRUD}{CRUD}{Create, Read, Update, Delete}
\newacronym{CS}{CS}{Command Sourcing}
\newacronym{CSG}{CSG}{Constructive Geometry Solid}
\newacronym{CSCW}{CSCW}{Computer-Supported 
	Cooperative Work}
\newacronym{CSS}{CSS}{Cascading Style Sheet}

\newacronym{DDD}{DDD}{Domain Driven Design}
\newacronym{DDS}{DDS}{Data Distribution Service}
\newacronym{DTLS}{DTLS}{Datagram Transport Layer Security}
\newacronym{DHT}{DHT}{Distributed Hash Table}
\newacronym{DOM}{DOM}{Document Object Model}
\newacronym{DTO}{DTO}{Data Transfer Object}


\newacronym{ECMA}{ECMA}{Ecma International - European association for 
standardizing information and communication systems}
\newacronym{EDA}{EDA}{Event Driven Architecture}
\newacronym{EP}{EP}{Event Processing}
\newacronym{ES}{ES}{Event Sourcing}
\newacronym{ESM}{ESM}{Event Stream Manager}
\newacronym[longplural={Environnements 
Virtuels},plural=EVs]{EV}{EV}{Environnement Virtuel}

\newacronym[longplural={Environnements Virtuels 
Collaboratifs},plural=EVCs]{EVC}{EVC}{Environnement Virtuel Collaboratif}
\newacronym[longplural={Environnements Virtuels Collaboratifs 3D},plural=EVCs 
3D]{EVC3D}{EVC 3D}{Environnement Virtuel Collaboratif 3D}
\newacronym[firstplural=Environnements Virtuels 3D (EV 3D),plural=EV 
3D]{EV3D}{EV 3D}{Environnement Virtuel 3D}
\newacronym[longplural={Collaborative Virtual 
Environments},plural=CVEs]{CVE}{CVE}{Collaborative Virtual 
	Environment}


\newacronym{glTF}{glTF}{GL Transmission Format}
\newacronym{GPU}{GPU}{Graphic Processing Unit}

\newacronym{HTML}{HTML}{HyperText Markup Langage}

\newacronym{ICE}{ICE}{Interactive Connectivity Establishment}
\newacronym{IETF}{IETF}{Internet Engineering Task Force}
\newacronym{IHM}{IHM}{Interface Homme Machine}
\newacronym{IU}{IU}{Interface Utilisateur}
\newacronym{UI}{UI}{User Interface}
\newacronym{IM}{IM}{Instance Manager}

\newacronym{JS}{JS}{JavaScript}
\newacronym{JSON}{JSON}{JavaScript Object Notation}

\newacronym{LAN}{LAN}{Local Area Network}
\newacronym{MAN}{MAN}{Metropolitan Area Network}
\newacronym{WAN}{WAN}{Wide Area Network}



\newacronym{NAT}{NAT}{Network Address Translator}
\newacronym[longplural={Network Bridges}, 
plural=NBs]{NB}{NB}{Network Bridge}

\newacronym[longplural={Operational Transformations }, 
plural=OTs]{OT}{OT}{Operational Transformation}
\newacronym{ORM}{ORM}{Object-Relational Mapping}
\newacronym{OWL}{OWL}{Web Ontology Language}

\newacronym{P2P}{P2P}{pair à pair}
\newacronym{PDM}{PDM}{Product Data Management}
\newacronym{PLM}{PLM}{Product Lifecycle Management}
\newacronym{PubSub}{Pub / Sub}{Publish-Subscribe}

\newacronym{REST}{REST}{REpresentational State Transfer}
\newacronym{RRTCC}{RRTCC}{Receiver-side Real-Time Congestion 
Control}
\newacronym{RTP}{RTP}{Real-Time Transport Protocol}
\newacronym{RTCP}{RTCP}{Real-time Transfert Control Protocol}

\newacronym{RV}{RV}{Réalité Virtuelle}

\newacronym[longplural={Systèmes d'Edition 
Collaborative},plural=SEC]{SEC}{SEC}{Système d'Edition Collaborative}
\newacronym[longplural={Systèmes d'Edition Collaborative en 
	P2P},plural={SEC P2P}]{SECP2P}{SEC P2P}{Système d'Edition 
	Collaborative en P2P}

\newacronym{SVG}{SVG}{Scalable Vector Graphics}
\newacronym{SIG}{SIG}{Système d'Information Géographique}
\newacronym{STUN}{STUN}{Simple Traversal of UDP through NATs}
\newacronym{SCTP}{SCTP}{Stream Control Transmission Protocol}
\newacronym{TCP}{TCP}{Transmission Control Protocol}
\newacronym{TLS}{TLS}{Transport Layer Security}
\newacronym{TURN}{TURN}{Traversal Using Relays around NAT}

\newacronym{UDP}{UDP}{User Datagram Protocol}
\newacronym[longplural={Uniform Resource 
Identifiers},plural=URIs]{URI}{URI}{Uniform Resource Identifier}

\newacronym{W3C}{W3C}{World Wide Web Consortium}
\newacronym{WebRTC}{WebRTC}{Web Real-Time Communication}

 \newglossaryentry{XHR}{
	name=XHR,
	description={(abbréviation pour XMLHttpRequest) est un objet du navigateur accessible en JavaScript qui permet d'obtenir des données à l'aide de requêtes HTTP}
}
 \newglossaryentry{SIP}{
	name=SIP,
	description={(abbréviation pour Session Initiation Protocol) est est un protocole standard ouvert de gestion de sessions.}
}

\newacronym{NoSQL}{NoSQL}{Not Only SQL}
\newglossaryentry{WebSocket}{
	name=WebSocket,
	description={Protocole réseau standard du Web visant à créer des canaux de communication full-duplex par dessus une connexion TCP}
}

\newglossaryentry{EventStore}{
	name=Event Store,
	description={Mémoire permettant le stockage des évènements en mode 
	"\textit{append-only}"}
}

\newglossaryentry{snapshot}{
	name=Snapshot,
	description={(contexte : CQRS) Instantané de l'état d'un agrégat convertissant 
	l'objet du domaine en un objet de transferts de données (\gls{DTO})}
}
\newglossaryentry{RTPdef}
{
	name=Real-Time Transport Protocol,
	description={est un protocole de 
	communication permettant le transport de données soumises à des 
	contraintes de temps réel (flux média audio ou vidéo). Le protocole 
	ajoute un en-tête spécifique aux paquets UDP pour spécifier le type et 
	le format (codec) du média transporté, numéroter les paquets, et fournir 
	une indication d'horloge.}
}

\newglossaryentry{RTCPdef}
{
	name=Real-time Transfert Control Protocol,
	description={est un protocole de contrôle des flux \gls{RTP} qui transmet de 
	manière périodique par tous les participants des paquets de contrôle sur 
	une session (informations basiques sur les participants d'une session, sur 
	la qualité de service)}
}


\newglossaryentry{streaming3D}{
	name=Streaming 3D,
	text={streaming 3D},
	description={est une technique permettant d'envoyer un contenu 3D sous la 
	forme d'un flux continu par le biais d'internet, qui peut être utilisé/lu au fur et à 
	mesure qu'il est reçu}
}

\newglossaryentry{framework}{
	name=Framework,
	text={framework},
	plural={frameworks},
	description={(ou structure logicielle en français) est un ensemble de 
	composants génériques proposant un cadre de travail guidant l'architecture 
	logicielle}
}


\newglossaryentry{workflow}{
	name=Workflow,
	text={workflow},
	plural={workflow},
	description={(ou flux de travaux en français) est la représentation d'une suite 
	d'opérations effectuées par une entité (personne, groupe\ldots). L'expression 
	renvoie au passage d'une ressource d'une étape à l'autre}
}

\newglossaryentry{computer}{
	name=computer,
	description={is a programmable machine that receives input,
	stores and manipulates data, and provides
	output in a useful format}
}


\newglossaryentry{NATT}{
	name=NAT Traversal,
	description={est une technique pour établir et maintenir les connexions internet 
	à travers les passerelles qui implémentent \gls{NAT}. \gls{NAT} casse le 
	principe de connectivité de bout en bout originalement envisagée lors de la 
	conception d'internet.}
}

\newglossaryentry{ICEF}{
	name=ICE Framework,
	description={est une technique dans le domaine du réseau pour trouver le 
	chemin entre deux machine pour communiquer l'une avec l'autre, le plus 
	directement possible en P2P. Le framework permet aux pairs en cours 
	d'établissement de connexion de découvrir et communiquer leur adresse IP 
	publique dans le but de pouvoir être atteint par d'autres pairs.}
}