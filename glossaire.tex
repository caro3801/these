%!TEX root = _main.tex
%definition glossaire

\newacronym[longplural={Application Programming 
Interfaces},plural=APIs]{API}{API}{Application Programming Interface}
\newacronym{SVG}{SVG}{Scalable Vector Graphics}
\newacronym{SIG}{SIG}{Système d'Information Géographique}
\newacronym{CSS}{CSS}{Cascading Style Sheet}

\newacronym{RRTCC}{RRTCC}{Receiver-side Real-Time Congestion 
Control}

\newacronym{CRUD}{CRUD}{Create, Read, Update, Delete}
\newacronym{REST}{REST}{REpresentational State Transfer}
\newacronym{DTLS}{DTLS}{Datagram Transport Layer Security}
\newacronym{WebRTC}{WebRTC}{Web Real-Time Communication}
\newacronym{NAT}{NAT}{Network Address Translator}
\newacronym{STUN}{STUN}{Simple Traversal of UDP through NATs}
\newacronym{TURN}{TURN}{Traversal Using Relays around NAT}
\newacronym{CSCW}{CSCW}{Computer-Supported 
	Cooperative Work}

\newacronym[longplural={Environnements Virtuels Collaboratifs},plural=EVCs]{EVC}{EVC}{Environnement Virtuel Collaboratif}

\newacronym[longplural={Environnements Virtuels},plural=EVs]{EV}{EV}{Environnement Virtuel}

\newacronym[longplural={Environnements Virtuels Collaboratifs 3D},plural=EVCs 
3D]{EVC3D}{EVC 3D}{Environnement Virtuel Collaboratif 3D}
\newacronym[firstplural=Environnements Virtuels 3D (EV 3D),plural=EV 3D]{EV3D}{EV 3D}{Environnement Virtuel 3D}
\newacronym[longplural={Collaborative Virtual Environments},plural=CVEs]{CVE}{CVE}{Collaborative Virtual 
Environment}
\newacronym{RV}{RV}{Réalité Virtuelle}
\newacronym{IHM}{IHM}{Interface Homme-Machine}
\newacronym{DTO}{DTO}{Data Transfer Object}
\newacronym{ICE}{ICE}{Interactive Connectivity Establishment}
\newacronym[longplural={Systèmes d'Edition Collaborative en 
P2P},plural={SEC P2P}]{SECP2P}{SEC P2P}{Système d'Edition 
Collaborative en P2P}
\newacronym[longplural={Systèmes d'Edition Collaborative},plural=SEC]{SEC}{SEC}{Système d'Edition Collaborative}

\newacronym{CE}{CE}{cohérence éventuelle}
\newacronym{LAN}{LAN}{Local Area Network}
\newacronym{MAN}{MAN}{Metropolitan Area Network}
\newacronym{WAN}{WAN}{Wide Area Network}

\newacronym{PLM}{PLM}{Product Lifecycle Management}
\newacronym{PDM}{PDM}{Product Data Management}

\newacronym{DDS}{DDS}{Data Distribution Service}

\newacronym{UDP}{UDP}{User Datagram Protocol}
\newacronym{CAO}{CAO}{Conception Assistée par Ordinateur}
\newacronym{BIM}{BIM}{Business Information Modeling}

\newacronym{BSP}{BSP}{Binary Space Partitioning}
\newacronym{CSG}{CSG}{Constructive Geometry Solid}

\newacronym{3D}{3D}{tridimension}
\newacronym{CAP}{CAP}{Consistency, Availability, and Partition Tolerance}
\newacronym{CDP}{CDP}{Cohérence, Disponibilité, et Tolérance au partitionnement}

\newacronym{DHT}{DHT}{Distributed Hash Table}
\newacronym{TLS}{TLS}{Transport Layer Security}
\newacronym{TCP}{TCP}{Transmission Control Protocol}
\newacronym{SCTP}{SCTP}{Stream Control Transmission Protocol}
\newacronym{W3C}{W3C}{World Wide Web Consortium}
\newacronym{IETF}{IETF}{Internet Engineering Task Force}
\newacronym{P2P}{P2P}{pair à pair}
\newacronym{GPU}{GPU}{Graphic Processing Unit}
\newacronym{CPU}{CPU}{Central Processing Unit}
\newacronym{JSON}{JSON}{JavaScript Object Notation}
\newacronym{DOM}{DOM}{Document Object Model}
\newacronym{BSON}{BSON}{Binary JSON}
\newacronym{glTF}{glTF}{GL Transmission Format}
\newacronym{OWL}{OWL}{Web Ontology Language}
\newacronym{BI}{BI}{Business Intelligence}
\newacronym[longplural={Operational Transformations }, 
plural=OTs]{OT}{OT}{Operational Transformation}
\newacronym[longplural={Commutative Replicated Data 
Types },plural=CRDTs]{CRDT}{CRDT}{Conflict-Free Replication Data 
Type}
\newacronym[longplural={Uniform Resource Identifiers},plural=URIs]{URI}{URI}{Uniform Resource Identifier}


 \newglossaryentry{XHR}{
	name=XHR,
	description={(abbréviation pour XMLHttpRequest) est un objet du navigateur accessible en JavaScript qui permet d'obtenir des données à l'aide de requêtes HTTP}
}
 \newglossaryentry{SIP}{
	name=SIP,
	description={(abbréviation pour Session Initiation Protocol) est est un protocole standard ouvert de gestion de sessions.}
}


\newacronym{HTML}{HTML}{HyperText Markup Langage}
\newacronym{JS}{JS}{JavaScript}
\newacronym{PubSub}{Pub / Sub}{Publish-Subscribe}
\newacronym{ES}{ES}{Event Sourcing}
\newacronym{CS}{CS}{Command Sourcing}
\newacronym{CQRS}{CQRS}{Command Query Responsability Segregation}
\newacronym{EP}{EP}{Event Processing}
\newacronym{CCI}{CCI}{Causality, Convergence, Intention}

\newacronym{CDN}{CDN}{Content Delivery Network}
\newacronym{EDA}{EDA}{Event Driven Architecture}

\newacronym{DDD}{DDD}{Domain Driven Design}
\newacronym{UI}{UI}{User Interface}
\newacronym{IU}{IU}{Interface Utilisateur}
\newacronym{IHM}{IHM}{Interface Homme Machine}
\newacronym{AIC}{AIC}{Architecture, Ingénierie et Construction}
\newacronym{ORM}{ORM}{Object-Relational Mapping}

\newacronym{NoSQL}{NoSQL}{Not Only SQL}
\newacronym{RTP}{RTP}{Real-Time Transport Protocol}
\newacronym{RTCP}{RTCP}{Real-time Transfert Control Protocol}
\newacronym{CEP}{CEP}{Complex Event Processing}
\newglossaryentry{WebSocket}{
	name=WebSocket,
	description={Protocole réseau standard du Web visant à créer des canaux de communication full-duplex par dessus une connexion TCP}
}

\newglossaryentry{cloud}{
	name=cloud,
	description={Définir le cloud}
}
\newglossaryentry{EventStore}{
	name=Event Store,
	description={Mémoire permettant le stockage des évènements en mode 
	"\textit{append-only}"}
}

\newglossaryentry{snapshot}{
	name=Snapshot,
	description={(contexte : CQRS) Instantané de l'état d'un agrégat convertissant 
	l'objet du domaine en un objet de transferts de données (\gls{DTO})}
}
\newglossaryentry{RTPdef}
{
	name=Real-Time Transport Protocol,
	description={est un protocole de 
		communication permettant le transport de données soumises à des 
		contraintes de temps réel (flux média audio ou vidéo). Le protocole 
		ajoute un en-tête spécifique aux paquets UDP pour spécifier le type et 
		le format (codec) du média transporté, numéroter les paquets, et fournir 
		une indication d'horloge.}
}

\newglossaryentry{RTCPdef}
{
	name=Real-time Transfert Control Protocol,
	description={est un protocole de contrôle des flux \gls{RTP} qui transmet de 
		manière périodique par tous les participants des paquets de contrôle sur 
		une session (informations basiques sur les participants d'une session, sur 
		la qualité de service)}
}


\newglossaryentry{streaming3D}{
	name=Streaming 3D,
	text={streaming 3D},
	description={est une technique permettant d'envoyer un contenu 3D sous la forme d'un flux continu par le biais d'internet, qui peut être utilisé/lu au fur et à mesure qu'il est reçu}
               }

               \newglossaryentry{framework}{
	name=Framework,
	text={framework},
	plural={frameworks},
	description={(ou structure logicielle en français) est un ensemble de composants génériques proposant un cadre de travail guidant l'architecture logicielle}
               }
               
               
               \newglossaryentry{workflow}{
               	name=Workflow,
               	text={workflow},
               	plural={workflow},
               	description={(ou flux de travaux en français) est la représentation d'une suite d'opérations effectuées par une entité (personne, groupe\ldots). L'expression renvoie au passage d'une ressource d'une étape à l'autre}
               }
               
              \newglossaryentry{computer}{
	name=computer,
	description={is a programmable machine that receives input,
               stores and manipulates data, and provides
               output in a useful format}
               }

   
              \newglossaryentry{NATT}{
              	name=NAT Traversal,
              	description={est une technique pour établir et maintenir les connexions internet à travers les passerelles qui implémentent \gls{NAT}. \gls{NAT} casse le principe de connectivité de bout en bout originalement envisagée lors de la conception d'internet.}
              }
          
                        \newglossaryentry{ICEF}{
          	name=ICE Framework,
          	description={est une technique dans le domaine du réseau pour trouver le chemin entre deux machine pour communiquer l'une avec l'autre, le plus directement possible en P2P. Le framework permet aux pairs en cours d'établissement de connexion de découvrir et communiquer leur adresse IP publique dans le but de pouvoir être atteint par d'autres pairs.}
          }

          
              
             