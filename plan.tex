\documentclass[final,10pt,doubleside]{book}
\usepackage[]{../these/manuscriptThese}

\begin{document}
	\completetable

\chapter{Contexte}

\section{La collaboration}

	\subsection{Les systèmes collaboratifs}
	\subsection{Les systèmes d'édition collaboratifs}
		\subsubsection{Modèle d'édition collaborative}

\section{La collaboration 3D en accord avec l'évolution du web}

	\subsection{Introduction}
	\subsection{Le web et le P2P : WebRTC}
	\subsection{Le web et la 3D :  WebGL}

\section{Les architectures évènementielles pour la collaboration}
	\subsection{Sensibilisation lors de la collaboration}
	\subsection{Intégration des contraintes métiers}


\chapter{État de l'art}

\section{La visualisation et manipulation 3D collaborative sur web}
	\subsection{Introduction}
	\subsection{Les approches centralisées}
	\subsection{Les approches décentralisées}
	\subsection{Conclusion}

\section{Systèmes d'édition collaborative}
	\subsection{Le modèle de cohérence CCI}
	\subsection{Les approches pour les données 3D}
	\subsection{Conclusion}


\section{Les systèmes évènementiels distribués pour la collaboration}

	\subsection{Introduction}
		\subsubsection{Les évènements comme base du comportement réactif}
		\subsubsection{Systèmes Publish / Subscribe}
		\subsubsection{Outils de monitoring et de benchmark}
		
	\subsection{DDD}
	\subsection{CQRS}
	\subsection{ES}
		\subsubsection{Définition}
		\subsubsection{ES vs AR}
		
	\subsection{Conclusion}


\chapter{Contributions scientifiques}

\section{Modèle évènementiel pour l'intégration du domaine 3D lors de la 
	manipulation d'objets 3D}
	\subsection{Introduction}
		\subsubsection{Constat}
		\subsubsection{Contribution}
	\subsection{Modèle général}
	
	

\section{Architecture de communication hybride}

	\subsection{Introduction}
		\subsubsection{Constat}
		\subsubsection{Contribution}
		
	\subsection{Présentation générale}
	
	\subsection{Gestion de la cohérence}
		\subsubsection{Respect de la causalité}
		\subsubsection{Convergence des répliques}
		\subsubsection{Préservation de l'intention}

\chapter{Implantation}

\section{3DEvent : Plateforme web de manipulation et visualisation 
collaborative 
d'objets 3D}

	\subsection{Introduction}
	\subsection{Interface utilisateur}
		\subsubsection{Présentation de l'interface}
		\subsubsection{Flexibilité de la visualisation}

\section{Intergiciel P2P pour l'échange de données 3D}

	\subsection{Introduction}
	\subsection{Interface utilisateur}
		\subsubsection{Présentation de l'interface}
		\subsubsection{Flexibilité de la visualisation}
		
\chapter{Expérimentations}

\section{Assemblage d'objets 3D sur le web par état}
	\subsection{Présentation de l''expérimentation}
	\subsection{Résultats}
	\subsection{Discussion et Conclusion}
	
\section{Assemblage d'objets 3D sur le web avec architecture évènementielle}
	\subsection{Présentation de l''expérimentation}
	\subsection{Résultats}
	\subsection{Discussion et Conclusion}
	
	\section{Comparaison entre l'expérimentation 1 et l'expériementation 2}
		\subsection{Résultats}
		\subsection{Discussion et Conclusion}
	
\chapter{Conclusion}
\end{document}