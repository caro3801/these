%!TEX root = main.tex
\chapter{Conclusion}
\chaptertable

Cette thèse présente une architecture évènementielle pour les \gls{EVC} sur le 
web. Les contributions de cette thèse sont multiples. Il y a d'une part les 
contributions scientifiques qui porte sur l'\gls{EDA} pour la 
modélisation 3D collaborative et d'autre part, il y a la présentation d'une 
architecture de communication hybride permettant de synchroniser les 
\glspl{EVC3D} sur le web de différents client.

Chacune de ces contributions est accompagnée de son implantation. 

De plus, la réalisation de plusieurs prototypes fonctionnant sur différents 
paradigme pour les expérimentation ont permis de montrer les avantages et 
inconvénients de ceux-ci.

Enfin, les expérimentations principalement portées sur l'étude des utilisateurs 
permet de souligner plusieurs aspects du modèle. Le premier est de montrer que 
l'intégration du métier a permis l'observation minutieuse du travail réalisé au sein 
de l'environnement. 