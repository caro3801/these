%!TEX root = main.tex
\chapter{Conclusion}
\chaptertable

Cette thèse s'inscrit dans le contexte la visualisation et de la manipulation d'objets 
3D dans un \gls{EVC} sur le web. Ce sujet intègre de nombreux verrous 
concernant différents domaines que la collaboration a tendance à réunir : gestion 
de la cohérence dans un environnement collaboratif massif en \gls{P2P}, 
architecture de communication efficace pour le passage à l'échelle et la résilience, 
modèle de données dédiée au métier de la modélisation 3D collaborative. 

\section{Discussion}
Pour répondre aux différentes questions de 
recherche posées en introduction de cette thèse, plusieurs contributions ont été 
proposées, implantées et soumises à des expérimentations, voir 
\cite{Desprat2015a,Desprat2015b,Desprat2016,Desprat2017}. 


%	\item[QR 1] Quelle architecture réseau est la plus adaptée pour une gestion 
%efficace, robuste et temps réel des données \gls{3D} dans un environnement 
%web ?
La \textbf{QR 1} concerne l'architecture réseau qu'un \gls{EVC3D} doit proposer 
pour 
pouvoir s'adapter à une gestion efficace, robuste et temps réel des données dans 
un environnement web. En effet, pour pouvoir collaborer, les utilisateurs ont besoin 
d'une architecture de communication qui les relie de manière transparente, en 
intégrant un modèle de cohérence des données. 

L'architecture de communication de l'\gls{EVC3D} a été l'objet de deux 
contributions successives, la seconde s'inspirant des verrous issus de la 
première. 
L'architecture hybride à maillage complet se compose d'éléments simples. 
Seuls les navigateurs des utilisateurs participent au réseau \gls{P2P}. Les 
données correspondent à l'état des objets manipulés. 
Les modifications sont transférées 
sous la forme d'un différentiel d'états. Bien que la politique de connectivité ne 
permette pas un passage à l'échelle efficace, la cohérence des données sur le 
réseau est garantie de manière forte et pessimiste respectivement par le protocole 
de transport, configuré comme fiable et ordonné, et par le mécanisme de 
verrouillage évitant l'édition concurrente d'objets 3D.

Partant de ces observations, la seconde contribution liée à l'architecture de 
communication repose sur le paradigme événementiel. Une politique de 
connectivité plus souple  permet aux pairs de devenir des relais des informations. La 
partie serveur est \og descendue\fg{} au niveau de la couche \gls{P2P} pour 
permettre un accès réparti des pairs producteurs de données (navigateurs) à la
persistance des données via des pairs relais. L'ensemble des pairs partage un 
journal d'événements qui est maintenu à jour et cohérent sur deux niveaux de 
vérification : au niveau du pair, avec un système de version lié à l'ordre causal, 
puis au niveau de la grappe, avec un système de vote lors de la détection de 
conflits pour accepter ou rejeter collectivement un événement. La cohérence 
repose sur une stratégie optimiste avec des contraintes sur les règles métiers 
dont la permissivité peut être configurée.
L'expérimentation 2, reposant sur cette seconde architecture, a montré que ce 
modèle optimiste est moins frustrant pour les utilisateurs. La couche \gls{P2P} est 
enrichie de nouveaux pairs dédiés à la transmission des données pour 
une dissémination plus stable et rapide. Selon le nombre d'utilisateurs, la montée 
en charge peut plus facilement être absorbée.


%\item[QR 2] Quelle architecture logicielle confère une traçabilité des données 
%conforme aux règles métier liées à la manipulation d'objet \gls{3D} ? 

Dans la \textbf{QR 2} sont évoqués les verrous souvent rencontrés dans la 
modélisation 3D et particulièrement lorsque le volet collaboratif s'invite. 
L'historique et le \og défaire / refaire\fg{} sont des fonctionnalités fondamentales en 
modélisation 3D. Pour assurer un suivi de la création, sans perdre l'intention de 
l'utilisateur de son action à sa visualisation, la modélisation sur le paradigme 
événementiel s'adapte par ses capacités à permettre une forte traçabilité des 
changements qui se produisent au sein du système. 

La modélisation logicielle passe, 
dans un premier temps, par le suivi des bonnes pratiques énoncées par le 
\gls{DDD} pour cerner les différents agrégats qui composent le domaine de la 
modélisation 3D collaborative. 
Puis, dans un second temps, deux principes sont combinés dans l'application 
cliente : 
\begin{enumerate*}[label=(\roman*)]
	\item \label{item:cqrs}la séparation de la partie écriture de la partie lecture 
	(\gls{CQRS}) ;
	\item \label{item:es} la création d'événements auto-descriptifs du changement 
	qui vient de se 
	produire (\gls{ES}).
\end{enumerate*}
%\begin{itemize}
Grâce à \ref{item:cqrs}, le système impose plusieurs niveaux de vérification des données : de 
la saisie utilisateur à la validation les règles métiers. Ainsi, le système, qui ne 
manipule que des événements immuables, est garanti d'avoir des données valides.
Côté lecture, les vues matérialisées de l'interface sont issues de projections faites 
à partir du flux d'événements traités par le système. Les vues sont spécifiques à 
l'activité, ce qui permet de présenter une interface orientée tâches plus proche des 
attentes métiers de l'utilisateur. La projection découple les flux de la base 
de données des évènements servant une ou plusieurs vues. En effet, son rôle 
d'intermédiaire est de fournir les événements d'intérêts sous une forme 
dénormalisée pour les vues (mise en cache).
Le point \ref{item:es} reprend principalement les avantages de l'\gls{ES}, i.e. 
enregistrer la 
séquence d'événements qui a mené à l'état courant de l'application. La 
fonctionnalité d'historique est implicitement présente dans ce patron. Même ce qui 
pourrait être considéré comme une \og annulation\fg{} par l'utilisateur est conservé 
sous la forme d'un événement de compensation. L'immuabilité des événements 
capte toutes les manipulations métiers de l'utilisateur. Par conséquent, la 
reconstruction des agrégats du domaine permet d'avoir un état cohérent garanti 
par l'aspect fonctionnel de ce patron. C'est particulièrement important pour 
\begin{enumerate*}[label=(\roman*)]
	\item améliorer la sensibilisation des collaborateurs à l'historique des objets 
	manipulés ;
	\item faciliter le développement d'outils de surveillance pour observer, analyser 
	et améliorer l'application ;
	\item et organiser des audits (exemple : respect d'une procédure).
\end{enumerate*}
%
%\item[QR 3] Quels sont les mécanismes assurant à l'utilisateur d'être 
%autonome tout en ayant la possibilité de collaborer ?

Une partie de la contribution décrivant le modèle événementiel révèle une partie de 
la réponse à la \textbf{QR 3}.
Le fait de déporter le patron \gls{CQRS} en totalité sur le client évite la dépendance du client au serveur pour valider le contenu créé. L'utilisateur a accès 
localement aux événements. Il peut ainsi générer de nouveaux événements sur la 
base de ceux qu'il possède déjà. Les projections fonctionnent également sur ces 
événements stockés. Le cycle des données est donc complètement géré en 
interne et libère l'utilisateur d'une forte liaison avec la base de données centrale. 
La synchronisation des données produites hors ligne est effectuée à la 
reconnexion du pair au réseau. En effectuant la différence entre les agrégats 
locaux et distants, les répliques sont capables de se synchroniser et de mettre à 
jour leur Event Store. 

L'autre partie de la réponse à la \textbf{QR 3} est apportée par la contribution 
portant sur la présentation d'une architecture de communication hybride. Le 
système renforce l'autonomie des utilisateurs en leur permettant de s'échanger 
directement des 
informations. Le serveur n'est plus une source d'engorgement pour l'accès à la 
base de données, mais un ensemble de pairs reliés à une source de vérité de 
l'application. Les pairs liés à la base de données sont répartis dans le réseau de 
manière à être plus disponibles et proches pour les pairs sur lesquels les 
utilisateurs travaillent. 
En cas d'arrêt temporaire de quelques-uns de ces pairs, l'application 
peut continuer de fonctionner en collaboration sur les pairs restants. 
La mise en place d'un environnement collaboratif nécessite également un 
mécanisme de gestion de conflits. Cette thèse présente successivement deux 
implantations de ce mécanisme, l'un pessimiste (verrou), l'autre optimiste 
(\textit{Last Write Wins}). 
Considérant que le premier peut générer des effets de bord considérables dans 
l'environnement et amener à la frustration des utilisateurs, le second, plus 
permissif et accompagné de mécanismes de sensibilisation au groupe (boîtes 
englobantes) et de sensibilisation à l'interaction (sélection fantôme), 
renie ces inconvénients et laisse plus de liberté à l'utilisateur dans ses choix de 
sélection, et plus largement, d'interaction.
%
%\item[QR 4] Comment faciliter l'implémentation d'un tel système en garantissant 
%le respect des métiers liées à la manipulation d'objets  \gls{3D}?

La \textbf{QR 4} s'intéresse à l'implémentation d'un \gls{EVC3D} sur le web. La 
question s'oriente principalement sur les choix qu'impose ce type d'environnement 
très contraint, tant au niveau du métier de la modélisation 3D collaborative 
qu'au niveau des technologies et des techniques qui s'y rapportent.
Les prototypes web sont réalisés à partir de 
technologies issues de standards (définis par \gls{W3C}, \gls{IETF} et 
\acrshort{ECMA} 
pour \gls{JS}) largement supportés comme WebGL pour le rendu de 
l'environnement 3D et parfois encore en évolution comme WebRTC DataChannel 
pour la transmission de données en \gls{P2P}. 
Dans le prototype reposant sur le paradigme événementiel, l'implantation d'une 
\gls{IU} orientée tâches rend possible un accompagnement de l'utilisateur dans 
chaque action, en lui procurant une interface adaptée à la tâche en cours de 
réalisation. Le pendant de ce découpage de l'interface par tâches est un passage 
à l'échelle, facilité par la production de vues matérialisées issues de projections 
d'événement qui peuvent être mises en cache. Une projection pouvant servir 
plusieurs vues, il est possible de décliner plusieurs visualisations des mêmes 
données. La projection liée à l'environnement 3D peut, par exemple, proposer 
plusieurs déclinaisons des mêmes données pour chaque vue qui en dépend. Cela 
peut se traduire par un rendu effectué par des bibliothèques 3D différentes 
(Three.JS, BabylonJS) ou un rendu adapté selon un profil utilisateur. Les 
projections sont également un facteur facilitant l'évolution de l'application,
simplifiant le processus d'agrégation et de filtrage des événements.

La réalisation de plusieurs prototypes fonctionnant sur différents 
paradigmes pour les expérimentations a permis de mettre en évidence les 
avantages et inconvénients de ceux-ci.
Passer d'un paradigme orienté état à un paradigme événementiel a apporté
plusieurs bénéfices liés au métier, notamment en ce qui concerne la conservation 
de l'intention de l'utilisateur au moment de la réalisation d'une action. En effet, les 
événements sont auto-descriptifs et sont également porteurs du lien de causalité 
qui reflète cette intention. Le réseau, responsable de la dissémination de ces 
informations, ne porte pas la responsabilité de la conservation de cette propriété, 
c'est à la couche applicative (aux projections) de s'en charger. 
Cela permet d'alléger les échanges 
avec des connexions \gls{P2P} ayant une configuration non fiable et non 
ordonnée. Le protocole d'échanges proposé repose sur une politique push / pull qui 
s'assure de la synchronisation des différents pairs. En cas de conflits, les 
participants sont notifiés par le gestionnaire de conflits, ce qui leur permet de 
relancer la synchronisation sur une base commune en cas de problème insolvable. 
Le ré-ordonnancement passe par le mécanisme de visualisation flexible. Les 
événements sont publiés a destination des projections une fois reçu et validés. 
Seuls les événements en séquence (pas d'événements manquants, de \og 
trou\fg{}) sont transmis aux projections, ce qui permet de stocker en avance les futurs 
événements et de déclencher une nouvelle demande des événements manquants. 
Cela permet de proposer rapidement un retour visuel à l'utilisateur d'une scène en 
cours de synchronisation par exemple même si tous les événements ne sont pas 
réceptionnés pour qu'il commence à percevoir, s'immerger voire travailler sur la 
scène.

Enfin, l'expérimentation 2 a directement bénéficié de l'implantation d'un modèle 
orienté événements pour le prototype utilisé : l'intégration du métier a permis 
l'observation minutieuse du travail réalisé par les utilisateurs au sein de 
l'environnement. C'est la base du travail d'analyse des interactions de cette 
expérimentation.

%\item[QR 5] Quelles sont les métriques (réseau, collaboration) permettant 
%d'évaluer un tel système de manière quantitative ? Qualitative ?
%
Dans la \textbf{QR 5}, la volonté de trouver des métriques adaptées à l'évaluation 
d'un \gls{EVC3D} est exprimée tant pour ce qui concerne le réseau que la 
collaboration. La 
question se pose à la fois en termes quantitatifs et qualitatifs.
Les expérimentations détaillées dans cette thèse présentent plusieurs facteurs 
d'évaluation qualitatifs. En effet, les deux expérimentations portent sur une étude 
des utilisateurs dans le cas d'application de l'assemblage de modèles 3D de 
manière collaborative pour la visualisation et la manipulation d'objets dans un 
environnement web.
La mise en place des protocoles d'expérimentation a permis de soumettre les 
utilisateurs à différentes épreuves d'assemblage : contraint par un objectif 
déterminé, création libre. 
L'évaluation a porté sur les observations de l'expérimentateur et des réponses 
données par les participants au questionnaire concernant différents aspects : le 
ressenti du participant par rapport à la latence dans l'environnement, la gestion de 
la cohérence, la qualité de la récupération des données en cas de déconnexion, 
mais également sur l'apprentissage et la facilité d'utilisation de l'application. La 
qualité de la collaboration est également soumise au questionnaire pour savoir si 
la coopération améliore l'efficacité et / ou la vitesse de réalisation de la tâche.
Les contraintes liées à ce type d'expérimentations sont nombreuses : nombre de 
participants, taille des données 3D, gestion de la cohérence\dots 
Les métriques quantitatives n'ont pu être abordée au cours de cette thèse, mais 
plusieurs perspectives sont évoquées dans la Section \todo{ref section} 
ci-après.\todo{enlever le quantitatif de la question?}


\section{Perspectives}
Les évaluations présentées sont à compléter avec des évaluations quantitatives 
permettant de comparer les architectures de communication hybrides exposées avec 
d'autres systèmes pour la modélisation 3D collaborative. 

\paragraph{Virtualisation des comportements collaboratifs}
La simulation de comportements 
collaboratifs sous forme de scénarios enregistrés (grâce à l'\gls{ES}) aurait pu 
permettre d'obtenir des résultats sur des métriques quantitatives fiables telles que la 
quantité de données transmises, la variation de la topologie du 
réseau. Alors que des auteurs sont capables de réunir jusqu'à cent personnes 
simultanément sur une expérimentation pour montrer le passage à l'échelle de leur 
système \cite{Hu2017}, il semble préférable pour des questions de temps, de 
logistiques et de coûts, d'évoquer les possibilités de virtualisation offertes par les 
plateformes réceptives au standard WebRTC pour tester les capacités du système 
\cite{}.

Pour cela il est nécessaire d'avoir un système de virtualisation du réseau pour 
pouvoir faire varier les différents critères -- nombre 
d'utilisateurs, nombre de messages, nombre d'événements traités, taille des 
données 3D (impact sur la visualisation), longueur des scénarios -- qui indiqueraient de 
manière quantitative la capacité du système à passer à l'échelle 
lors de la collaboration. Le temps de synchronisation semble être un bon indicateur 
concernant cette propriété. Ces données pourraient permettre d'appuyer les résultats
qualitatifs qui sont présentés dans cette thèse. 
Les outils liés à WebRTC sont encore au stade expérimental mais produisent déjà 
des statistiques par rapport à quelques un des critères énoncés sur les données 
transitant sur une connexion DataChannel.

\paragraph{Compatibilité et plasticité}
Le standard WebRTC, sur lequel repose la couche \gls{P2P}, est encore en 
évolution, parfois instable, hétérogène\footnote{Les standards ne sont pas toujours 
respectés\dots} ou indisponible, c'est pourquoi en proposant l'utilisation 
de protocoles alternatifs à WebRTC (WebSocket, HTTP, ORTC) l'accessibilité est 
renforcée. En fonction du type de connexion et d'appareil, l'application pourrait 
basculer d'un protocole à l'autre pour stabiliser la connexion en choisissant selon 
différents facteurs (énergétique, bande passante) indiqués par l'appareil ou 
l'utilisateur. 
Le type de modélisation peut également s'adapter en fonction de ces critères  
pour ne pas dégrader trop l'expérience utilisateur selon les cas d'utilisation en 
proposant par exemple un rendu image du modèle 3D effectué sur un serveur 
\cite{Abdallah2014} ou réalisé par une grappe de pairs \cite{Li2015}.
Ces perspectives d'adaptation du rendu 3D, ou plus généralement de l'interface 
font référence à un intérêt pour la plasticité dans les interfaces 3D 
\cite{Lacoche2014} en s'appuyant sur les métadonnées issues de l'utilisateur, du 
système, de l'appareil, de la connexion et de la sémantique du contenu produit et 
des contraintes de temps liées au cas d'utilisation.



\paragraph{Simplification et compression de maillage}
Si la plateforme web est accessible depuis n'importe quel appareil possédant un navigateur, 
la transmission (le téléversement ou le téléchargement) de modèles 3D peut s'avérer long et fastidieux.
Afin de réduire les latences sur des scènes plus large, la transmission des 
données 3D pourrait être améliorée grâce à un format de 
compression propre aux événements produits. Grâce au 
réseau \gls{P2P} il est possible de rapprocher les ressources des utilisateurs 
selon leurs intérêts. Le rendu assisté par les pairs \cite{Zhu2011} couplé avec une 
couche dédiée au routage par sujet \cite{Baldoni2007} pourrait encourager un 
meilleur équilibre dans la répartition de la charge de transmission.

%\paragraph{Modèle de cohérence}\todo{faire?}

%
% It would be now interesting to focus on enhancing 3D 
%streaming in the P2P network to reduce CPU memory bu ering overhead of 
%client 
%for sending, storing, visualizing data. Because we are using event data 
%structures, 
%it is also important to look for a method mixing
%losing information. The robustness and the consistency checking of the 
%synchronization mechanisms used in this model do not allow a lot of 
%collaborators. 
%Improving these two parameters would allow for a bigger team collaborating in 
%real-time. Even though the event processing is quite developed, improvements 
%in 
%terms of granularity would be interesting to embrace  ner details of 3D 
%collaborative modeling such as collaboration pattern detection through complex
%event processing.


