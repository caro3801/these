%!TEX root = main.tex
%\chapter{Implantation}

3DEvent est à la fois un \gls{framework} et un éditeur 3D pour la visualisation et la 
manipulation d'objets 3D. La partie \gls{framework} est basée évènement pour 
répondre à des contraintes liés à la temporalité de l'information traitée via l'éditeur 
ainsi qu'à la distribution de l'information en terme d'intégrité et de poids. L'éditeur 
va réagir et interpréter les informations (évènements) fournies par le 
\gls{framework} pour proposer visualisation et interactions adaptées.

\section{3DEvent : Plateforme web de manipulation et visualisation 
	collaborative 
	d'objets 3D}


\subsection{Introduction}

3DEvent propose un éditeur permettant de visualiser et manipuler des 
objets 3D  avec des collaborateurs en temps-réel. L'application 3DEvent est un 
éditeur 3D reposant sur les principes et les technologies du web qui permet de 
manipuler des objets 3D de manière 
collaborative en temps-réel. Les interactions possibles sont : 
\begin{description}
	
	\item[Visualiser, naviguer, utiliser les outils de transformation] L'utilisateur peut 
	com\-me dans un environnement 3D classique interagir avec la vue en utilisant 
	la souris (survol, clic) et en bougeant la caméra (déplacements). Il peut 
	utiliser les commandes clavier et souris pour effectuer des opérations de 
	translation, rotation et homothétie directement dans le \textit{viewport} ou via le 
	menu ou via la console du navigateur.
	\item[Charger des modèles 3D] L'éditeur gère la plupart des formats de fichier 
	3D \info{ref [Bou12]}(OBJ, PLY, DAE, glTF\ldots)
	\item[Changement de référentiel] La modification des coordonnées de 
	réfé\-ren\-ces (local/global)  pour les différentes transformations possibles
	\item[Grid snapping] Cette fonctionnalité permet d'aligner les modèles avec la 
	grille avec un effet de magnétisme sur les intersection de la grille.
	\item[Changement de point de vue] L'utilisateur peut à tout moment passer de 
	son point de vue à celui d'un autre utilisateur. Le choix d'implanter ce type de 
	fonctionnalité s'inscrit dans la perspective de sensibilisation de l'utilisateur au 
	travail de ses collaborateurs. Ainsi, lors de la session, le fait de prendre de 
	point de vue d'un collaborateur, se mettre à sa place, est une manière de 
	comprendre son point de vue, son fonctionnement et d'imaginer ses 
	perspectives de conception à travers le point de vue qu'il a choisit.
\end{description}


\subsection{Interface utilisateur}
Dans le but de proposer une \gls{IU} proche des fonctionnalités métier liées à la 
modélisation 3D avec une interface orientée tâche. De cette manière, le modèle 
orienté évènements présenté dans le précédent chapitre est directement orientée 
métier.
\subsubsection{Présentation de l'interface}

Lorsqu'un utilisateur se connecte à une scène, il a accès à une interface web 
(dans un navigateur) qui représente l'espace de travail collaboratif lui permettant 
d'utiliser différentes fonctionnalités. Les deux modalités d'interaction sont le clavier 
et la souris\info{est ce qu'on parle de mobile?}. Le premier niveau de cette 
interface est scindée en deux panneaux~: 
\begin{enumerate}
	\item L'espace 3D consacré à la visualisation des objets et à leur manipulation 
	dans l'environnement 3D~;
	\item La barre d'outils qui contient trois onglets~:~
	\begin{enumerate}
		\item "Scene" contient tous les détails de la scène et des maillages qu'elle 
		inclue~; 
		\item "Collaboration" fournit les informations liées à la collaboration~;
		\item "History" liste tous les évènements qui ont eut lieu dans la scène et 
		leurs  détails. 
	\end{enumerate}
\end{enumerate}

L'onglet "Scène" possède un bloc contenant les détails d'un maillage en cours de 
sélection. Cela permet d'avoir la description des propriétés de l'objet sélectionné et 
une manipulation de ses paramètres (position, rotation et mise à l'échelle) plus 
précise que via l'espace 3D avec le cliqué/déplacé .

L'onglet "Collaboration" présente la liste des collaborateurs qui participent à la 
scène. Chacun d'eux est décrit par son nom, son état  (connecté ou déconnecté) 
et son rôle (administrateur, éditeur, lecteur ou autre\footnote{Un rôle peut être 
défini par le biais du \gls{framework} 3DEvent}). En cliquant sur un élément de la 
liste, l'utilisateur accède au dernier point de vue dans l'espace 3D connu du 
collaborateur représenté.

L'onglet "History" liste tous les évènements passés dans la scène en fournissant 
l'accès à leur détail. Pour chaque évènement, le système est capable de montrer 
dans l'espace 3D la différence entre l'état  après l'évènement cliqué $state_x$ et 
l'état courant $state_n$. L'utilisateur peut à partir de cette visualisation choisir de 
<<~revenir en arrière~>> sans perdre les données entre $state_n$ et $state_x$ 
car dans notre système cela s'effectue par compensation (cf Event-Sourcing 
Section X)\improve{annulation d'un évènement ou juste ES}.

Dans chaque onglet on trouve donc différent blocs \gls{HTML}, avec des 
comportements spécifiques à un agrégat et injectés dynamiquement. Ces blocs 
correspondent aux Views de notre modèle.

\subsubsection{Flexibilité de la visualisation}
Dans l'approche \gls{CQRS}, une projection est une dérivation de l'état courant à 
partir du flux d'évènements. Pour Abdullin, <<la projection est le processus de 
conversion (ou d'agrégation) d'un flux d'évènement en une représentation 
structurelle. Cette dernière (qui est mise à au moment où le flux est parcourue) 
peut être avoir différentes appellations : modèle de lecture persistent, vue ou 
état.>>\info{footnote abullin prj}
La partie lecture du modèle (l'affichage sur interface utilisateur) bénéficie des 
projections en lui permettant de réduire l'afflux des évènements, ne laissant filtrer 
que ceux qui sont pertinents pour la vue. La projection fournit une vue adaptée 
(filtrée, enrichie\ldots) du flux d'évènements au client. Elle peut également être 
utilisée pour mettre en avant des aspects experts (notifications, déclenchement 
d'action) ou des raisons de confidentialité.
Une projection peut être créée de manière synchrone (à la volée) au fur et à 
mesure de la publication des évènements ou de manière asynchrone et donc 
découplée du flux des évènements. 


Du fait de la nature d'un réseau \gls{P2P}, les pairs ne reçoivent pas forcément les 
paquets réseau de manière ordonnée.
Par conséquent, les messages pouvant arriver dans n'importe quel ordre, 
qu'arriverait-il si un évènement A ($eA$) nécessitant un autre évènement B ($eB$) 
pour être appliqué arrivait avant? Dans cette situation, le système va générer une 
erreur en essayant d'appliquer $eA$ sur un état inadéquat car il n'a pas 
d'information sur la hiérarchie d'application des évènements ($eB$ puis $eA$).

Pour pallier ce problème, l'introduction du système de projection permet d'avoir un 
mécanisme (comme un automate fini) qui défini les transitions nécessaires pour 
passer d'un état à l'autre et qui réalisent les actions déterminées en fonction des 
évènements qui arrivent. Par exemple, si on essaie d'ajouter un objet dans une 
scène  ($eA$) sans avoir créer la scène ($eB$) la projection met en attente $eA$ 
jusqu'à recevoir $eB$. Dans le cas où $eB$ n'arrive jamais, la projection ne pourra 
jamais utiliser $eA$.

\section{Intergiciel P2P pour l'échange de données 3D}

L'assomption est faite que tous les clients utilisent des navigateurs qui 
implémentent et supportent le protocole WebSocket et l'\gls{API} 
RTCDataChannel. 

La topologie de l'architecture de communication repose sur la mise en relation 
automatique des clients par le biais du serveur pour établir une connexion 
\gls{WebRTC}. Pour ce faire, chaque client envoie son identifiant (ID) lors de sa 
première requête au serveur qui le stocke. Selon le paramétrage de la connectivité 
directe minimum établie préalablement, le serveur recherche aléatoirement l'ID 
d'autres clients qui satisfont la règle de connectivité. Cette règle de connectivité 
minimum permet d'ajuster la densité du maillage (connectivité élevée: maillage 
partiel dense voire complet ; connectivité faible: maillage partiel éparse) et 
d'obtenir une topologie maillée adaptée aux besoins de l'application en termes de 
synchronisation (temps-réel ou pas) ou aux capacités des appareils. Il faut noter 
cependant que plus la connectivité est faible, plus l'information a besoin de « 
rebondir » pour atteindre tous les pairs et par conséquent le temps de 
transmission est augmenté (exemple d'une distribution en ligne). 

\subsubsection{De navigateur à serveur}
La connexion entre un pair (client) et le serveur est établie sur la base du protocole 
\gls{WebSocket}. Cette connexion bi-directionnelle est initialisée lors de la 
première requête du client pour récupérer le contenu de l'application. Cette 
connexion sert à la fois pour la phase de \textit{signaling} lors de l'établissement 
d'une connexion WebRTC mais également pour que le client puisse envoyer des 
mises à jours originales à la base de données via le serveur.

\subsubsection{De navigateur à navigateur}
Lors de la connexion d'un nouveau client à la scène, le serveur effectue la phase 
de signaling permettant de le mettre en relation avec un autre client. Le 
mécanisme est répété tant que la règle de connectivité peut s'appliquer. Le client 
reçoit une notification de l'établissement de la connexion avec un autre client ce 
qui lui permet de démarrer l'échange de données.

L'API RTCDataChannel permet à chaque pair d'échanger des données arbitraires 
avec d'autres à partir du navigateur avec des propriétés de livraison 
personnalisables (fiable ou non fiable, ordonné ou non ordonné) selon le transport 
sous-jacent\info{GrI13}. Dans 3DEvent, le choix d'avoir un transport fiable 
\unsure{fiable ou pas fiable et pk...}et non ordonné a été fait pour respectivement 
garantir l'arrivée d'une donnée émise par l'utilisateur au sein de l'application et 
permettre des échanges asynchrones.
\improve{add \S sur "en cas de défaillance? }
En cas d'arrêt soudain du serveur, si une connexion a été établie préalablement 
entre les clients et est toujours en cours, elle n'est pas impactée par la défaillance 
du serveur.
\subsection{Données d'échange}
En sachant que le modèle est conçu pour des applications web, 3DEvent a besoin 
d'un format de données permettant de faire communiquer des acteurs hétérogènes 
du système. Le format \gls{JSON}, dérivé de la notation des objets du langage 
JavaScript, il est lisible et interopérable. 
%Son pendant binaire est le format \gls{BSON}.
Le format de fichier \gls{glTF} se base sur la représentation \gls{JSON} afin de 
décrire une scène 3D.
Ce type de données est assez abstrait et suffisamment générique pour 
représenter n'importe quelle donnée. Par exemple le format de fichier 
\gls{glTF}\info{ref gltf section}se base sur la représentation \gls{JSON} afin de 
décrire une scène 3D.
Le format \gls{JSON} est aussi utilisé pour la sérialisation et la désérialisation des 
objets transmis par RTCDataChannel qui prend n'importe quel format de données.

L'\acrshort{API} RTCDataChannel supporte beaucoup de types de données 
différents (chaînes de caractères, types binaires : Blob, ArrayBuffer\dots). Dans 
un environnement multi-utilisateur avec des données hétérogènes (3D, images, 
informations) tel que 3DEvent cela facilite l'interopérabilité.


\section{Synchronisation des données}
\subsection{Persistance à court terme}
Le navigateur (client) offre un espace de stockage avec l'interface \textit{Storage} 
de l'API Web Storage qui donne accès au \texttt{session storage} ou au  
\texttt{local storage}. Grâce à un système clé/valeur, il est possible d'avoir une 
persistance des données à travers les sessions du navigateur. Le contenu stocké 
correspond aux données générées par un utilisateur et par ses collaborateurs. Les 
répliques stockées sur chaque navigateur permettent à un utilisateur une plus 
grande 
autonomie en cas de déconnexion. C'est également un moyen de distribuer les 
mises à jour générées par les clients entre les clients grâce au protocole de 
\gls{streaming3D} (cf. \ref{streamingprotocol}) sans passer par le serveur.

Ce stockage fonctionne sur un système de clé/valeur qui rend facile l'accès aux 
évènements enregistrés sur le client. La configuration du client est également 
stockée localement. 

\subsection{Protocole de streaming pour la synchronisation}
\label{streamingprotocol}

Il existe plusieurs méthodes de transmission de contenu au sein d'un réseau P2P 
que l'on peut catégoriser selon deux modes : le téléchargement (\textit{download}) 
et le flux continu (\textit{streaming}). Le téléchargement requière que le contenu 
soit entièrement téléchargé pour pouvoir être lu. Tandis que le flux continu peut 
être lu au fur et à mesure de sa récupération. Ce mode est principalement utilisé 
pour la lecture de vidéo en ligne. En comparaison le mode téléchargement est 
moins restrictif et relativement simple à mettre en place. Tout comme les 
systèmes utilisant une architecture client-serveur, la transmission de contenu en 
P2P peut également être catégorisée selon ces deux modes. Une catégorisation 
plus précise du flux continu peut être donnée selon quand le contenu est généré : 
en direct (live) et à la demande (\textit{on-demand}).


Le mécanisme de routage que nous avons utilisé dans \cite{Desprat2015a} est 
proche du routage de GNutella. 


%
%\begin{table}[]
%	\centering
%	\caption{Type de messages}
%	\label{table:messagetype}
%	\begin{tabular}{ll}
%		Message                & Description \\ \hline
%		STREAM\_SYNC\_ASK      &  Demande de synchronisation d'un 
%		\textit{stream}           \\
%		CHUNK                  &     Réception d'une donnée \textit{chunk})        
%		\\
%		READY\_ASK             &      Prêt pour la démarrer la demande de 
%		données de 
%		synchronisation        \\
%		READY                  &       Prêt pour démarrer la réception de 
%		données de 
%		synchronisation      \\
%		ALL\_EVENTS\_SYNC\_ASK &     Demande de toutes les données 
%		typées 
%		évènement           \\
%		EVENTS\_SYNC           &        Réception de données (en cours de 
%		synchronisation)       \\
%		META\_DATA\_ASK        &     Demande de métadonnées       \\
%		META\_DATA             &      Réception de métadonnées       \\
%		SYNC                   &      Réception de données (en cours de 
%		synchronisation)         \\
%		EVENT                  &     Réception d'une donnée typée 
%		évènement        \\
%		END\_SYNC              & Fin de la synchronisation
%	\end{tabular}
%\end{table}
%
%
%\begin{table}[]
%	\centering
%	\caption{Statut du n\oe ud}
%	\label{table:nodestatus}
%	\begin{tabular}{ll}
%		Message             & Description \\ \hline
%		ERROR               &             \\
%		READY               &             \\
%		META\_DATA\_ASK     &             \\
%		META\_DATA\_RECEIVE &             \\
%		CLOSE               &             \\
%		RECEIVE\_SYNC       &             \\
%		CONNECTED           &             \\
%		INIT                &             \\
%		OK                  &             \\
%		SEND\_SYNC          &             \\
%		END\_SYNC           &            
%	\end{tabular}
%\end{table}

\subsubsection{Base de données NoSQL}\label{p:nosql} L'évolution de 
l'utilisation du web en tant que plateforme applicative a encouragé le changement 
dans le stockage des données pour de nouveaux besoin supportant de larges 
volumes de données (comme les données 3D). Une base de données \gls{NoSQL} 
fournie un schéma libre et dynamique ainsi qu'une API de requête riche pour la 
manipulation de données. De plus, la possibilité d'enrichir un document à la volée 
facilite l'évolution des objets (3D) et la maintenance.\info{parler de la 
maintenance?} de l'application.\improve{eventual consistency, scalabilty}
3DEvent intègre un système de persistance sur le long terme caractérisé par une 
base de données \gls{NoSQL}.

La base de données (\gls{NoSQL}) conserve tous les évènements qui 
sont produits dans une scène.\improve{voir ce qu'il faut ajouter éventuellement le 
	schéma aussi.} 
La mise en place d'une base de données centralisée apporte de la robustesse au 
système en lui fournissant un référent sans toutefois le surcharger ainsi qu'une 
expérience utilisateur transparente limitant les interruptions de service.

\subsection{Introduction}
\subsection{Interface utilisateur}
\subsubsection{Présentation de l'interface}
\subsubsection{Flexibilité de la visualisation}
