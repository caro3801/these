%!TEX root = main.tex

\chapter{Implantation}
\chaptertable

\section{Introduction} 


\section{3DEvent : Plateforme web de manipulation collaborative d'objets 3D}
Cette section présente l'implantation de 3DEvent, la plateforme web de 
manipulation et visualisation collaborative d'objets 3D réalisée à partir du modèle 
présenté dans le chapitre précédent. La première partie s'intéresse à l'intégration 
du framework pour proposer une application d'assemblage d'objets 3D. La seconde 
partie expose les choix techniques pour l'interface utilisateur et la mise en avant 
du système de visualisation flexible.

%\chapter{Implantation}
%\section{3DEvent : Plateforme web de manipulation et visualisation 
%	collaborative 
%	d'objets 3D}
\subsection{Éditeur 3DEvent}
L'intérêt de proposer une application web se 
retrouve principalement dans 
l'accessibilité qu'elle propose. En effet, n'importe 
quel terminal muni d'un 
navigateur web peut y accéder, ce qui la rend distribuée et multi-plateforme. 
Les fonctionnalités graphiques proposées par WebGL sont un peu réduites par 
rapport à celles d'OpenGL dont l'\gls{API} évolue plus vite et propose plus de 
flexibilité et d'optimisations. Cependant, les performances graphique restent très 
correctes car le navigateur est quand même capable d'utiliser les processeurs 
graphiques du terminal (GPU) pour les calculs et rendus \gls{3D}.

Pour faire le lien entre le modèle et l'expérimentation, l'implantation du modèle a 
pris la forme d'un éditeur pour la modélisation \gls{3D} haut niveau permettant de 
visualiser et manipuler des objets \gls{3D} de manière collaborative dans un 
environnement web (aussi 
appelée \og application 3DEvent\fg{}). Au sein de la plateforme, les interactions 
possibles sont : 
\begin{description}
	
	\item[Visualiser, naviguer, utiliser les outils de transformation] L'utilisateur peut, 
	com\-me dans un environnement \gls{3D} classique, interagir avec la vue en 
	utilisant 
	la souris (survol, clic) et en bougeant la caméra (déplacements). Il peut 
	utiliser les commandes clavier et souris pour effectuer des opérations de 
	translation, de rotation et d'homothétie de trois manières différentes: directement dans le \textit{viewport}, via le 
	menu ou via la console du navigateur.
	\item[Charger des modèles \gls{3D}] L'éditeur gère la plupart des formats de 
	fichier 
	3D \info{ref [Bou12]}(OBJ, PLY, DAE, glTF\ldots)
	\item[Changement de référentiel] La modification des coordonnées de 
	réfé\-ren\-ces (local/global)  pour les différentes transformations possibles
	\item[Grid snapping] Cette fonctionnalité permet d'aligner les modèles avec la 
	grille avec un effet de magnétisme sur les intersection de la grille.
	\item[Changement de point de vue] L'utilisateur peut à tout moment passer de 
	son point de vue à celui d'un autre utilisateur. Le choix d'implanter ce type de 
	fonctionnalité s'inscrit dans la perspective de sensibilisation de l'utilisateur au 
	travail de ses collaborateurs. Ainsi, lors de la session, le fait de prendre le 
	point de vue d'un collaborateur est une manière de 
	comprendre son fonctionnement et d'imaginer ses 
	perspectives de conception à travers son angle de caméra qu'il a choisi.
\end{description}



\subsection{Interface utilisateur orientée tâche}

Dans le but de proposer une \gls{IU} proche des fonctionnalités métiers liées à la 
modélisation \gls{3D}, l'éditeur possède une interface orientée \og tâche\fg{}, en 
comparaison avec des \gls{IU} \gls{CRUD}. En effet, les \gls{IU} \gls{CRUD} 
réduisent la sémantique métier du domaine d'application à la création, la lecture, la 
mise à jour et la suppression, omettant toutes les subtilités que peuvent dégager 
ces actions en perdant l'intention de l'utilisateur dès le niveau de l'interface. Une 
interface orientée tâche a tendance à s'attarder sur toutes les nuances que le 
domaine possède en caractérisant chaque action sans subir d'effet de 
simplification. Cette proximité avec le métier permet de calquer directement 
l'interface du modèle événementiel sur l'\gls{IU} et de guider l'utilisateur dans ses 
activités. L'utilisabilité, qualité de l'expérience utilisateur fournit par un système 
pour réaliser une tâche, est alors maximisée en terme d'efficacité, d'efficience et 
de satisfaction. 
Ce type d'\gls{IHM} s'organise autour de cas d'utilisation. Cela permet, 
d'une part, de présenter clairement les 
actions (\og ajouter une géométrie à la 
bibliothèque à partir d'un fichier\fg{} plutôt que \og téléverser un fichier\fg{}) : 
l'intention est clairement définie. D'autre part, lorsque l'utilisateur s'apprête à faire 
une action, seules les informations utiles sont affichées. Enfin, l'application fournit 
simplement l'information dans le contexte où elle doit être présentée, évitant à 
l'utilisateur d'aller la chercher ailleurs.
L'\gls{IU} devient alors une couche de l'application qui nécessite d'agréger, croiser 
et filtrer des données. La dénormalisation proposée par \gls{CQRS} remédie à ce 
besoin dans le cadre de la consultation de données. 


\subsubsection{Présentation de l'interface}


%\begin{figure}[h!]
%	\centering
%	\begingroup
%	
%	\subfloat[Rotation (vue \gls{3D}) et outils de manipulation d'objet 
%	\gls{3D} 
%	(panneau 
%	
%latéral)]{\includegraphics[width=0.75\textwidth]{eps/2rotatedetail.eps}\label{fig:ui2}}\hfill
%	
%	\subfloat[Translation (vue \gls{3D}) et visualisation de l'historique 
%	(panneau 
%	
%latéral)]{\includegraphics[width=0.75\textwidth]{eps/1translatehisto.eps}\label{fig:ui1}}\hfill
%	
%	\subfloat[Mise à l'échelle (vue \gls{3D}) et liste des collaborateurs 
%	(panneau 
%	
%latéral)]{\includegraphics[width=0.75\textwidth]{eps/3scalecollab.eps}\label{fig:ui3}}\hfill
%	
%	\endgroup
%	\caption{Interface utilisateur pendant une session collaborative (trois 
%personnes)}
%	\label{fig:screenshots}
%\end{figure}
Lorsqu'un utilisateur se connecte à une scène, il a accès à une interface web 
(dans un navigateur) qui représente l'espace de travail collaboratif lui permettant 
d'utiliser différentes fonctionnalités. Les deux modalités d'interaction sont le clavier 
et la souris\info{est ce qu'on parle de mobile?}. Le premier niveau de cette 
interface est scindée en deux panneaux~: 
\begin{enumerate}
	\item L'espace \gls{3D} consacré à la visualisation des objets et à leur 
	manipulation 
	dans l'environnement \gls{3D}~;
	\item La barre d'outils qui contient trois onglets~:~
	\begin{enumerate}
		\item "Scene" contient tous les détails de la scène et des maillages qu'elle 
		inclue~; 
		\item "Collaboration" fournit les informations liées à la collaboration~;
		\item "History" liste tous les événements qui ont eut lieu dans la scène et 
		leurs  détails. 
	\end{enumerate}
\end{enumerate}

\begin{figure}[ht]
	\centering
	\begingroup
	
	\subfloat[Onglet \og outils de manipulation sur la 
	scène\fg{}]{\includegraphics[width=0.38\textwidth]{eps/scenecontrol.eps}\label{fig:uicontrol}}
	\hfill
	\subfloat[Onglet \og collaboration\fg{}]
	{\includegraphics[width=0.27\textwidth]{eps/collaboration.eps}\label{fig:uicollab}} 
	\hfill
	\subfloat[Onglet \og 
	historique]{\includegraphics[width=0.32\textwidth]{eps/history.eps}\label{fig:uihisotry}}
	
	\endgroup
	\caption{Onglets du panneau latéral de l'interface}
	\label{fig:uipanneau}
\end{figure}
La Figure \ref{fig:uipanneau} montre quelques captures d'écran durant une 
session collaborative sur le modèle Rotor.

L'onglet "Scene" (Figure \ref{fig:uicontrol}) possède un bloc contenant les détails 
d'un 
maillage en cours de 
sélection. Cela permet d'avoir la description des propriétés de l'objet sélectionné et 
une manipulation de ses paramètres (position, rotation et mise à l'échelle) plus 
précise que via l'espace \gls{3D} avec le cliqué / déplacé. "Scene" intègre 
également un espace réservé aux géométries disponibles dans la scène appelé 
Bibliothèque (de géométries).

L'onglet "Collaboration" (Figure \ref{fig:uicollab}) présente la liste des 
collaborateurs qui 
participent à la 
scène. Chacun d'eux est décrit par son nom, son état  (connecté ou déconnecté) 
et son rôle (administrateur, éditeur, lecteur ou autre\footnote{Un rôle peut être 
défini par le biais du \gls{framework} 3DEvent}). En cliquant sur un élément de la 
liste, l'utilisateur accède au dernier point de vue dans l'espace \gls{3D} connu du 
collaborateur représenté.

L'onglet "History" (Figure \ref{fig:uihisotry}) liste tous les événements passés dans 
la 
scène en fournissant 
l'accès à leur détail. Pour chaque événement, le système est capable de montrer 
dans l'espace \gls{3D} la différence entre l'état  après l'événement cliqué $state_x$ 
et l'état courant $state_n$. L'utilisateur peut à partir de cette visualisation choisir 
de \og revenir en arrière\fg{} sans perdre les données entre $state_n$ et $state_x$ 
car dans notre système cela s'effectue par compensation (cf Event-Sourcing 
Section X)\improve{annulation d'un événement ou juste ES}.

Dans chaque onglet se trouvent différents blocs \gls{HTML}, avec des 
comportements spécifiques à un agrégat et injectés dynamiquement. Ces blocs 
correspondent aux Views de notre modèle.

Les boîtes englobantes représentent la sélection des différents collaborateurs 
pendant la session.

Parmi les Views disponibles dans le système, une grande partie est dédiée à 
l'\gls{IU} de l'application web pour le cas d'utilisation de la modélisation 3D. 
D'autres Views sont disponible pour un autre cas d'utilisation destiné à 
l'observation des comportements des utilisateurs qui est primordiale dans le cadre 
des expérimentations.



\paragraph{Exemple d'interaction}
La Figure \ref{fig:cqrs-example} décrit la façon dont le système traite l'exécution 
d'une commande de translation déclenchée par l'utilisateur et comment cette 
information est diffusée à ces collaborateurs\footnote{Pour que l'exemple 
fonctionne, la scène, la géométrie du cube et le maillage \textit{cube1} doivent 
avoir été créés en amont.}.
Dans l'étape (a), la commande déclenchée par l'utilisateur s'adresse à l'agrégat 
$cube1$ et contient les paramètres de la translation (vecteur x,y,z). L'agrégat qui 
modélise le domaine d'un maillage, génère l'événement de translation $e1$ (étape 
(b)) si tout est valide d'un point de vue métier. L'événement $e1$ est ensuite 
passé à l'Event Store. 
Le composant responsable de la détection de conflit permet au développeur 
d'implémenter ses propres règles de résolution de conflit. Le composant déclenche 
une exception lorsque le numéro de version reçu et le numéro de version courant 
de l'agrégat sont identique (Figure \ref{fig:cqrs-example} étape (c)). Selon les 
règles métiers définies et les exceptions liés à la cohérence, l'événement peut être 
rejeté. Ce traitement peut être à l'origine de la génération de nouveaux 
événements.


\begin{figure}[]
	\centering
	\includegraphics[width=\columnwidth]{eps/example10.eps}
	\caption[Flux de la collaboration dans le framework 3DEvent entre 3 
	utilisateurs]{Exemple d'édition collaborative où User A est connecté à User  B, 
		lui 
		même connecté à User C. Le cycle montre les différentes étapes du 
		déclenchement de la commande au rendu visuel en passant par la 
		génération 
		de l'événement, la 
		synchronisation du journal d'événements et l'impact sur le rendu des autres 
		utilisateurs pour une translation sur un cube.}\label{fig:cqrs-example}
\end{figure}
\subsection{Flexibilité de la visualisation}
\label{sec:flexviz}
Dans l'approche \gls{CQRS}, une projection est définie comme une dérivation de l'état courant à 
partir du flux d'événements. Pour Abdullin, \og la projection est le processus de 
conversion (ou d'agrégation) d'un flux d'événement en une représentation 
structurelle. Cette dernière (qui est mise à au moment où le flux est parcourue) 
peut être avoir différentes appellations : modèle de lecture persistent, vue ou 
état\fg{}\cite{Abdullin2011}.
La partie lecture du modèle (l'affichage sur interface utilisateur) bénéficie des 
projections en lui permettant de réduire l'afflux des événements, ne laissant filtrer 
que ceux qui sont pertinents pour la vue. La projection fournit une vue adaptée 
(filtrée, enrichie\ldots) du flux d'événements au client. Elle peut également être 
utilisée pour mettre en avant des aspects experts (notifications, déclenchement 
d'action) ou des raisons de confidentialité.
Une projection peut être créée de manière synchrone (à la volée) au fur et à 
mesure de la publication des événements ou de manière asynchrone et donc 
découplée du flux des événements. 


Du fait de la nature d'un réseau \gls{P2P}, les pairs ne reçoivent pas forcément les 
paquets réseau de manière ordonnée.
Par conséquent, les messages peuvent arriver dans n'importe quel ordre.
Qu'arriverait-il alors si un événement A ($eA$) nécessitant un autre événement B ($eB$) arrivait avant celui-ci? Dans cette situation, le système génére une 
erreur en essayant d'appliquer $eA$ sur un état inadéquat car il n'a pas 
d'information sur la hiérarchie d'application des événements ($eB$ puis $eA$).

Pour pallier ce problème, l'introduction du système de projection permet d'avoir un 
mécanisme (comme un automate fini) qui défini les transitions nécessaires pour 
passer d'un état à l'autre. Les transitions réalisent les actions déterminées en fonction des 
événements qui arrivent. Par exemple, si on essaie d'ajouter un objet dans une 
scène  ($eA$) sans avoir créer la scène ($eB$) la projection met en attente $eA$ 
jusqu'à recevoir $eB$. Dans le cas où $eB$ n'arrive jamais, la projection ne pourra 
jamais utiliser $eA$.

\begin{figure}
	\centering
	\inputTikZ{0.9}{eps/tikz/streams/aggregate.tex}
	\caption{Exemple d'agrégats}{Structure d'un agrégat et ses versions. Chaque 
	version est un état de l'agrégat qui correspond à l'empilement des instances 
	d'événements (ei) qu'il contient. Les types des événements sont relatifs au type 
	d'agrégat dans lequel il est contenu.}
	\label{fig:aggregate}
\end{figure}

\paragraph{Projections}
chaque partie de l’interface est liée à une proj de la bdd
les actions user et les actions des autres users passent par le meme cycle
pas de diff de prise en compte des evnts de l’interaction user ou de la couche 
reseau
en CS : les actions users -> actions -> recup info
action push du servuer qui peuvent etre gerees de maniere diff
\subsection{Bilan}

 L'application 3DEvent repose sur les principes et les 
technologies du web pour permettre de visualiser et manipuler des objets \gls{3D} 
de 
manière 
collaborative en temps-réel.

\section{Intergiciel P2P pour l'échange de données 3D}

L'architecture de communication décrite dans le chapitre précédent nécessite 
l'implantation d'un intergiciel P2P compatible avec les besoins liés à la 3D, le 
web et la collaboration. 

%!TEX root = main.tex
%\chapter{Implantation}
%editeur...
%\section{Intergiciel P2P pour l'échange de données 3D}
%\subsubsection{La virtualisation des clients}
%\label{virtualisation}
%\todo{mettre ailleurs?}
%Une des problématiques soulevée par la collaboration \gls{P2P} est de permettre 
%la reproductibilité des expérimentations dans un environnement contrôlé et 
%réaliste.
%Réussir à simuler un réseau virtuel de clients en \gls{P2P} en utilisant le 
%protocole 
%\gls{WebRTC} est un défi encore compliqué. Il existe des outils pour simuler des 
%réseaux \gls{P2P} tels que PeerSim \cite{Montresor2009} ou \textit{ns-3} 
%\cite{Riley2010}. Ces outils sont 
%plus orientés sur la façon de distribuer les informations lors de la simulation et 
%leur 
%dissémination au sein du réseau plutôt que de reproduire les protocoles et 
%l'infrastucture de manière réaliste avec des données issues de l'\gls{IU}. Dans 
%de 
%récents travaux utilisant \gls{WebRTC}, les expérimentations ne 
%sont pas encore facilement reproductibles du fait qu'il n'existe pas d'outil facile à 
%prendre en main pour effectuer ce genre de simulation à base d'entrées 
%utilisateur 
%fiables vis à vis des impératifs métier. 
%
%La virtualisation implique également de pouvoir simuler des comportements sur 
%la base d'interactions issues de \gls{IU} \cite{Hu2017} comme on peut le trouver 
%dans la simulation d'\gls{IU} web.
%Dans ce contexte, ce type de tests permet de vérifier la compatibilité et la 
%réactivité des différentes plateformes, versions de navigateurs et types 
%d'appareils 
%en fonction d'entrées utilisateur. C'est également utile pour faire des tests de 
%performance ou de montée en charge 
%concernant l'interface. 
%Le service testRTC\footnote{\url{testrtc.com}. Consulté le 
%	07/07/2017} est un service payant qui propose un outil de test et de monitoring 
%pour un grand banc de machines de sessions audio et vidéo WebRTC .
%
%
%%TODO mettre ailleurs?
%L'intérêt d'utiliser un modèle de réseau \gls{P2P} virtuel comporte plusieurs 
%avantages. En reprenant les points proposés par \cite{Haque2016}, on peut citer 
%: 
%\begin{itemize}
%	\item Pas d'installation nécessaire. Plusieurs outils et logiciels existent pour 
%	simuler des réseaux \gls{P2P} \cite{Montresor2009} ou nécessite encore 
%	l'installation de clients lourds (clients BitTorrent) par les utilisateurs pour 
%réaliser 
%	les mesures. Cela implique le fait de comprendre les principes de base 
%	concernant la configuration réseau (routeurs, pare-feu) et le protocole utilisé 
%	(BitTorrent). Très peu de travaux concernant \gls{WebRTC} ont réussi à 
%	virtualiser les clients participants aux expérimentations.
%	\item Opérabilité et interopérabilité dans un environnement contrôlé.  
%	L'installation d'un client sur une machine requiert certaines autorisations liés à 
%	la politique de l'organisation, la licence logicielle et le support logiciel. Ce type 
%	d'environnement est assez typique dans l'industrie, c'est pourquoi il est 
%	intéressant de proposer un modèle qui puisse s'exécuter sans difficulté grâce 
%à 
%	l'utilisation de clients web. Les navigateurs qui servent de clients web 
%	s'accordent généralement avec les standards proposés par le \gls{W3C}, ce 
%qui 
%	facilite également l'interopérabilité du logiciel souvent déployé dans un parc 
%	hétérogène de machines.
%	\item Indépendance de la situation géographique. Tout comme les 
%	infrastructures \textit{cloud} (souvent un service tiers) qui sont distantes par 
%	l'intermédiaire d'un réseau , généralement internet les utilisateurs peuvent se 
%	connecter sur un réseau virtuel  \gls{P2P} à partir de n'importe quel lieu. 
%	\item Simplification de la maintenance. Les applications, standards et 
%	protocoles autour du \gls{P2P} sont en constante évolution. L'implémentation 
%de 
%	la méthode de distribution des données nécessite par conséquent de 
%fréquente 
%	mises à jour pour être la plus efficace possible. Dans le cas d'une 
%	implémentation d'un client virtuel, la mise à jour qui est distribuée par le 
%serveur 
%	sera automatique et la même sur tous les clients ce qui facilite la 
%maintenance 
%	car c'est le distributeur qui est responsable de la mise à jour et non le client.
%	\item Mobilité et accès au réseau. La mise en place d'un réseau P2P permet 
%de 
%	découpler l'accès à l'information et aux ressources du système. De ce fait, 
%	les clients peuvent travailler directement entre eux sans supervision après 
%mise 
%	en relation et partager leurs ressources avec les autres clients qui en ont 
%	besoin. Le réseau peut évoluer sans que cela ait un fort impact sur la 
%	collaboration. Les clients peuvent être plus mobiles du fait de la grande 
%	disponibilité offerte par cette architecture à moindre coût.
%	\item \gls{NATT} et pare feu. Les applications traditionnelles de P2P comme 
%	BitTorrent ne permettent pas à deux pairs de communiquer directement 
%	lorsqu'ils sont derrière un \gls{NAT}. Grâce à l'utilisation du protocole \gls{ICE} 
%	les appareils peuvent atteindre plus de pairs, augmentant la vitesse d'échange.
%\end{itemize}
%
%Cette liste est un point de départ pour créer un service de virtualisation de 
%clients 
%pour le partage de données (3D) avec WebRTC. 
%La mise à disposition volontaire de ressources (calcul, mémoire) en partage sur 
%le 
%réseau permet d'une part la coopération entre personnes afin de résoudre des 
%problèmes nécessitant un haut degré de computation et d'autre part l'utilisation 
%de 
%ressources qui ne seraient pas ou sous utilisées.
%
%
%
%
%En 2001, le standard \gls{DDS} est un standard machine-à-machine massif, en 
%temps-réel, hautement performant avec un système d'échange de données 
%interopérables. \gls{DDS} s'adresse principalement à des problématiques 
%d'échanges financiers, de contrôle aérien, et de réseau électrique intelligent 
%(\textit{smart grid}). Il a fortement été promu pour mettre en place des 
%applications 
%liées à l'internet des objets. Les spécifications proposent deux niveaux 
%d'interfaces. Le premier se concentre sur la mise à disposition d'un système 
%\gls{PubSub} bas niveau centré données pour permettre la livraison efficace de 
%la 
%bonne information au bon destinataire. Le second, niveau optionnel, 
%est une couche de reconstruction locale de la donnée permettant une integration 
%plus simple de \gls{DDS} au sein d'une application. \gls{DDS} est donc un 
%intergiciel réseau basé sur une architecture \gls{PubSub} qui gère la livraison 
%de messages sans nécessiter l'intervention d'un utilisateur. Il détermine qui doit 
%recevoir les messages, où sont situés les destinataires et ce qu'il se passe si 
%un 
%message n'est pas délivré. En cela, \gls{DDS} permet une gestion plus fine de la 
%qualité de service notamment concernant les paramètres de découverte des 
%pairs.



L'assomption est faite que tous les clients utilisent des navigateurs qui 
implémentent et supportent le protocole WebSocket et l'\gls{API} 
RTCDataChannel. 

La topologie de l'architecture de communication repose sur la mise en relation 
automatique des clients par le biais du serveur pour établir une connexion 
\gls{WebRTC}. Pour ce faire, chaque client envoie son identifiant (ID) lors de sa 
première requête au serveur qui le stocke. Selon le paramétrage de la connectivité 
directe minimum établie préalablement, le serveur recherche aléatoirement l'ID 
d'autres clients qui satisfont la règle de connectivité. Cette règle de connectivité 
minimum permet d'ajuster la densité du maillage (connectivité élevée: maillage 
partiel dense, voire complet ; connectivité faible: maillage partiel éparse) et 
d'obtenir une topologie maillée adaptée aux besoins de l'application en termes de 
synchronisation (temps-réel ou non) ou aux capacités des appareils. Il faut noter 
cependant que plus la connectivité est faible, plus l'information a besoin de « 
rebondir » pour atteindre tous les pairs et par conséquent le temps de 
transmission est augmenté (exemple d'une distribution en ligne). 

\subsubsection{De navigateur à serveur}
La connexion entre un pair (client) et le serveur est établie sur la base du protocole 
\gls{WebSocket}. Cette connexion bi-directionnelle est initialisée lors de la 
première requête du client pour récupérer le contenu de l'application. Cette 
connexion sert à la fois pour la phase de \textit{signaling} lors de l'établissement 
d'une connexion WebRTC mais également pour que le client puisse envoyer des 
mises à jours originales à la base de données via le serveur.

\subsubsection{De navigateur à navigateur}
Lors de la connexion d'un nouveau client à la scène, le serveur effectue la phase 
de signalement permettant de le mettre en relation avec un autre client. Le 
mécanisme est répété tant que la règle de connectivité peut s'appliquer. Le client 
reçoit une notification de l'établissement de la connexion avec un autre client ce 
qui lui permet de démarrer l'échange de données.

L'API RTCDataChannel permet à chaque pair d'échanger des données arbitraires 
avec d'autres à partir du navigateur avec des propriétés de livraison 
personnalisables -- fiable ou non fiable (Section \ref{sec:fiabilite}), ordonné ou non 
ordonné (Section \ref{sec:ordre}). 

Dans 3DEvent, le choix d'avoir un transport 
fiable et non ordonné a été fait pour respectivement 
garantir l'arrivée d'une donnée émise par l'utilisateur au sein de l'application et 
permettre des échanges asynchrones.
\improve{add \S sur "en cas de défaillance? }
En cas d'arrêt soudain du serveur, si une connexion a été établie préalablement 
entre les clients et est toujours en cours, elle n'est pas impactée par la défaillance 
du serveur.

\subsection{Données d'échange}
En sachant que le modèle est conçu pour des applications web, 3DEvent a besoin 
d'un format de données permettant de faire communiquer des acteurs hétérogènes 
du système. Le format \gls{JSON}, dérivé de la notation des objets du langage 
JavaScript, est lisible et inter-opérable. Le Listing \ref{jsonexemple} montre un 
exemple d'événement en format JSON.
%Son pendant binaire est le format \gls{BSON}.
%Le format de fichier \gls{glTF} se base sur la représentation \gls{JSON} afin de 
%décrire une scène 3D.
Ce type de données est assez abstrait et suffisamment générique pour 
représenter n'importe quelle donnée. Par exemple le format de fichier 
\gls{glTF}\info{ref gltf section} se base sur la représentation \gls{JSON} afin de 
décrire une scène 3D.
Le format \gls{JSON} est aussi utilisé pour la sérialisation et la désérialisation des 
objets transmis par RTCDataChannel. 

L'\acrshort{API} RTCDataChannel supporte beaucoup de types de données 
différents (chaînes de caractères, types binaires : Blob, ArrayBuffer\dots). Dans 
un environnement multi-utilisateurs tel que 3DEvent avec des données hétérogènes (3D, images, 
informations) l'inter-opérabilité est facilité.

\begin{lstlisting}[language=json,firstnumber=1,label=jsonexemple,caption=Mesh 
added to Scene event and parameters]
{
{
"sceneId": "scene-turbine",
"meshId":"406514c6-306b-f0f9643a037e",
"geometryId": "37076875-ea1c-bbd300481345",
"name": "blade",
"color": "#963912"
},
"version": 17,
"author": "Foo"
}
\end{lstlisting}

\subsection{Synchronisation des données}
\subsubsection{Persistance à court terme}
Le navigateur (client) offre un espace de stockage avec l'interface \textit{Storage} 
de l'API Web Storage qui donne accès au \texttt{session storage} ou au  
\texttt{local storage}. Ce stockage fonctionne sur un système de clé/valeur qui rend facile l'accès aux 
évènements enregistrés sur le client. La configuration du client est également 
stockée localement. Grâce à ce système , il est possible d'avoir une 
persistance des données à travers les sessions du navigateur. Le contenu stocké 
correspond aux données générées par un utilisateur et par ses collaborateurs. Les 
répliques stockées sur chaque navigateur permettent à un utilisateur une plus 
grande 
autonomie en cas de déconnexion. C'est également un moyen de distribuer entre les clients les 
mises à jour qu'ils génèrent grâce au protocole de 
\gls{streaming3D} (cf. \ref{streamingprotocol}) sans passer par le serveur.

événements enregistrés sur le client. La configuration du client est également 


\subsubsection{Protocole de streaming pour la synchronisation}
\label{streamingprotocol}

Il existe plusieurs méthodes de transmission de contenu au sein d'un réseau P2P 
que l'on peut catégoriser selon deux modes : le téléchargement (\textit{download}) 
et le flux continu (\textit{streaming}). Le téléchargement requière que le contenu 
soit entièrement téléchargé pour pouvoir être lu, tandis que le flux continu peut 
être lu au fur et à mesure de sa récupération. Ce dernier mode, principalement 
utilisé pour la lecture de vidéo en ligne, s'applique bien à la transmission de 
contenu 3D : niveaux de détails \cite{Chu2012,Hu2008}, progressif 
\cite{Cheng2009,Limper2014}, mise en cache \cite{Jia2014}, compression / 
décompression
\cite{Lavoue2013,Ponchio2015,Maglo2013a}. 
Une catégorisation plus précise du flux continu peut être donnée selon quand le 
contenu est généré : 
en direct (\textit{live} ou \textit{push}) et à la demande (\textit{on-demand} ou 
\textit{pull}).  


Le mécanisme de routage que nous avons utilisé dans \cite{Desprat2015a} est 
proche du routage de GNutella. 



\subsubsection{Gestion des événements des agrégats}

La configuration des connexions RTCDataChannel peut être configurée de selon 
les critères (Section \todo{ref section config}) de fiabilité et d'ordonnancement. 
Dans l'intergiciel de 3DEvent, cette configuration est : non fiable et non ordonnée. 
C'est l'application qui est en charge de \og ré-ordonner\fg{} les événements. En 
effet, les événements sont associés à des agrégats. Comme chaque agrégat 
possède un flux d'événements dédié et numéroté il est alors simple de les 
ordonner lorsque. 
Lors de la synchronisation (Section \todo{ref section sync}), les 
événements manquants sont demandés successivement à tous les pairs. Lors 
qu'un pair reçoit les événements il les stocke dans le flux correspondant à 
l'agrégat. Si un événement est manquant, les suivants sont quand même stockés 
en laissant l'espace de l'événement manquant dans le tableau. L'événement est 
redemandé jusqu'à son obtention, laquelle provoque le déblocage de l'état de 
l'agrégat jusqu'au prochain événement manquant (ou la fin du flux). Les 
événements qui ont été stockés \og en attendant\fg{} permettent à l'application 
d'être directement en capacité de poursuivre la construction de l'état de l'agrégat. 
Cette mécanique tire avantage des Snapshots (présentés dans \todo{ref section}) 
qui réduisent la taille des flux en mémoire pour chaque agrégat qui part déjà d'un 
état avancé.

\subsubsection{Sélection fantôme}
Les interactions utilisateurs doivent être adaptées à la collaboration et aux 
manipulations à effectuer. Pour cela, l'éditeur 3DEvent introduit la fonctionnalité de 
sélection \og fantôme\fg{}. Lorsqu'un utilisateur souhaite sélectionner un objet de 
la scène, l'objet original ($O_o$) est 
cloné et devient l'objet fantôme ($O_f$) (mêmes propriétés, représenté avec de la 
transparence d'où le terme \og fantôme\fg{}). 
L'objet $O_f$ prend alors le focus de sélection pour que l'utilisateur le manipule à 
la place de l'objet $O_f$. 
En différenciant l'actuel objet que l'utilisateur souhaite sélectionné ($O_{o}$) de 
celui manipulé ($O_f$), plusieurs aspects de l'interactions sont mis en valeur pour :
\begin{itemize}
	\item l'ergonomie dans l'environnement 3D :
	$O_f$ est un objet temporaire qui permet à l'utilisateur
	d'avoir une visualisation de l'objet en cours de manipulation tout en 
	conservant le dernier état de $O_o$ visible. 
	$O_o$ peut être considéré comme un point de repère visuel pour l'utilisateur 
	lorsqu'il effectue sa manipulation. 
	L'$O_f$ a aussi un rôle d'intermédiaire entre l'utilisateur et la 
	finalité de l'interaction en donnant un support visuel à sa réflexion experte.
	Grâce à $O_f$, l'utilisateur peut également révoquer sa manipulation en 
	cours sans avoir eu d'impact sur $O_o$ en évitant des actions inutiles (faire l'action 
	puis la défaire) pour le métier et coûteuses pour le réseau.
	
	\item la collaboration : si un collaborateur effectue une modification 
	à destination du même $O_o$ alors la représentation de $O_o$ chez 
	l'utilisateur est également modifiée. $O_f$ par contre ne subit pas d'impact ; 
	l'utilisateur peut continuer sa manipulation et~/~ou l'ajuster en fonction des 
	nouvelles informations liées à $O_o$ ou même révoquer sa manipulation 
	en cours si cela lui convient.

	\item le métier : seules les manipulations menées à terme sont 
	considérées comme des commandes. Cela évite d'avoir des événements qui ne 
	sont pas pertinents pour le métier dans le journal d'événements (comme lorsque 
	l'utilisateur change d'idée lors de l'interaction ou
	suite à une intervention concurrente). L'utilisateur n'a un impact sur l'application 
	que lorsqu'une modification métier est réalisée.
	
	\item le réseau : l'information importante à faire transité est l'événement 
	correspondant à la modification métier pas toutes les positions intermédiaires 
	même si intuitivement l'idée de temps réel pourrait conduire à cette solution. La 
	quantité de messages produite surchargerai à la fois le réseau et le fil 
	d'exécution principale de l'application. En effet, WebRTC a l'inconvénient pour le 
	moment de ne pouvoir s'exécuter dans un \textit{Web Worker} (fil d'exécution 
	parallèle en JavaScript). Cette solution imposerai des 
	latences réseau et d'\gls{IU} qui affecteraient gravement l'expérience utilisateur
	sans apporter d'informations supplémentaires à l'aspect métier de la 
	collaboration. 
\end{itemize}

Lorsque l'utilisateur relâche $O_f$, alors la modification intentée s'applique sur 
$O_o$ avec le principe du \textit{Last Write Last Win} (le dernier gagne).


\section{Résumé des choix techniques}

\paragraph{Base de données NoSQL}\label{p:nosql} L'évolution de 
l'utilisation du web en tant que plateforme applicative a encouragé le changement 
dans le stockage des données pour de nouveaux besoin supportant de larges 
volumes de données (comme les données 3D). Une base de données \gls{NoSQL} 
fournit un schéma libre et dynamique ainsi qu'une API de requête riche pour la 
manipulation de données. De plus, la possibilité d'enrichir un document à la volée 
facilite l'évolution des objets (3D) et la maintenance.\info{parler de la 
	maintenance?} de l'application.\improve{eventual consistency, scalabilty}
3DEvent intègre un système de persistance sur le long terme caractérisé par une 
base de données \gls{NoSQL}.

La base de données (\gls{NoSQL}) conserve tous les évènements qui 
sont produits dans une scène.\improve{voir ce qu'il faut ajouter éventuellement le 
	schéma aussi.} 
La mise en place d'une base de données centralisée apporte de la robustesse au 
système en lui fournissant un référent sans toutefois le surcharger ainsi qu'une 
expérience utilisateur transparente limitant les interruptions de service.
\section{Bilan}


