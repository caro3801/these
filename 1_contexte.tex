%!TEX root = main.tex
\chapter{Contexte}
\chaptertable



\section{La collaboration}
La collaboration \info{https://wiki.p2pfoundation.net/Collaboration} est souvent 
définie \info{ref}comme un processus récursif
\footnote{Le « principe de boucle 
	récursive » se retrouve dans le concept de la pensée complexe. Edgar Morin  
	explique qu'« un processus récursif est un processus où les produits et les 
	effets 
	sont en même temps causes et producteurs de ce qui les produit » 
	\cite[p. 100]{Morin1990}.} où deux (ou plus) personnes (ou organisations) 
travaillent ensemble à la croisée de buts communs 
en partageant leurs connaissances pour apprendre et bâtir un consensus. 
La collaboration permet l'émergence de conceptions partagées dans la réalisation 
de visions partagées dans des environnements et des systèmes complexes. 
Les imbrications de chaque domaine et la transdiciplinarité sont acceptés comme 
dans le concept de pensée complexe. D'ailleurs, dans sa définition de la 
complexité, Edgar Morin fait référence au sens étymologique latin \og 
complexus\fg{} qui signifie \og ce qui est tissé ensemble\fg{} \cite{Morin1990a}.
La plupart des collaborations requièrent un élément dirigeant qui peut prendre une 
forme sociale (personne) au sein d'un groupe décentralisé et égalitaire (tout le 
monde au même niveau). L'élément dirigeant va souvent aider également trouver 
des consensus. 
La disponibilité des ressources peut également devenir un élément dirigeant dans 
la collaboration.
Une équipe travaillant de manière collaborative peut concentrer plus de 
ressources, de reconnaissances et de récompenses lors d'une 
compétition comportant des ressources finies. 
La collaboration est aussi présente dans la recherche de buts opposés mettant en 
avant la notion de collaboration contradictoire (en opposition avec la collaboration 
constructive) ; la négociation et la compétition peuvent également faire partie du 
terrain 
de la collaboration.

Une application collaborative peut aussi intégrer les notions de coordination et de 
coopération:
\begin{itemize}
	\item La \textbf{coordination} se base sur le principe d'harmonisation des 
	tâches, des rôles et du calendrier dans des systèmes et environnements 
	simples.
	\item La \textbf{coopération} permet de résoudre des problèmes dans des 
	systèmes et environnements complexes dans le les participants aurait été 
	incapables (temps, espace, connaissance, matériel) d'accomplir le travail seul.
\end{itemize}

Dans les années 2000, deux classifications ont été retenue concernant la 
collaboration.
La première classification, décrite en 2007 par Gotta \cite{Gotta2007}, propose 
un modèle segmentant la collaboration de manière structurée en quatre catégories 
: de la plus dirigée à la plus volontaire, en passant par l'hybride.
\begin{itemize}
	\item \textit{Collaboration centrée processus.} Les conditions requises du 
	processus 
	nécessitent l'engagement de l'utilisateur, qui doit, de part son rôle ou sa 
	responsabilité, diriger ses efforts dans la collaboration avec les autres. Cette 
	stratégie se concentre sur les activités de manipulation collaborative 3D plutôt 
	que sur leur contexte organisationnel afin de favoriser la synergie autour de la 
	réussite d'un processus. Par exemple, pour la création et l'utilisation d'un 
	modèle 	3D dans un \gls{BIM}, il s'agit de favoriser la prise de décision sur un 
	projet et 	communiquer à propos.
	
	\item \textit{Collaboration centrée activité}
	Les activités partagées créent un sentiment de co-dé\-pendance qui motive la 
	collaboration entre les membres. La co-dépendance prend l'avantage sur le 
	propre intérêt de chacun comme motivation pour collaborer. Le groupe a besoin 
	de chacun pour que l'objectif soit considéré comme réalisé. L'intérêt personnel 
	ou l'allégeance à l'esprit d'équipe peut aussi promouvoir la collaboration. Par 
	exemple, la visualisation collaborative des activités des différents contributeurs 
	dans l'\gls{EVC3D} permet à chacun de rendre compte de ses réalisations. Ce 
	type de collaboration doit beaucoup à l'ergonomie de l'activité qui insiste sur la 
	différence entre le travail prescrit et le travail réel : la tâche et l'activité.
	
	\item \textit{Collaboration centrée communauté.}
	La participation de la communauté à la collaboration induit la contribution. En 
	effet, les interactions professionnelles ou sociales peuvent encourager ou 
	persuader les utilisateurs de partager leurs informations ou connaissances 
	(exemple : les logiciels \textit{open-source})
	
	\item \textit{Collaboration centrée réseau.}
	Les connexions réseau favorisent la coopération réciproque. Dans le but de 
	récupérer des avis ou du savoir faire externe, un utilisateur peut faire appel à 
	son réseau social pour supplémenter une autre interaction collaborative. 
	Souvent utilisé dans le cadre d'urgences écologiques ou sanitaires (exemple: 
	contributions OpenStreetMap lors d'ouragan ou de tsunami), la collaboration 
	centrée réseau est très présente dans des situations l'expertise est fortement 
	valorisée comme dans les \gls{BIM} ou la visualisation scientifique, les 
	contributeurs peuvent être intégrés en fonction des besoins des utilisateurs déjà 
	présents sur le projet.
\end{itemize}
Dans le cadre de cette thèse, en se référant à cette première classification, les 
aspects centrés sur les activités sont mis en avant. En effet, la collaboration 
portant sur la modélisation 3D attend un résultat porté sur l'activité de conception. 
Celle-ci nécessite l'implication de personnes avec différentes compétences / 
connaissances qui doivent s'entraider pour parvenir à la mise en commun des 
objets 3D et réaliser leur objectif. \info{add figure}

Une seconde classification proposée par Callahan et al. 
\cite{Callahan2008} s'intéresse au triplé collaboration par équipe, collaboration 
communautaire, collaboration en réseau. En contraste avec la précédente 
classification qui se concentre sur les équipes et une collaboration formelle et 
structurée, celle-ci offre plus d'ouverture :
\begin{itemize}
	\item \textit{Collaboration par équipe.}
	Dans une équipe tous les membres se connaissent. Il y a une interdépendance 
	claire des tâches à effectuer où la réciprocité est attendue, avec un échéancier 
	et des objectifs explicites. Pour réaliser son but, l'équipe doit réaliser les tâches 
	dans un temps imparti. La collaboration par équipe suggère que les membres 
	coopèrent sur un pied d'égalité (bien qu'il y ait souvent un chef) recevant une 
	reconnaissance égale.
	
	\item \textit{Collaboration communautaire.}
	L'objectif de ce type de collaboration est plus orienté sur la possibilité 
	d'apprendre que sur la tâche elle-même, même si les centres d'intérêt sont 
	partagés par la communauté. Les utilisateurs sont là pour partager et construire 
	la connaissance plus que compléter un projet. Les membres vont aller voir leur 
	communauté pour demander de l'aide sur un problème ou un avis et ramener la 
	solution à implémenter dans leur équipe. L'adhésion peut être limitée et 
	explicite, mais les périodes de temps sont souvent ouvertes. Les membres 
	sont considérés comme égaux bien que les plus expérimentés peuvent avoir 
	des statuts privilégiés. La réciprocité est un facteur important dans la 
	communauté pour que cela fonctionne.
	
	\item \textit{Collaboration en réseau.}
	La collaboration en réseau est une sur-couche de la collaboration 
	traditionnellement centrée sur la relation d'une équipe ou d'une communauté. 
	Elle s'appuie sur une action individuelle et un intérêt personnel qui resurgissent 
	ensuite sur le réseau sous la forme de personnes qui contribuent ou cherchent 
	quelque chose à partir du réseau. L'adhésion et les périodes sont ouvertes et 
	non limitées. Il n'y a pas de rôle explicite. Les membres ne se connaissent pas 
	forcément. Le pouvoir est distribué. Cette forme de collaboration est dirigée par 
	l'avènement des réseaux sociaux, des accès à internet omniprésents et la 
	capacité de se connecter avec divers individus malgré la distance.
\end{itemize}
Cette thèse, en se référant à cette seconde classification, s'intéresse plutôt 
sur la collaboration par équipe. La conception d'un objet 3D et ses différentes 
phases de modélisation constitue une problématique nécessitant l'apport de 
plusieurs intervenants avec leurs capacités propres et travaillant de concert à la 
réalisation d'un objectif commun dans un temps imparti (exemple : revue de 
projet). Là où la coopération et l'effort conjoint pour réaliser un objectif sont 
nécessaires, le facteur temps reste un élément important à prendre en compte 
pour évaluer la productivité d'une session collaborative.

Le travail dans un \gls{EVC3D} facilite la compréhension 
de certaines problématiques liées à l'espace 3D ; c'est également un point de 
rencontre et d'échanges entre contributeurs sur le court terme et le long terme. 
Le croisement de ces deux dimensions, spatiale et temporelle, implique une 
multiplication des points de vue et donc des données à traiter sur le problème lors 
de la collaboration.

\subsection{Les systèmes collaboratifs}
\subsection{Les systèmes d'édition collaboratifs}
\subsubsection{Modèle d'édition collaborative}

\section{La collaboration 3D en accord avec l'évolution du web}

\subsection{Introduction}
\subsection{Le web et le P2P : WebRTC}
\subsection{Le web et la 3D :  WebGL}

\section{Les architectures évènementielles pour la collaboration}
\subsection{Sensibilisation lors de la collaboration}
\subsection{Intégration des contraintes métiers}
