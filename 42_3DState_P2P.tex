\subsection{Implantation des composants pour la 
	communication}
		\paragraph{Serveur}
		
Le prototype utilise un serveur Node.js qui sert à la fois de serveur de 
\textit{signaling} pour mettre en liaison les utilisateurs mais il sert également 
de lien avec la base de données. Node.js permet d'utiliser du \gls{JS} côté 
serveur, simplifiant la compréhension et la maintenance de l'environnement. 
L'application utilise le micro-framework ExpressJS par dessus Node.js qui a 
l'avantage de simplifier le routage réseau.

		\paragraph{Base de données}

Pour assurer la sauvegarde au long-terme des modifications faite par les 
utilisateurs, la base de données MongoDB stocke l'état de chaque scène de 
l'application. L'utilisation d'une base de données NoSQL orientée document 
permet de stocker toutes les données pertinentes ensemble, dans un même 
document. Un document est auto-descriptif et peut imbriquer des données 
dans une structure d'arbre hiérarchisé. Une collection regroupe des 
documents. Dans ce prototype, il existe trois collections : \textit{Scenes} et 
\textit{Geometries} et \textit{Meshes}. 
La collection \textit{Scenes} contient chaque document concernant une 
scène : identifiant de la scène, métadonnées de l'espace virtuel (nom de la 
scène, liste des utilisateurs), liste des maillages inclus (identifiant maillage). 
La collection \textit{Geometries} comprend les géométries importées dans 
l'application sous la forme de : identifiant de la géométrie, données 3D.
La collection \textit{Meshes} représente les maillages utilisés dans les 
scènes : identifiant du maillage, nom du maillage, identifiant de la géométrie.
Les paramètres liés aux opérations \gls{CRUD} de la base de données sont 
fournis part la requête \gls{REST}. 

		\paragraph{Couche \gls{P2P}} 

La compatibilité entre navigateurs 
(cross-plateforme) concernant 
l'\gls{API} Datachannel issue du standard \gls{WebRTC} n'est pas encore 
effective. C'est pourquoi ce prototype ne fonctionne que sur les dernières 
version de Chrome (v49+), Firefox (v55+) et Opera (v47+). 
PeerJS\footnote{PeerJS  \url{http://peerjs.com}} est une 
bibliothèque \textit{open source} qui enveloppe WebRTC et fournie une \gls{API} 
de connexion navigateur-à-navigateur.
Le client possède un identifiant, donné par le serveur de \textit{signaling} lors 
de sa première connexion, qui lui permet de se connecter à un pair distant 
dont il a obtenu l'identifiant par le serveur. Le mécanisme 
de \textit{signaling} est délégué à PeerJS qui se charge de l'implantation de 
\gls{ICEF} et des problématiques \gls{NAT}. 


		\paragraph{Couche applicative}
		
Les différentes actions utilisateurs sont relayés par un système 
événementiel. L'interface intègre une couche de messagerie pour notifier les 
composant d'un nouvel événement \gls{JS}. Dans le prototype, la 
bibliothèque 
\textit{js-signals} sert de gestionnaire d'événements aux différents 
composants de l'\gls{IU} pour 
communiquer. Dans ce contexte chaque \textit{signal} possède un 
contrôleur, qui simplifie le contrôle de la réaction des événements en 
fournissant des fonctionnalités de souscription et de diffusion aux
événements~\gls{JS}. 

En comparaison avec une implémentation \textit{Event 
emitter / dispatcher} et \gls{PubSub}, le patron de conception 
Observateur dans \textit{js-signals} n'utilise pas de chaîne de caractère pour 
décrire les types d'événements \gls{JS} (évite les erreurs).
Une instance \textit{signal} enregistre des procédures (événements \gls{JS}, 
callbacks) qui peuvent être asynchrones. 
Lorsqu'il est intercepté -- n'importe où dans le contexte de l'application -- les 
procédures qui lui sont associées sont également déclenchées.