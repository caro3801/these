%!TEX root = main.tex
\chapter{Introduction}
\chaptertable
\section{Contexte}
Le besoin de collaboration et de manipulation en temps réel d'objets en \gls{3D} 
ainsi que leur contrôle de version a été très tôt une raison de la dissémination des 
outils pour la \gls{CAO}. Tandis que l'ingénierie et la visualisation scientifique ont 
produit des quantités de données massives dans des domaines tels que la 
conception architecturale, l'industrie des jeux ou encore l'impression \gls{3D}, un 
besoin croissant de maintenir, visualiser et manipuler de larges scènes \gls{3D} 
polygonales pouvant être éditées par de multiples utilisateurs de manière 
concurrente se fait sentir \info{better read the first 
	one!}\cite{Chandrasegaran2013,Wu2014}. Or la quantité et la taille des données 
étant toujours croissante, il devient de plus en plus difficile de partager les 
modèles, particulièrement avec les utilisateurs n'ayant pas accès aux derniers 
logiciels et matériels graphiques. En conséquence,
le développement de plateformes légères déployées sur le web pour répondre à ces 
besoins est en forte progression. 
Dans un \gls{EV3D}, la simulation virtuelle et simulation distribuée d'un environnement 3D, un grand nombre d'utilisateurs situés à différents endroits géographiques peuvent 
interagir les uns avec les autres en temps réel. Un \gls{EVC3D} insiste sur l'aspect collaboratif des interactions qui se produisent entre les utilisateurs.
Cette distorsion de l'espace physique et temporel impose aux \glspl{EVC3D} des 
mécanismes de communication rapides, conservant un environnement cohérent
des données partagées durant la collaboration. 
L'utilisation d'un client comme medium pour 
accéder à l'\gls{EVC3D} est requise pour envoyer des demandes au serveur. Le 
client peut être manipulé par un utilisateur ou un agent logiciel (bot).
Un \gls{EVC3D} se compose également d'un protocole de communication qui 
permet aux différents clients d'échanger les mises à jours correspondant aux 
modifications effectuées dans l'espace 3D virtuel partagé. La distribution des 
données devient alors un enjeu majeur dans ce type d'application en terme de 
temps, de sécurité et de fiabilité. 



\subsection{La collaboration}
La collaboration \info{https://wiki.p2pfoundation.net/Collaboration} est souvent 
définie comme un processus récursif
\footnote{Le « principe de boucle 
	récursive » se retrouve dans le concept de la pensée complexe. Edgar Morin  
	explique qu'« un processus récursif est un processus où les produits et les 
	effets 
	sont en même temps causes et producteurs de ce qui les produit » 
	\cite[p. 100]{Morin1990}.} où au moins deux acteurs (personnes ou organisations) 
travaillent ensemble à la croisée de buts communs 
en partageant leurs connaissances pour apprendre et bâtir un consensus. 
La collaboration permet l'émergence de conceptions partagées dans la réalisation 
de visions partagées dans des environnements et des systèmes complexes. 
Les imbrications de chaque domaine et la transdiciplinarité sont acceptés comme 
dans le concept de pensée complexe. Dans sa définition de la 
complexité, Edgar Morin fait d'ailleurs référence à l'étymologie latine \og 
complexus\fg{} qui signifie \og ce qui est tissé ensemble\fg{} \cite{Morin1990a}.
La plupart des collaborations requièrent un élément dirigeant qui peut prendre une 
forme sociale (personne) au sein d'un groupe décentralisé et égalitaire (tout le 
monde au même niveau). L'élément dirigeant va également souvent aider à trouver 
des consensus. 
La disponibilité des ressources peut aussi devenir un élément dirigeant dans 
la collaboration.
Une équipe travaillant de manière collaborative a la possibilité de concentrer plus de 
ressources, de reconnaissances et de récompenses lors d'une 
compétition comportant des ressources finies. 
La collaboration est aussi présente dans la recherche de buts opposés mettant en 
avant la notion de collaboration contradictoire (en opposition avec la collaboration 
constructive) ; la négociation et la compétition peuvent également faire partie du 
terrain 
de la collaboration.

Une application collaborative peut aussi intégrer les notions de coordination et de 
coopération:
\begin{itemize}
	\item La \textbf{coordination} se base sur le principe d'harmonisation des 
	tâches, des rôles et du calendrier dans des systèmes et environnements 
	simples.
	\item La \textbf{coopération} permet de résoudre des problèmes dans des 
	systèmes et environnements complexes dans lesquels les participants auraient 
	été 
	incapables (temps, espace, connaissance, matériel) d'accomplir le travail seuls.
\end{itemize}

Dans les années 2000, deux classifications ont été retenues concernant la 
collaboration.
La première classification, décrite en 2007 par Gotta \cite{Gotta2007}, propose 
un modèle segmentant la collaboration de manière structurée en quatre catégories 
: de la plus dirigée à la plus volontaire, en passant par l'hybride.
\begin{itemize}
	\item \textit{Collaboration centrée processus.} Les conditions requises du 
	processus 
	nécessitent l'engagement de l'utilisateur, qui doit, de par son rôle ou sa 
	responsabilité, diriger ses efforts dans la collaboration avec les autres. Cette 
	stratégie se concentre sur les activités de manipulation collaborative 3D plutôt 
	que sur leur contexte organisationnel afin de favoriser la synergie autour de la 
	réussite d'un processus. Par exemple, pour la création et l'utilisation d'un 
	modèle 	3D dans un \gls{BIM}, il s'agit de favoriser la prise de décision sur un 
	projet et de communiquer à ce propos.
	
	\item \textit{Collaboration centrée activité.}
	Les activités partagées créent un sentiment de co-dé\-pendance qui induit la 
	collaboration entre les membres. La co-dépendance prend l'avantage sur l'intérêt personnel comme motivation à collaborer. Le groupe a besoin 
	de chacun pour que l'objectif soit considéré comme réalisé. L'intérêt personnel 
	ou l'allégeance à l'esprit d'équipe peut aussi promouvoir la collaboration. Par 
	exemple, la visualisation collaborative des activités des différents contributeurs 
	dans l'\gls{EVC3D} permet à chacun de rendre compte de ses réalisations. Ce 
	type de collaboration doit beaucoup à l'ergonomie de l'activité qui insiste sur la 
	différence entre le travail prescrit et le travail réel : la tâche et l'activité.
	
	\item \textit{Collaboration centrée communauté.}
	La participation de la communauté à la collaboration induit la contribution. En 
	effet, les interactions professionnelles ou sociales peuvent encourager ou 
	persuader les utilisateurs de partager leurs informations ou connaissances 
	(exemple : les logiciels \textit{open-source})
	
	\item \textit{Collaboration centrée réseau.}
	Les connexions réseau favorisent la coopération réciproque. Dans le but de 
	récupérer des avis ou du savoir-faire externe, un utilisateur peut faire appel à 
	son réseau social pour supplémenter une autre interaction collaborative. 
	Souvent utilisée dans le cadre d'urgences écologiques ou sanitaires (exemple: 
	contributions OpenStreetMap lors d'ouragans ou de tsunamis), la collaboration 
	centrée réseau est très présente dans des situations où l'expertise est 
	fortement 
	valorisée comme dans les \gls{BIM} ou la visualisation scientifique. Les 
	contributeurs peuvent être intégrés en fonction des besoins des utilisateurs déjà 
	présents sur le projet.
\end{itemize}
Dans le cadre de cette thèse, en se référant à cette première classification, les 
aspects centrés sur les activités sont mis en avant. En effet, la collaboration 
portant sur la modélisation 3D attend un résultat porté sur l'activité de conception. 
Celle-ci nécessite l'implication de personnes dotées de différentes compétences/ connaissances qui doivent s'entraider pour mettre en commun des 
objets 3D et réaliser leur objectif. \info{add figure}

Une seconde classification proposée par Callahan et al. 
\cite{Callahan2008} s'intéresse au triplé \og{} collaboration par équipe\fg{}, \og{} 
collaboration 
communautaire\fg{} et \og{} collaboration en réseau\fg{}. En contraste avec la 
précédente 
classification qui se concentre sur les équipes et une collaboration formelle et 
structurée, celle-ci offre davantage d'ouverture :
\begin{itemize}
	\item \textit{Collaboration par équipe.}
	Dans une équipe, tous les membres se connaissent. Il y a une interdépendance 
	claire des tâches à effectuer où la réciprocité est attendue, avec un échéancier 
	et des objectifs explicites. Pour réaliser son but, l'équipe doit réaliser les tâches 
	dans un temps imparti. La collaboration par équipe suggère que les membres 
	coopèrent sur un pied d'égalité (bien qu'il y ait souvent un chef) recevant une 
	reconnaissance égale.
	
	\item \textit{Collaboration communautaire.}
	L'objectif de ce type de collaboration est plus orienté sur la possibilité 
	d'apprendre que sur la tâche elle-même, même si les centres d'intérêt sont 
	partagés par la communauté. Les utilisateurs sont là pour partager et construire 
	la connaissance davantage que pour compléter un projet. Les membres vont aller voir leur 
	communauté pour demander de l'aide sur un problème ou un avis et rapporter la 
	solution à implémenter dans leur équipe. L'adhésion peut être limitée et 
	explicite, mais les périodes de temps sont souvent ouvertes. Les membres 
	sont considérés comme égaux, bien que les plus expérimentés puissent avoir 
	des statuts privilégiés. La réciprocité dans la 
	communauté est un facteur important pour que cela fonctionne.
	
	\item \textit{Collaboration en réseau.}
	La collaboration en réseau est une surcouche de la collaboration 
	traditionnellement centrée sur la relation d'une équipe ou d'une communauté. 
	Elle s'appuie sur une action individuelle et un intérêt personnel qui resurgissent 
	ensuite sur le réseau, sous la forme de personnes qui contribuent ou cherchent 
	quelque chose à partir du réseau. L'adhésion et les périodes sont ouvertes et 
	non limitées. Il n'y a pas de rôle explicite. Les membres ne se connaissent pas 
	forcément. Le pouvoir est distribué. Cette forme de collaboration est dirigée par 
	l'avènement des réseaux sociaux, d'un accès omniprésent à internet et de la 
	capacité de se connecter avec divers individus malgré la distance.
\end{itemize}
Cette thèse, en se référant à cette seconde classification, se concentre sur la collaboration par équipe. La conception d'un objet 3D, et ses différentes 
phases de modélisation, constitue une problématique nécessitant l'apport de 
plusieurs intervenants avec leurs capacités propres et travaillant de concert à la 
réalisation d'un objectif commun dans un temps imparti (exemple : revue de 
projet). Là où la coopération et l'effort conjoint pour réaliser un objectif sont 
nécessaires, le facteur temps reste un élément important à prendre en compte 
pour évaluer la productivité d'une session collaborative.

Le travail dans un \gls{EVC3D} facilite la compréhension 
de certaines problématiques liées à l'espace 3D ; c'est également un point de 
rencontre et d'échanges entre contributeurs sur le court terme et le long terme. 
Le croisement de ces deux dimensions, spatiale et temporelle, implique une 
multiplication des points de vue et donc des données à traiter sur le problème lors 
de la collaboration.


\subsection{Les environnements virtuels collaboratifs 3D}

Un \gls{EVC3D} est un environnement virtuel 3D où plusieurs utilisateurs locaux 
ou à distance peuvent se rejoindre et partager une expérience collaborative 
interactive en 3D. La principale caractéristique d'un \gls{EVC3D} est la simulation 
immersive et interactive 3D d'environnements virtuels, tels que les jeux sérieux ou 
multijoueurs. \gls{EVC3D} permet à plusieurs utilisateurs d'interagir les uns avec 
les autres en temps (quasi) réel, même s'ils sont situés dans des lieux différents. 
D'autres fonctionnalités sont proposées par ce type de plateformes comme le 
partage d'espace, de présence et du temps. Selon Singhal et Zyda 
\cite{Singhal1999}, \gls{EVC3D} est constitué de quatre composants principaux :
\begin{enumerate*}[label=(\roman*)]
	\item un moteur graphique pour l'affichage ;
	\item des appareils de contrôle et de communication :
	\item un système de traitement ;
	\item et un réseau de transmission des données. 
\end{enumerate*}

Ce type d'environnement nécessite également d'impliquer l'utilisateur dans une 
boucle \og ac\-tion~/~per\-ception\fg{} pour lui permettre prendre conscience
que ses actions contribuent à la modification de l'environnement virtuel. Les 
interactions classiques se font par le biais d'appareils dédiés.
Il existe trois catégories d'interactions selon Chris Hand \cite{Hand1997}: la 
navigation (interaction avec le point de vue de l'utilisateur) ; la manipulation des 
objets virtuels issus de l'environnement virtuel (sélection d'objet, manipulation 
d'objet) ; et le contrôle de l'application (interaction avec une interface 3D pour 
modifier des paramètres de l'environnement virtuel).

La manipulation d'objets figure parmi les interactions les plus fondamentales 
lorsqu'on 
aborde la \gls{CAO}, le \gls{BIM} ou le \gls{PLM}. La manipulation 
collaborative d'objets virtuels par plusieurs utilisateurs fait ainsi partie des 
nouveaux 
besoins liés aux développement de ces activités.
Elle s'avère indispensable dans plusieurs types 
d'applications comme le prototypage virtuel, les simulations d'entraînement ou la 
simulation d'assemblage et de maintenance. L'expérience de modélisation 
\gls{CAO} collaborative peut être accompagnée de mécanismes ludiques pour 
intégrer une capture temps réel de la connaissance \cite{Kosmadoudi2013}.
Ces lieux virtuels sont l'occasion pour les utilisateurs de participer de manière 
naturelle et efficace à la création et à la vie de l'objet manipulé dans 
l'environnement virtuel sans danger et à bas coût. 
Une autre utilisation des \gls{EVC3D} sert la navigation virtuelle, c'est-à-dire les 
visites collaboratives (musées, héritage culturel, les revues de projets 
architecturaux / urbains, ou encore les jeux collaboratifs (courses, simulateur de 
jeux collectifs). 
Les \gls{EVC3D} permettent non seulement aux utilisateurs de communiquer de 
manière distante, mais aussi de partager des 
interactions dans le monde virtuel. 
Ces interactions peuvent s'appliquer à différents objets, différentes parties du 
même objet, ou encore sur la même partie (en même temps) d'un objet virtuel 
partagé.
Plusieurs problèmes, liés aux domaines des systèmes distribués (et des 
protocoles 
réseaux) et d'\acrshort{IHM}, doivent être résolus pour concevoir un 
\gls{EVC3D} fiable, ergonomique et en temps réel. 



\subsection{Les systèmes d'édition collaboratifs}
En 1989, Ellis et Gibbs proposent la définition d'un \textit{groupware}, terme 
généralement traduit par collecticiel : 
\textcquote{Ellis1989}{\textit{computer-based systems that support groups of 
		people engaged in a common task (or goal) and that provide an interface to a 
		shared environment.}}. 
Une application collaborative est une application qui accompagne un groupe de 
participants dans la réalisation d'une tâche ou d'un objectif en fournissant une 
interface vers un environnement partagé. 
Cette définition, suffisamment large, englobe un ensemble très hétérogène d'applications 
permettant le travail collaboratif. 

Vidot propose de distinguer les différentes catégories de collecticiels en fonction 
de leur rapport au temps et à l'espace \cite{Vidot2002}. 
L'aspect temporel se réfère à la synchronicité des interactions. 
Une synchronicité élevée implique que les actions des utilisateurs sont visibles en 
temps réel (action synchrone). 
Au contraire, une synchronicité faible, implique qu'un temps notable s'écoule entre 
l'action d'un utilisateur et sa visibilité chez les autres (action asynchrone). L'aspect 
spatial s'intéresse à la répartition géographique des utilisateurs. Les 
environnements sont dits répartis lorsque les utilisateurs sont physiquement 
distants. Cette thèse concerne les applications collaboratives appartenant à la 
catégorie 
des applications synchrones réparties.

Un modèle d'édition collaborative en temps réel permet à plusieurs utilisateurs de 
visualiser et d'éditer de manière simultanée le même document (texte, image, 3D) 
depuis plusieurs sites connectés par un réseau de communication. Un modèle 
d'édition collaborative se compose d'un nombre connu ou inconnu de répliques. 
Une réplique est une copie du document partagé modifiable à n'importe quel 
instant. L'exécution d'une opération de modification se fait localement sur la 
réplique, qui la diffuse ensuite à l'ensemble des répliques. La notification de la 
modification qui contient l'opération de modification est communiquée de manière 
asynchrone. Les répliques distantes qui reçoivent cette notification exécutent 
alors l'opération localement. L'une des difficultés principale de ces systèmes 
est d'en maintenir la cohérence. Il existe différents facteurs pouvant 
pousser un système à ne plus être cohérent :
\begin{itemize}
	\item \textbf{La divergence.} Les opérations peuvent arriver sur plusieurs sites 
	dans différents ordres, provoquant des résultats dissemblables sur chaque réplique 
	participant à l'édition collaborative - à moins que les opérations ne soient 
	commutatives (ce qui est rarement le cas). Dans les cas d'application où la 
	cohérence du résultat final est nécessaire, la divergence doit être évitée. C'est possible grâce à l'utilisation d'un protocole de 
	sérialisation.
	\item \textbf{La violation de la causalité.} Le fait que la latence dans une 
	communication 
	soit intrinsèquement non-déterministe peut conduire le système à une violation de 
	la causalité : les opérations peuvent arriver et être exécutées dans un ordre 
	qui ne respecte pas l'ordre causal.
	\item \textbf{La violation de l'intention.} Suite à la génération d'opérations 
	concurrentes, 
	l'effet actuel d'une opération au moment de son exécution peut être différent de 
	l'intention de cette opération au moment de sa génération.
\end{itemize}
%TODO exemple dpour chacun des critères
Ces trois facteurs d'incohérence sont indépendants, notamment en ce qui concerne la violation de l'intention et le problème de 
divergence qui sont deux problèmes d'incohérence de différente nature : 
le premier facteur peut toujours être résolu en utilisant un protocole de sérialisation 
alors que le dernier ne peut utiliser un protocole de sérialisation fixé si les 
opérations sont exécutées sous leurs formes originelles \cite{Sun1997}. 
Le modèle \gls{CCI} proposé par Sun et al. \cite{Sun1998} propose un modèle de 
cohérence pour les systèmes d’édition collaborative qui s’adresse à ces facteurs 
afin d'assurer « la préservation 
de la convergence, de la causalité et la préservation de l’intention » 
\cite{Sun1998}. Une majorité de travaux se sont intéressés à ces propriétés sur 
des documents textuels \cite{Weiss2010}. 

Cette thèse s’intéresse précisément à l’édition collaborative d’espace de travail en 
3D. 
Sont considérées ici les opérations d’ajout et de suppression d’éléments 3D dans un espace partagé, 
ainsi que les modifications de haut niveau sur ces éléments 3D telles que la 
translation, la rotation et l’homothétie.
%Ces différents facteurs ont fait l'objet de 
%nombreux travaux au sein de la communauté \gls{CSCW}. 

%\subsubsection{Les systèmes d'édition collaboratifs}
%
%\paragraph{Modèle d'édition collaborative}
%
%\subsection{La collaboration 3D en accord avec l'évolution du web}
%
%\subsubsection{Introduction}
%\subsubsection{Le web et le P2P : WebRTC}
%\subsubsection{Le web et la 3D :  WebGL}

\subsection{Les architectures orientées événements pour la collaboration}

Un événement est un élément omniprésent de la vie. Le terme \textit{événement} 
existe dans presque tous les champs en science, avec différents sens. Le but de 
cette section est de répertorier les différentes notions d'événements. Pour cela, 
différentes descriptions du terme sont proposées ainsi que leur notion 
liée à la littérature informatique. \jp{pourquoi ne pas mettre directement la définition 
	qui nous intéresse ?}

Dans la littérature, une variété de définitions du terme événement existe. 
Habituellement, un événement est considéré comme quelque chose qui 
\og se produit\fg{}, en particulier quelque chose d'inhabituel ou d'important. 
Cependant la plupart des travaux de la littérature évitent de définir le sens précis 
des événements en n'indiquant pas le champ d'application dans lequel ils sont 
utilisés. 
Il existe trois entrées pour le terme événement dans le dictionnaire. 
La première entrée se réfère à la physique (dans le cadre de la théorie 
de la relativité), où un événement est <<~un phénomène considéré comme 
localisé 
et instantané, survenant en un point et un instant bien déterminés~>>. La seconde 
se réfère à la théorie des probabilités, indiquant qu'un événement est <<~la partie 
d'un univers $\Omega$ réalisée quand l'une des éventualités la composant se 
réalise~>>. La troisième définition se rapporte à la psychologie impliquant <<~tout 
ce qui est capable de modifier la réalité interne d'un sujet (fait extérieur, 
représentation, etc.)~>>. Cette dernière définition est très proche de ce que l'on 
retrouve en informatique, comme les changements d'état ou les actions entraînant 
certaines conséquences. Dans une application, il peut être important de 
s'intéresser au déplacement d'un objet de quelques unités (\textit{object moved 
	event}) ou d'apprendre qu'un utilisateur est passé de déconnecté à connecté 
(\textit{status changed}). On s'applique donc à identifier ce qui modifie l'état 
intrinsèque d'un objet et sa représentation. Bien que ces différentes descriptions 
permettent d'avoir une compréhension générale du terme événement, chaque 
sous-discipline de l'informatique comprend 
ses propres associations avec le terme événement.

L'implantation d'une architecture orientée événements (de l'anglais \gls{EDA}) 
nécessite l'instanciation d'une 
architecture abstraite, 
le positionnement des composants sur des machines, ainsi que des protocoles 
pour subvenir à l'interaction, en utilisant des technologies et des produits 
spécifiques. Une telle instance est appelée architecture système. De ce fait, les 
architectures de systèmes distribués orientés événements doivent 
répondre aux exigences des utilisateurs et aux problèmes liés à la nature des 
plateformes et applications. 
Le passage à l'échelle (nombre de d'utilisateurs, ressources distribuées sur de 
grandes zones géographiques) génère un grand nombre d'événements qui doivent 
être traités de façon efficace. 

Les systèmes complexes distribués sont construits à partir de collections 
couplées de manière dite \og lâche\fg{} (\textit{loosely coupled}) \footnote{Un 
	couplage lâche indique que les 
	composants échangent peu d'informations. Contrairement à un couplage fort, où 
	les 
	composants ont une faible indépendance fonctionnelle et sont donc peu 
	réutilisables, le couplage faible s'appuie sur l'établissement d'un protocole 
	d'échange faisant le moins d'a priori sur les composants. Cela permet de fixer 
	un 
	cadre d'interaction entre les composants.}, technologiquement neutre 
et indépendante de la localisation des services. 
Le développement d'applications dirigées par les événements est un challenge 
tripartite : la production d'événements, le traitement des événements et la 
consommation des événements \cite{Chandy2011}.
Cristea et al. \cite{Cristea2011} présentent un aperçu des architectures distribuées 
pour les systèmes orientés événements. 
Cette approche est utilisée dans de nombreuses applications réactives. Souvent 
appliquées à la finance ou aux systèmes logistiques, de telles solutions peuvent 
également intégrer les besoins de plateformes distribuées à grande échelle, 
comme les applications web et le travail collaboratif. 



\subsubsection{Sensibilisation et perception du groupe}
Le travail collaboratif peut également être supporté par des modèles orientés 
événements prenant en compte la sensibilisation au groupe (\textit{awareness}) 
dans des activités. Ces systèmes sont présents dans la littérature depuis les 
prémices de la technologie web \cite{Bentley1997,Steinfield1999,You2001}. Ces 
propositions ont commencé par approcher la sensibilisation à l'espace de travail 
dans le but d'informer les utilisateurs des changements se produisant dans 
l'espace 
de travail partagé. 
Des travaux plus récents se sont concentrés sur des 
nouveaux paradigmes comme les systèmes \gls{P2P} pour proposer des 
\textit{groupware} décentralisés ubiquitaires et sensibles à l'environnement. La 
sensibilisation au sein du groupe n'est pas seulement désignée 
pour notifier les utilisateurs ; son but est également d'aider dans les processus de 
groupes pour éviter les problèmes. Ces derniers peuvent prendre différentes 
formes comme :  l'inefficacité due à une information limitée et la fourniture d'un 
système de communication; la disponibilité d'informations superflues; la difficulté 
d'extraire l'information pour surveiller ou faire des rapports; l'utilisation des 
données issues des ressources produites par le groupe pour améliorer sa 
perception des données disponibles. Par exemple, Xhafa et Poulovassilis 
\cite{Xhafa2010} exposent une approche distribuée basée événements pour gérer 
des collecticiels (\textit{groupware}) en \gls{P2P}. 
\improve{ca veut rien dire}
Leur méthode permet de développer des applications de collecticiels spécifiques  
sur les opérations et les services primitifs liés à la sensibilisation. Ces derniers 
sont directement implémentés dans l'intergiciel \gls{P2P}. Le modèle propose 
différentes approches dépendant de la plateforme (web ou mobile) avec une 
granularité différente de l'information 
d'\textit{awareness} dont :

%Dans les collecticiels événementiels P2P, les conditions à intégrer dans le 
%développement de l'intergiciel P2P, pour proposer une sensibilisation au 
%groupe  -- activité, process, contexte, communication, disponibilité -- sont 
%nombreuses \cite{Xhafa2010} : 
\begin{description}
	\item[La sensibilisation distribuée] A l'inverse d'une approche centralisée, où 
	l'information de sensibilisation est gérée par un serveur central, dans un 
	système décentralisé, l'information sur laquelle la sensibilisation est construite 
	est répartie sur les sites des différents pairs. Dans les approches basées 
	serveur, le traitement des événements est effectué sur le serveur qui stocke 
	les événements ainsi que la base de données contenant l'historique et fournit 
	l'interface de requête pour extraire les informations d'intérêt. Dans le collecticiel 
	\gls{P2P}, le stockage, le traitement et les requêtes d'événements doivent être 
	faits 
	de manière répartie.
	
	\item[La sensibilisation à la dynamicité des événements] Les systèmes 
	\gls{P2P} 
	sont dynamiques par nature (\textit{join / leave} des pairs). La fonctionnalité de 
	sensibilisation doit prendre en compte cette dynamicité. En effet, la 
	synchronisation et la cohérence de l'information sur les différents sites sont 
	cruciales et plus difficiles à mettre en place que dans un système 
	client-serveur. La sensibilisation sous contrainte de dynamicité du réseau doit 
	intégrer des mécanismes de propagation et de réplication fournissant le contenu 
	au groupe. 
	
	\item[La généricité des événements] Un système basé événements manipule 
	plusieurs types d'événements. C'est pourquoi il est nécessaire que les 
	événements soient le plus génériques possibles pour faciliter la construction de 
	requêtes à travers un système lâchement couplé.
	
	\item[Les mécanismes à empreinte mémoire réduite] L'utilisation de 
	mécanismes à empreinte mémoire réduite est indispensable dans le but de 
	réduire la surcharge causée par la génération d'événements, le traitement des 
	événements et les notifications, ainsi que pour permettre le support de la 
	sensibilisation par des pairs qui possèdent des capacités limités.
	
\end{description}
%A parallel grid-based implementation for real-time processing of event log data of 
%collaborative applications
%\cite{Xhafa2010a}
%
%Using mobile-based complex event processing to realize collaborative remote 
%person monitoring
%\cite{Stojanovic2014}
%\info{Xhafa2010 parle aussi des mobiles}

Brown et al. \cite{Brown2003} proposent un mécanisme de distribution de 
données basé événements. Le \textit{Battlefield Augmented Reality System} 
(BARS) se situe dans le contexte de la collaboration sur mobile en réalité 
augmentée et environnement virtuel. Cet environnement a besoin de conserver 
une information cohérente au cours du temps, qui plus est de rendre compte de la 
situation (\textit{situation awareness}) et de permettre la coordination d'équipe sur 
mobile entre les utilisateurs. Ils définissent \textit{situation awareness} comme le 
fait que <<~chaque utilisateur doit obtenir une meilleure compréhension de 
l'environnement~>>. Pour cela, ils se placent dans un cadre où il est nécessaire 
de gérer plusieurs utilisateurs avec des connexions différentes (bas débit et haut 
débit) avec des connexions au réseau qui ne sont pas fiables et une réplication 
partielle des données pour minimiser les événements non voulus.\info{consistency cohérence conflits} L'analyse des applications utilisant des 
systèmes orientés événements explique 
qu'ils sont obligés de faire des choix spécifiques au métier lié à l'application 
\cite{Hinze2009}. Par exemple, dans les applications de jeux vidéo, la distribution 
des événements est souvent liée à un mécanisme \gls{PubSub} centralisé qui 
modifie les souscriptions au fur et à mesure que les joueurs se déplacent dans 
l'espace. De plus, les événements dans un jeu doivent être protégés contre 
l'altération pour éviter la triche ou la dissémination d'événements faux. Cela 
s'applique également aux données d'un \gls{EVC3D} destiné à un usage industriel 
(données confidentielles ou critiques).

\subsubsection{Intégration des règles métiers}
Comme les bureaux d'études en ingénierie et en architecture travaillent sur des 
projets (visualisation \gls{CAO}, \gls{BIM}, gestion et arrangement d'espaces 
architecturaux), la collaboration de professionnels venant de milieux différents 
avec des compétences et connaissances variées doit pouvoir s'accomplir autour d'un 
outil qui connaît leur langage, leurs contraintes - leur expertise. En effet, les 
modifications de modèles en \gls{3D} doivent être revues par des gestionnaires de 
projet, des clients et les intervenants impliqués qui peuvent à leur tour suggérer 
des modifications sur la conception. Tout cela doit se faire en accord avec des 
contraintes métiers transparentes.
Les règles métier doivent donc apparaître dès la conception pour être 
intégrées tôt dans la modélisation 3D. 

Les architectures orientées 
événements sont bien adaptées pour intégrer dans la description des 
événements plusieurs aspects liés au métier. 
Cette sémantique porte avec elle les connaissances des manipulations expertes.
Un aspect majeur de cette thèse repose sur l'intégration du métier dans le 
processus de la manipulation et de la visualisation des objets 3D. 

À l'ajout d'information sémantique sur les données, les \glspl{EDA} enrichissent 
les applications de la possibilité 
d'implantation d'outils de surveillance de flux métiers pour l'observation en temps 
réel ou l'analyse a posteriori de ces données pour fixer des objectifs ou repérer 
des 
problèmes de performance dans la collaboration par exemple.
%Toutes ces entités ont besoin d'être capables de 
%charger les ressources pour pouvoir les inspecter et les analyser. Ces 
%contraintes 
%se retrouvent dans le domaine du \gls{BIM}, l'architecture, l'héritage culturel, ou 
%plus généralement dans des milieux transdisciplinaires concentrés sur la 
%\gls{3D}. 
%
%L'évolution de ces ressources passent par une visualisation et une manipulation 
%collaborative efficace en terme de chargement de ressources et de transmission 
%des mises à jour.
\section{Problématique}


Cette thèse se situe à l'interface de trois champs de recherches (Figure 
\ref{fig:problematique}). Le premier concerne les environnements de modélisation 
3D. Souvent commerciaux (Clara.io, OnShape, Verold Studio), les modeleurs 
utilisent des technologies supportées par les navigateurs web qui leur permettent 
d'être disponibles sur la plupart des plateformes. 
Cependant ces dernières reposent sur une gestion centralisée des données qui rend 
les utilisateurs très dépendants de la disponibilité de ces plateformes et d'une 
connexion internet pour la distribution des informations. 
Mises à part quelques exceptions, les fonctionnalités collaboratives 
sont souvent présentées comme mineures. Même si le \og partage\fg{} de la 
visualisation à la manière des \og réseaux sociaux\fg{} est assez courant, l'édition 
collaborative est souvent complexe à implémenter car les besoins sont nombreux. 
Parmi eux, on trouve tout d'abord le besoin d'avoir un système distribué 
conservant la cohérence des modifications de chacun des utilisateurs. Or 
le nombre d'utilisateurs ne doit pas affecter l'expérience utilisateur ; le système 
doit supporter le passage à l'échelle. 
Un nombre d'utilisateurs élevé implique la mise en place d'un système de 
distribution de données adapté. Ce système est en plus contraint par la dynamicité
des arrivées et départs des collaborateurs (\textit{churn}) au cours d'une session. 
La dynamicité, qui ne permet pas d'avoir de pouvoir compter sur un nombre fixe de 
ressources, doit s'accorder avec les besoins variables en termes de ressources. 
Chaque client est porteur de ressources rarement complètement exploitées lors 
d'épisodes collaboratifs.

La visualisation et la manipulation collaborative d'objets \gls{3D} sont sujettes à 
plusieurs problématiques telles que la cohérence des ressources manipulées au 
cours du temps. Dans ce contexte, il existe un fort besoin de gérer l'évolution des 
versions permettant la revue des modèles \gls{3D}. 
En s'appuyant sur les ressources à disposition, telles que les clients (navigateurs 
web) sur lesquels les utilisateurs manipulent les modèles \gls{3D}, le passage à 
l'échelle est plus flexible car les ressources augmentent avec le nombre de clients 
et procurant également plus d'autonomie aux utilisateurs utilisant leurs ressources 
propres.
L'architecture de communication et les contraintes liées aux architectures 
distribuées et décentralisées correspondent au second axe de recherche. 
Enfin, le dernier champ 
s'adresse à la partie que nous appellerons \og métier\fg{} qui se rapporte au 
domaine de la manipulation d'objets 3D, lequel inclut la 
gestion du cycle de vie des données 3D, de l'interaction utilisateur à leur stockage 
en passant par leur distribution. Ces aspects se rapportent principalement à 
l'historique, la traçabilité de l'information et l'expertise embarquée dans le système.

\begin{figure}[hbt]
	\centering
	\includegraphics[width=\columnwidth]{problematique.eps}
	\caption{Place de la contribution dans les différents champs de recherche}
	\label{fig:problematique}
\end{figure}

L'objectif de cette thèse est triple : 
\begin{enumerate*}[label=(\roman*)]
	\item extraire les informations liées aux règles métier pour l'affichage et la 
	manipulation d'objets \gls{3D} en collaboration nécessaires à la traçabilité de 
	l'information,
	\item repérer les principales problématiques de gestion des données sur le 
	réseau pour avoir une transmission efficace et transparente pour l'utilisateur,
	\item proposer un \gls{framework} pour un \gls{EVC3D} web intégrant ces 
	contraintes réseau, métier, et \gls{3D} dans un navigateur.
\end{enumerate*}
Dans le but d'atteindre ces objectifs, voici les cinq Questions de Recherche 
posées :
\begin{description}
	\item[QR 1] Quelle architecture réseau est la plus adaptée pour une gestion 
	efficace, robuste et temps réel des données \gls{3D} dans un environnement 
	web ?
	
	\item[QR 2] Quelle architecture logicielle confère une traçabilité des données 
	conforme aux règles métier liées à la manipulation d'objet \gls{3D} ? 
	
	\item[QR 3] Quels sont les mécanismes assurant à l'utilisateur d'être 
	autonome tout en ayant la possibilité de collaborer ?
	
	
	\item[QR 4] Comment faciliter l'implémentation d'un tel système en garantissant 
	le respect des règles métier liés à la manipulation d'objet \gls{3D}?
	
	\item[QR 5] Quelles sont les métriques (réseau, collaboration) permettant 
	d'évaluer un tel système de manière quantitative ? Qualitative ? %Comment les 
	%utilisateurs recoivent la chose....
	
\end{description}

\section{Contributions}

La principale contribution de cette thèse est la définition et l'implémentation d'un 
ensemble de pratiques dans la gestion des données pour la manipulation d'objets 
3D de manière collaborative sur web. Cela est très utile pour la gestion de 
l'historique d'une scène \gls{3D} en utilisant des contenus \gls{3D} de différentes 
natures. Dans ce but, j'ai choisi le paradigme événementiel et je l'ai 
intégré à mon modèle de collaboration grâce à l'utilisation de solutions 
préexistantes pour le partage et la diffusion de contenus \gls{3D} dans le cadre 
d'applications collaboratives sur le web.

En constatant le manque de processus et d'outils qui pourraient réellement supporter la modélisation collaborative \gls{3D} sur le web intégrant l'identification de problèmes spécifiques et leur formulation liés a également été importante. 
La recherche, cependant, introduit de nouveaux concepts dans le domaine de la gestion de 
données \gls{3D}. Les contributions de cette thèse peuvent être résumées comme 
suit.


\subsection{Contributions théoriques}

\begin{itemize}
	\item \improve{réécrire + introduire ref chapitre}Proposition d'un système de 
	gestion et de visualisation de contenu \gls{3D} avec un historique non linéaire 
	orienté événements.
	\item Définition d'une architecture hybride (client serveur et \gls{P2P}) adapté à 
	la distribution de contenus \gls{3D} dans le cadre d'une collaboration sur web en 
	temps réel.
	\item Introduction des concepts d'événements métiers liées à la \gls{3D} 
	comme moyen d'interagir dans une scène  \gls{3D} multi-échelle
\end{itemize}
\subsection{Contributions pratiques}
\begin{itemize}
	\item \improve{réécrire + introduire ref chapitre}Définition d'une API ouverte et 
	de l'application cliente utilisation les principes et les technologies du web.
	\item Implémentation d'un prototype 3DEvent Client et de son évaluation 
	utilisateur. 
	\item Implémentation d'un prototype 3DEvent Architecture et de son évaluation.
	
\end{itemize}

%\bibentry{Desprat2017}

\section{Organisation du manuscrit}

%\section*{Organisation du manuscrit}
La \gls{3D} et les environnements virtuels collaboratifs \gls{3D} sont 
de plus en plus présents dans l'industrie de la \gls{CAO}, ce qui rend nécessaires de nouveaux 
modèles, outils et méthodes facilitant la mobilité et l'autonomie des utilisateurs. Le contenu multimédia \gls{3D} est également devenu omniprésent dans notre quotidien. De ce fait, développer 
de nouveaux outils pour créer manipuler et partager du contenu \gls{3D} est crucial.
Le chapitre \info{ref chapitre} présente une architecture de communication hybride 
permettant de faciliter la transmission des objets \gls{3D} en temps réel de 
manière transparente pour l'utilisateur. La création d'\gls{EVC3D} sur le web nécessite de nouvelles architectures tirant partie des nouveaux standards du 
\gls{W3C}. Puis le chapitre \info{ref cqrs} présente deux \info{2 contrib cqrs à 
	identifier} contributions reposant sur l'architecture cliente de 3DEvent pour 
manipuler des objets \gls{3D} avec une grande traçabilité et de manière autonome. 
%La première contribution propose 
Finalement, le chapitre\info{ref chap final} aborde le problème spécifique de 
la transmission du contenu \gls{3D} lors d'une session collaborative en temps réel. 
En considérant et développant un modèle orienté événements, 3DEvent dérive 
des 
stratégies pour délivrer et obtenir une scène 3D en continu (\textit{streaming} en 
anglais) en s'appuyant sur les différentes configurations matérielles, utilisateurs et 
réseaux.