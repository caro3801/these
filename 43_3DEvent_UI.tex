\subsection{Interface orientée tâche}
\label{sec:task-based-ui}
Dans le but de proposer une \gls{IU} proche des fonctionnalités métiers liées à la 
modélisation \gls{3D}, l'éditeur possède une interface orientée \og tâche\fg{}, en 
comparaison avec des \gls{IU} \gls{CRUD}. En effet, les \gls{IU} \gls{CRUD} 
réduisent la sémantique métier du domaine d'application à la création, la lecture, la 
mise à jour et la suppression, omettant toutes les subtilités que peuvent dégager 
ces actions en perdant l'intention de l'utilisateur dès le niveau de l'interface. Une 
interface orientée tâche a tendance à s'attarder sur toutes les nuances que le 
domaine possède en caractérisant chaque action sans subir d'effet de 
simplification. Cette proximité avec le métier permet de calquer directement 
l'interface du modèle événementiel sur l'\gls{IU} et de guider l'utilisateur dans ses 
activités. L'utilisabilité, qualité de l'expérience utilisateur fournit par un système 
pour réaliser une tâche, est alors maximisée en terme d'efficacité, d'efficience et 
de satisfaction. 
Ce type d'\gls{IHM} s'organise autour de cas d'utilisation. Cela permet, 
d'une part, de présenter clairement les 
actions (\og ajouter une géométrie à la 
bibliothèque à partir d'un fichier\fg{} plutôt que \og téléverser un fichier\fg{}) : 
l'intention est clairement définie. D'autre part, lorsque l'utilisateur s'apprête à faire 
une action, seules les informations utiles sont affichées. Enfin, l'application fournit 
simplement l'information dans le contexte où elle doit être présentée, évitant à 
l'utilisateur d'aller la chercher ailleurs.
L'\gls{IU} devient alors une couche de l'application qui nécessite d'agréger, croiser 
et filtrer des données. La dénormalisation proposée par \gls{CQRS} remédie à ce 
besoin dans le cadre de la consultation de données. 


\subsubsection{Présentation de l'interface}


%\begin{figure}[h!]
%	\centering
%	\begingroup
%	
%	\subfloat[Rotation (vue \gls{3D}) et outils de manipulation d'objet 
%	\gls{3D} 
%	(panneau 
%	
%latéral)]{\includegraphics[width=0.75\textwidth]{eps/2rotatedetail.eps}\label{fig:ui2}}\hfill
%	
%	\subfloat[Translation (vue \gls{3D}) et visualisation de l'historique 
%	(panneau 
%	
%latéral)]{\includegraphics[width=0.75\textwidth]{eps/1translatehisto.eps}\label{fig:ui1}}\hfill
%	
%	\subfloat[Mise à l'échelle (vue \gls{3D}) et liste des collaborateurs 
%	(panneau 
%	
%latéral)]{\includegraphics[width=0.75\textwidth]{eps/3scalecollab.eps}\label{fig:ui3}}\hfill
%	
%	\endgroup
%	\caption{Interface utilisateur pendant une session collaborative (trois 
%personnes)}
%	\label{fig:screenshots}
%\end{figure}
Lorsqu'un utilisateur se connecte à une scène, il a accès à une interface web 
(dans un navigateur) qui représente l'espace de travail collaboratif et qui lui 
permettant 
d'utiliser différentes fonctionnalités. Les deux modalités d'interaction sont le clavier 
et la souris\info{est ce qu'on parle de mobile?}. Le premier niveau de cette 
interface est scindée en deux panneaux~: 
\begin{enumerate}
	\item L'espace \gls{3D} consacré à la visualisation des objets et à leur 
	manipulation 
	dans l'environnement \gls{3D}~;
	\item La barre d'outils qui contient trois onglets~:~
	\begin{enumerate}
		\item "Scene" contient tous les détails de la scène et des maillages qu'elle 
		inclue~; 
		\item "Collaboration" fournit les informations liées à la collaboration~;
		\item "History" liste tous les événements qui ont eut lieu dans la scène et 
		leurs  détails. 
	\end{enumerate}
\end{enumerate}

\begin{figure}[]
	\centering
	\begingroup
	
	\subfloat[Onglet \og outils de manipulation sur la 
	scène\fg{}]{\includegraphics[width=0.38\textwidth]{eps/scenecontrol.eps}\label{fig:uicontrol}}
	\hfill
	\subfloat[Onglet \og collaboration\fg{}]
	{\includegraphics[width=0.27\textwidth]{eps/collaboration.eps}\label{fig:uicollab}} 
	\hfill
	\subfloat[Onglet \og 
	historique\fg{}]{\includegraphics[width=0.32\textwidth]{eps/history.eps}\label{fig:uihisotry}}
	
	\endgroup
	\caption{Onglets du panneau latéral de l'interface}
	\label{fig:uipanneau}
\end{figure}

La Figure \ref{fig:uipanneau} montre quelques captures d'écran durant une 
session collaborative sur le modèle Rotor. L'onglet "Scene" (Figure 
\ref{fig:uicontrol}) possède un bloc contenant les détails 
d'un 
maillage en cours de 
sélection. Cela permet d'avoir la description des propriétés de l'objet sélectionné et 
une manipulation de ses paramètres (position, rotation et mise à l'échelle) plus 
précise que via l'espace \gls{3D} avec le cliqué / déplacé. "Scene" intègre 
également un espace réservé aux géométries disponibles dans la scène appelé 
Bibliothèque (de géométries).

L'onglet "Collaboration" (Figure \ref{fig:uicollab}) présente la liste des 
collaborateurs qui 
participent à la 
scène. Chacun d'eux est décrit par son nom, son état  (connecté ou déconnecté) 
et son rôle (administrateur, éditeur, lecteur ou autre\footnote{Un rôle peut être 
	défini par le biais du \gls{framework} 3DEvent}). En cliquant sur un élément de 
	la 
liste, l'utilisateur accède au dernier point de vue dans l'espace \gls{3D} connu du 
collaborateur représenté.

L'onglet "History" (Figure \ref{fig:uihisotry}) liste tous les événements passés dans 
la 
scène en fournissant 
l'accès à leur détail. Pour chaque événement, le système est capable de montrer 
dans l'espace \gls{3D} la différence entre l'état  après l'événement cliqué $state_x$ 
et l'état courant $state_n$. L'utilisateur peut à partir de cette visualisation choisir 
de \og revenir en arrière\fg{} sans perdre les données entre $state_n$ et $state_x$ 
car dans le système présenté ici, cela s'effectue par compensation (cf 
Event-Sourcing 
Section \todo{ref section})\improve{annulation d'un événement ou juste ES}.

Dans chaque onglet se trouvent différents blocs \gls{HTML}, avec des 
comportements spécifiques à un agrégat et injectés dynamiquement. Ces blocs 
correspondent à des parties de l'interface (vues) dans le prototype.

Parmi les vues disponibles dans le système, une grande partie est dédiée à 
l'\gls{IU} de l'application web pour le cas d'utilisation de la modélisation 3D. 
D'autres vues sont disponibles pour un autre type d'utilisation destinée à 
l'observation des comportements des utilisateurs, élément primordiale dans 
cette expérimentation.



\subsubsection{Exemple d'interaction}
La Figure \ref{fig:cqrs-example} décrit la façon dont le système traite l'exécution 
d'une commande de translation déclenchée par l'utilisateur et comment cette 
information est diffusée aux collaborateurs\footnote{Pour que l'exemple 
	fonctionne, la scène, la géométrie du cube et le maillage \textit{cube1} doivent 
	avoir été créés en amont.}.
Dans l'étape (a), la commande déclenchée par l'utilisateur s'adresse à l'agrégat 
$cube1$ et contient les paramètres de la translation (vecteur x, y, z). L'agrégat, 
qui 
modélise le domaine d'un maillage, génère l'événement de translation $e1$ (étape 
(b)) si tout est valide d'un point de vue métier. L'événement $e1$ est ensuite 
passé à l'Event Store. 
Le composant responsable de la détection de conflit permet au développeur 
d'implémenter ses propres règles de résolution de conflit. Le composant déclenche 
une exception lorsque le numéro de version reçu et le numéro de version courant 
de l'agrégat sont identiques (Figure \ref{fig:cqrs-example} étape (c)). Selon les 
règles métiers définies et les exceptions liées à la cohérence, l'événement peut 
être 
rejeté. Ce traitement peut être à l'origine de la génération de nouveaux 
événements.


\begin{figure}[]
	\centering
	\includegraphics[width=\columnwidth]{eps/example10.eps}
	\caption[Flux de la collaboration dans le framework 3DEvent entre 3 
	utilisateurs]{Exemple d'édition collaborative où User A est connecté à User  B, 
		lui 
		même connecté à User C. Le cycle montre les différentes étapes du 
		déclenchement: la commande, la 
		génération 
		de l'événement, la 
		synchronisation du journal d'événements, l'impact sur le rendu des autres 
		utilisateurs pour une translation sur un cube et le rendu 
		visuel.}\label{fig:cqrs-example}
\end{figure}

\subsection{Sélection fantôme}
Les interactions utilisateurs doivent être adaptées à la collaboration et aux 
manipulations à effectuer. Pour cela, l'éditeur 3DEvent introduit la fonctionnalité de 
sélection \og fantôme\fg{}. Lorsqu'un utilisateur souhaite sélectionner un objet de 
la scène, l'objet original ($O_o$) est 
cloné et devient l'objet fantôme ($O_f$). $O_f$ conservent les mêmes propriétés, 
représenté avec de la 
transparence d'où le terme \og fantôme\fg{}. 
L'objet $O_f$ prend alors le focus de sélection pour que l'utilisateur le manipule à 
la place de l'objet $O_o$. 
Lorsque l'utilisateur relâche $O_f$, alors la modification intentée s'applique sur 
$O_o$ avec le principe du \textit{Last Write Wins} (le dernier gagne).
La Figure \ref{fig:ghostselection} 
représente la sélection fantôme lors de la translation du corps du rotor par 
l'utilisateur Foo : c'est l'objet transparent qui est manipulé alors que l'objet opaque 
représente sa position originale. 
En différenciant l'actuel objet que l'utilisateur souhaite sélectionné ($O_{o}$) de 
celui manipulé ($O_f$), l'interaction est mise en valeur sous quatre angles :
\begin{itemize}
	\item l'ergonomie dans l'environnement \gls{3D} :
	$O_f$ est un objet temporaire qui permet à l'utilisateur
	d'avoir une visualisation de l'objet en cours de manipulation tout en 
	conservant le dernier état de $O_o$ visible. 
	$O_o$ peut être considéré comme un point de repère visuel pour l'utilisateur 
	lorsqu'il effectue sa manipulation. 
	L'$O_f$ a aussi un rôle d'intermédiaire entre l'utilisateur et la 
	finalité de l'interaction en donnant un support visuel à sa réflexion experte.
	Grâce à $O_f$, l'utilisateur peut également révoquer sa manipulation en 
	cours sans avoir eu d'impact sur $O_o$ en évitant des actions inutiles (faire 
	l'action 
	puis la défaire) pour le métier et coûteuses pour le réseau.
	
	\item la collaboration : si un collaborateur effectue une modification 
	à destination du même $O_o$ alors la représentation de $O_o$ chez 
	l'utilisateur est également modifiée. $O_f$ par contre ne subit pas d'impact ; 
	l'utilisateur peut continuer sa manipulation et~/~ou l'ajuster en fonction des 
	nouvelles informations liées à $O_o$ ou même révoquer sa manipulation 
	en cours si cela lui convient.
	
	\item le métier : seules les manipulations menées à terme sont 
	considérées comme des commandes. Cela évite d'avoir des événements qui ne 
	sont pas pertinents pour le métier dans le journal d'événements (comme lorsque 
	l'utilisateur change d'idée lors de l'interaction ou
	suite à une intervention concurrente). L'utilisateur n'a un impact sur l'application 
	que lorsqu'une modification métier est réalisée.
	
	\item le réseau : l'information importante à faire transité est l'événement 
	correspondant à la modification métier pas toutes les positions intermédiaires 
	même si intuitivement l'idée de temps réel pourrait conduire à cette solution. La 
	quantité de messages produite surchargerai à la fois le réseau et le fil 
	d'exécution principale de l'application. En effet, \gls{WebRTC} a l'inconvénient 
	pour le 
	moment de ne pas pouvoir s'exécuter dans un \textit{Web Worker} (fil 
	d'exécution 
	parallèle en JavaScript). Cette solution imposerai des 
	latences réseau et d'\gls{IU} qui affecteraient gravement l'expérience utilisateur
	sans apporter d'informations supplémentaires à l'aspect métier de la 
	collaboration. 
\end{itemize}


\begin{figure}[ht]
	\centering
	\includegraphics[trim={0 0 0cm 0}, clip, 
	width=0.8\columnwidth]{eps/1translatehisto.eps}
	\caption{Illustration de la sélection fantôme dans l'environnement 3D}{
		Les boîtes englobantes représentent la sélection des différents 
		collaborateurs pendant la session.
	}
	\label{fig:ghostselection}
\end{figure}

\paragraph{Projections}
\todo{refomuler}
chaque partie de l’interface est liée à une proj de la bdd
les actions user et les actions des autres users passent par le meme cycle
pas de diff de prise en compte des evnts de l’interaction user ou de la couche 
reseau
en CS : les actions users -> actions -> recup info
action push du servuer qui peuvent etre gerees de maniere diff
\subsection{Bilan}

\todo{refomuler}
L'application 3DEvent repose sur les principes et les 
technologies du web pour permettre de visualiser et manipuler des objets \gls{3D} 
de 
manière 
collaborative en temps-réel.
\paragraph{Testabilité}
\section{Conclusion du chapitre}