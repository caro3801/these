%!TEX root = main.tex

\thispagestyle{empty}
\section*{Remerciements}
\addcontentsline{toc}{chapter}{Remerciements}
\adjustmtc

Je tiens tout d'abord à remercier les membres du jury pour avoir accepter de 
participer à ce comité, en particulier, les rapporteurs pour leurs commentaires 
pertinents et constructifs lors des pré-rapports.
\\

Ensuite, je tiens à remercier Hervé Luga, mon directeur de thèse, et Jean-Pierre 
Jessel, mon co-encadrant de thèse, pour l'encadrement qu'ils ont exercé durant 
ces trois ans (et onze mois), en m'accordant une grande confiance dans le choix 
et la réalisation du sujet de cette thèse. 
Hervé, qui depuis le stage de Master m'a suivie et donner la possibilité d'ouvrir 
ma recherche à d'autre champs disciplinaires. Jean-Pierre, qui a su distiller 
opportunités et ressources au cours de ma thèse pour voir au delà de la vie de 
laboratoire.
\\

Je crois avoir été un objet test de l'université fédérale de Toulouse entre mon 
université d'inscription et d'enseignement (UT2J), celle où se situait mon bureau 
(UT1-C) et celle des projets et des séminaires (UPS). Cela m'a permis de 
rencontrer de nombreuses personnes, collègues et amis formidables que je tiens 
également à remercier pour leur patience ainsi que les longs et passionnés 
échanges que 
nous avons eu. Je pense en particulier à la ME310, où l'ambiance chaleureuse 
(surtout l'été),à 
été le berceau de belles amitiés.\\

Les ami·e·s. Toi qui a relu un papier la veille pour le lendemain ou encore cette 
thèse sans n'y rien connaître, toi qui ne m'a pas posé la question \og alors c'est 
prévu pour quand la soutenance ?\fg{}, merci. 
\\

Papounet, toujours présent même à distance. Les bons petits plats, les 
petits sms, les voyages\dots~Toujours sensibles parfois 
silencieux, parfois musicaux ces encouragements m'ont toujours donné des 
appuis sur lesquels me reposer. Merci.\\

Julia, sista 
d'amour, la motivation derrière cette thèse te doit beaucoup et moi aussi. En 
partageant l'expérience de thèse de l'une et l'autre nous avons appris beaucoup 
sur chacune. J'ai trouvé dans ma sœur chérie et adorée une femme brillante, 
disponible et spontanée. Merci.
\\

Benoît, je pense que tu n'envisageais pas cette aventure de cette manière. Moi 
non plus. Toujours là, 
patient, à m'écouter, me proposer des idées et à me soutenir. Ces 
moments de thèses partagés ensemble sont des morceaux de vie uniques que je 
suis heureuse d'avoir partagés avec toi. Merci.
\clearpage
\pagebreak

\thispagestyle{empty}
\vspace*{\stretch{1}}
\begin{flushright}
\textit{À maman,}
\end{flushright}
\vspace*{\stretch{2}}

\pagebreak
