\subsection{De l'état à l'événement}
\label{sec:statetoevent}
Afin d'intégrer la notion d'historique dans une application web, plusieurs indicateurs 
sont nécessaires (version, granularité de l'historisation, type et taille des données 
manipulées). Cette section décrit le processus pour passer d'un système basé 
états à un système basé événements dans l'intérêt de réduire la somme des 
données transmises sans perdre d'information concernant la logique métier.
\paragraph{Description des variables}
Une scène $S$ contient l'ensemble des $n$ objets $x$ ($S =\{x_0,x_1,...,x_n\}$). 
Chaque élément de $S$ est associé à une taille calculée comme la somme des 
modifications $m$ selon la version $v$. 

Pour tout objet $x$ appartenant à la scène $S$ on associe une taille $w \in 
\mathbb{R}$. $w$ correspond à la taille totale des données transmises pour 
modifier l'objet à chaque version $v$ ($v$=0 correspond à la taille de l'objet à l'état 
original, i.e. l'\gls{EventStore} est vide). La taille totale de la scène, notée $Sw$, 
correspond à sommer la taille de tous ses objets (équation \ref{eq:sw}). Toutes les 
variables évoquées sont exprimées en octet pour avoir des termes avec des 
unités homogènes.

\begin{equation}
\label{eq:sw}
\text{Taille d'une scène $S$ : } S_w= \sum_{i=0}^{n}w_i
\end{equation}


\paragraph{Transmission par état}
Dans un système où à chaque modification on transmet l'état complet (équation 
\ref{eq:complet}), la taille des éléments transmis correspond à la somme de tous 
les états par lesquels l'objet est passé. Cette valeur va augmenter par paliers de 
tailles du même ordre de grandeur.
La transmission par état propose un historique issu d'une sauvegarde 
chronologique. Cependant elle est lourde voire redondante et n'offre pas 
d'information sur l'intention de l'utilisateur (quelle opération a permis d'arriver à cet 
état?).

\begin{equation}
\label{eq:complet}
\text{Par état : } w_i = \sum_{v=0}^{n}state_{v}
\end{equation}


\paragraph{Transmission par différentiel d'état}
\label{par:diff}
Dans un système où l'on transmet un différentiel d'état (équation 
\ref{eq:différentiel}), la différence entre deux états peut être difficile à exprimer (ex: 
MeshHisto \cite{Salvati2015})
car il s'agit d'un objet 3D dont les informations, stockées dans 
un fichier, ne sont pas linéaires. Le différentiel peut beaucoup varier jusqu'à 
atteindre une taille proche de l'état actuel si une transformation génère beaucoup 
de modifications par rapport à l'état précédent. A contrario, une modification sur la 
position d'un objet peut être très légère. La fonction représentant la taille des 
ressources transmises augmentera par palier avec au pire les propriétés de la 
proposition précédente (équation \ref{eq:complet}). Le différentiel d'état permet 
d'alléger la transmission par rapport à la proposition précédente, ne contenant 
cependant pas d'information sémantique sur ce qui est transmis et les opérations 
qui ont conduit à l'état d'un objet 3D de la scène.

\begin{equation}
\label{eq:différentiel}
\text{Par différentiel d'état : } w_i = state_0 + \sum_{v=1}^{n}(\underbrace{state_{v} 
- state_{v-1}}_{differential})
\end{equation}

\paragraph{Transmission par événement}
La transmission par événement (équation \ref{eq:evenement}) repose sur le 
principe du différentiel d'état ($differential$) exprimé ci-dessus (équation 
\ref{eq:différentiel}) en ajoutant la notion de type d'événement ($eventType$). Le 
type d'un événement indique le nom de la transformation à appliquer pour passer 
d'un état à l'autre. Un événement contient par nature une sémantique <<métier>> 
associée à chaque opération qui donne des indications sur l'intention de 
l'utilisateur. 

\begin{equation}
\label{eq:evenement}
\text{Par évènement : } w_i= state_0 + \sum_{v=1}^{n}(\underbrace{eventType + 
differential}_{event}) 
\end{equation}

Dans un \gls{EVC3D}, il y a des événements concernant à la fois la manipulation 
3D, la navigation et les éléments se rapportant à la collaboration (par exemple: 
prendre le point de vue d'un collaborateur). La taille d'un événement est variable 
(comme un différentiel d'état) avec un minimum contenant le type d'événement 
($eventType$) dû à sa structure type/paramètres. Cette proposition est un peu 
plus lourde à transmettre que la précédente mais ajoute une valeur sémantique à 
l'historique.
